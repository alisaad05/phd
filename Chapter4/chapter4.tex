\chapter{Macrosegregation with solidification shrinkage}
%\chaptermark{Nonlinear Temperature Solver}
\begin{nolinkcolors} 
\minitoc
\end{nolinkcolors}
\newpage

% ======================
\section{Introduction}
% ======================

Solidification shrinkage is, by definition, the effect of relative density change between the liquid and solid phases.
In general, it results in a progressive volume change during solidification, until the phase change has finished. 
The four stages in \cref{fig:real_ingot_stage_a,fig:real_ingot_stage_c,fig:real_ingot_stage_b,fig:real_ingot_stage_d} depict the volume change with 
respect to solidification time.
First, at the level of the first solid crust, near the local solidus temperature, the solid forms with a density smaller than 
the liquid's. This does not necessarily apply for all materials, but at least for steels it does. The subsequent volume difference 
tends to create voids with a big negative pressure, that needs to be compensated by a surrounding fluid. It thus drains 
the liquid metal in its direction (cf. \autoref{fig:real_ingot_stage_b}). As a direct result of the inward feeding flow, the ingot surface
tends to gradually deform in the feeding direction, forming the so-called \emph{shrinkage pipe}. Since the mass of the alloy and its 
chemical species is conserved, a density difference between the phases ($\rhol < \rhos \implies \frac{\rhol}{\rhos}<1$) eventually leads 
to a different overall volume ($V^s<V^l$) once solidification is complete, as confirm the following equations:
%------------
\begin{subequations}
\begin{align}
& \rhol V^l = \rhos V^s  \\ 
& V^s = \frac{\rhol}{\rhos} V^l
\end{align}
\end{subequations}
%------------
Solidification shrinkage is not the only factor responsible for volume decrease. Thermal shrinkage in both solid and liquid phases, as well 
as solutal shrinkage in the liquid phase are common causes in casting processes. Henceforth, we will focus on shrinkage due to phase change.

%=============
\begin{figure}[htbp]
\centering
%\begin{minipage}{.5\textwidth}
  \begin{subfigure}{0.3\textwidth}
    \centering
    \def\svgwidth{100pt}
	\import{Chapter4/Graphics/new/}{ingot_air_liq.pdf_tex}
	\caption{Initial state}
    \label{fig:real_ingot_stage_a}
  \end{subfigure}
  %\hfill
  \begin{subfigure}{0.3\textwidth}
    \centering
    \def\svgwidth{100pt}
	\import{Chapter4/Graphics/new/}{ingot_air_liq_mush_sol.pdf_tex} 
	\caption{Early intermediate state}
    \label{fig:real_ingot_stage_b}
  \end{subfigure}
%
\vskip\baselineskip
%======
%
  \begin{subfigure}{0.3\textwidth}
    \centering
    \def\svgwidth{100pt}
	\import{Chapter4/Graphics/new/}{ingot_air_mush_sol.pdf_tex}
	\caption{Late intermediate state}
    \label{fig:real_ingot_stage_c}
  \end{subfigure}
  %\hfill
  \begin{subfigure}{0.3\textwidth}
    \centering
    \def\svgwidth{100pt}
	\import{Chapter4/Graphics/new/}{ingot_air_sol.pdf_tex}
	\caption{Final state}
    \label{fig:real_ingot_stage_d}
  \end{subfigure}
  %
\caption{Schematic of the main cooling stages of an ingot against side and bottom mould walls (not shown)}
\label{fig:real_ingot_stages}
\end{figure}
\comment{ \textbf{Sutaria2012} talk about feeding paths, but more importantly they computed thermal shrinkage WITHOUT solving
NavierStokes equations. To predict the interface shape, they solve a LS transport with an imposed velocity given by Gada et Sharma 2009}

% ======================
\subsection*{Literature review}
% ======================
Talk and explain models in the literature that predict shrinkage (without or with macrosegregation): Beckermann, Wu ? \\
Show and comment the experiments that have been done: Hebditch and Hunt, Smacs Hachani  ...

%====================================================================
\section{FE model: Metal}
In this section, we start from a the monodomain finite element model presented in \cref{sec:monodomain} relevant to metal only, 
then present the essential assumptions and formulations that allow predicting solidification shrinkage in a Eulerian context.

\subsection{Mass Conservation}
% ======================
\subsubsection{Assumptions}
% ======================
\begin{itemize}
\itemsep0em
\item Two phases are considered, liquid $l$ and solid $s: \gl+\gs =1 $ % \implies \frac{dg^l}{dt} =- \frac{dg^s}{dt}$ 
\item The phase densities are constant but not equal: $ \rhol=cst_1 $ and $ \rhos=cst_2 $. Thermal and solutal expansion/contraction
is neglected
\item The solid phase is assumed static: $\vec{v^s}=\vec{0}$, which yields the following consequences:
\begin{enumerate} % \cancelto{0}{blablabla}
\itemsep0em
\item $ \avg{\vec{v}}= \gl \vec{v^l} + \cancel{\gs \vec{v}^s} = \gl \vec{v}^l $
\item $ \avg{\rho \vec{v}} = \gl \rhol \vec{v}^l + \cancel{\gs \rhol \vec{v}^s} = \gl \rhol \vec{v}^l $
\item $\nabvec \rhol = \nabvec \rhos = \vec{0}$
\end{enumerate}
\end{itemize}
% ======================
\subsubsection{Formulation}
% ======================
The mass balance equation averaged over the two phases, is expanded taking into account the aforementioned assumptions.
\begin{subequations}
\label{eq:shrinkage}
\begin{align}
& \frac{\partial \avg{\rho}}{\partial t} + \nabla \cdot \avg{\rho \vec{v}}  = 0 \\ 
& \frac{\partial }{\partial t} \brac{ \gl \rhol + \gs \rhos} + \nabla \cdot \brac{ \gl \rhol \vec{v}^l} = 0 \\ 
& \gl \cancel{\frac{\partial \rhol }{\partial t}} + \rhol \frac{\partial  \gl }{\partial t} 
	+ \gs \cancel{\frac{\partial \rhos }{\partial t}} + \rhos \frac{\partial  \gs }{\partial t} 
	+ \rhol \nabla \cdot \brac{ \gl \vec{v}^l} 
	+ \gl \vec{v}^l \cdot  \cancel{ \nabvec \rhol }	 = 0 \\
& \brac{ \rhol - \rhos } \frac{\partial  \gl }{\partial t} + \rhol \nabla \cdot \brac{ \gl \vec{v}^l}  = 0
\end{align}
\end{subequations}
\begin{align}
\label{eq:mass_balance}
 \boxed{\nabla \cdot \brac{ \gl \vec{v}^l} 
 	= \nabla \cdot \avg{\vec{v}^l} 
 	= \frac{\rho^l-\rho^s}{\rho^l} \frac{\partial  \gs }{\partial t}}
\end{align}
% ======================
\subsubsection{Discussion}
% ======================
With the assumptions of static solid phase and constant unequal phase densities, the average mass balance states that 
the divergence of the liquid velocity is proportional to the solidification rate, by a factor of density change, 
which results in a relative volume change. \Cref{eq:mass_balance} explains the flow due to shrinkage. In metallic alloys, the solid density is
usually greater than the liquid density, therefore the first term in the RHS is negative. As for the second term, if we
neglect remelting, then it'll be positive in the solidifying areas of the alloy. A negative divergence term in these areas, 
means that a liquid feeding is necessary to compensate for the density difference, hence acting as a flow driving force in the melt.
In the case of constant densities, we can easily deduce that the divergence term is null, and therefore no flow is induced
by solidification. Furthermore, additional terms should appear in the other conservation equations, balancing the volume 
change in the momentum, heat and species transport.

%====================================================================
\subsection{Energy Conservation}
% ======================
We have seen the averaged energy conservation equation in the case of two phases: 
a solid phase and an incompressible liquid phase. However, with the incorporation of
the shrinkage effect, new terms should appearin the advective-diffusive heat transfer equation. 
% ======================
\subsubsection{Assumptions}
% ======================
\begin{itemize}
\itemsep0em
\item The thermal conductivity is constant for both phases: $\avg{\kappa} = \ks = \kl= \kappa $ 
%\item consideration of a fixed solid ($ \vec{v}^s=\vec{0} $).
\item Consequence of the static solid phase: $\avg{\rho h \vec{v}} = \gl \rhol \hl \vl +  \cancel{\gs \rhos \hs \vs} = \gl \rhol \hl \vl$ 
\item The system's enthalpy may thermodynamically evolve with pressure, knowing that $h=e+\frac{p}{\rho}$, where $e$ is the internal energy and $p$ is the pressure. It infers that the heat transport
equation may contain a contribution attributed to volume compression/expansion:
\begin{align}
			 \frac{\partial p}{\partial t}+\nabla \cdot \brac{p \vec{v}}
			 = \frac{\partial p}{\partial t}+ p \nabla \cdot \vec{v} + \vec{v} \cdot \nabvec p 
\end{align}
In the literature, this contribution has been always neglected, even when accounting for solidification
shrinkage, owing to the small variations of pressure.
\item The heat generated by mechanical deformation, $\mathbb{S}:\dot{\varepsilon}$, is neglected
%\begin{enumerate}
%\item $\avg{\rho h}= \gl \rhol \hl + \gs \rhos \hs $
%\end{enumerate}
\end{itemize}
% ======================
\subsubsection{Formulation}
% ======================
The unknowns in the energy conservation are the average volumetric enthalpy $\avg{\rho h}$ and temperature $T$.The energy conservation equation writes:
\begin{subequations}
\begin{align}
	& \frac{\partial \avg{\rho h}}{\partial t} + \nabla \cdot \avg{\rho h \vec{v}} 
	= \nabla  \cdot \brac{\avg{\kappa} \nabvec T } \\
	& \frac{\partial \avg{\rho h}}{\partial t} + \nabla \cdot \brac{\gl \rhol \hl \vl}
	= \nabla  \cdot \brac{\kappa \nabvec T } \\ 
	& \frac{\partial \avg{\rho h}}{\partial t} 
		+ \rhol \hl  \nabla \cdot \vit
		+ \vit \cdot \nabvec \brac{\rhol \hl}
		= \nabla  \cdot \brac{\kappa \nabvec T } \\   
	& \frac{\partial \avg{\rho h}}{\partial t} 
		+ \cancel{\rhol} \hl  \frac{\rhol-\rhos}{\cancel{\rhol}} \frac{\partial  \gs }{\partial t}
		+ \vit \cdot \nabvec \brac{\rhol \hl}
		= \nabla  \cdot \brac{\kappa \nabvec T } \\ 
	& \frac{\partial \avg{\rho h}}{\partial t} 
		+ \brac{\rhol-\rhos} \hl \frac{\partial  \gs }{\partial t}
		+ \vit \cdot \nabvec \brac{\rhol \hl}
		= \nabla  \cdot \brac{\kappa \nabvec T }        
\end{align}
\end{subequations}
\begin{align}
\label{eq:energy_balance}
 \boxed{ \frac{\partial \avg{\rho h}}{\partial t} 
		+ \rhol \vit \cdot \nabvec \hl
		= \nabla  \cdot \brac{\kappa \nabvec T }
		+ \brac{\rhos-\rhol} \hl \frac{\partial  \gs}{\partial t}}
\end{align}

% ======================
\subsubsection{Discussion}
% ======================
In order to keep things simple, the term "enthalpy" will refer henceforth to "volume enthalpy",
otherwise, we will explicitly use the term "mass enthalpy". It is important to understand the 
meaning of the terms in equation \eqref{eq:energy_balance}.
The first term in the left-hand side is the temporal change in the system's average enthalpy,
i.e. a temporal change in the volume enthalpy of any of the phases in the course of solidification.
The second LHS term is a dot product between the superficial liquid velocity and the the gradient
of the liquid's enthalpy. Since phase densities are constant in our case, the gradient term reduces
to the liquid's mass enthalpy. If we consider a representative volume element (RVE) in the liquid
phase, far from the mushy zone, we can stipulate:
\begin{align}
\label{eq:gradient_liquid_enthalpy}
& \nabvec \hl = C_p^l \nabvec T
\end{align}
assuming that the phase mass specific heat, $ C_p^l $, is constant. Therefore, the liquid enthalpy
is advected in the case where the velocity vector is not orthogonal to the temperatre gradient.
The advection reaches its maximum when the two vectors have the same direction. Consider, for instance,
a filled ingot with a cooling flux applied to its bottom surface. If the density variation with temperature
were to be neglected, then the sole mechanical driving force in the melt is the density jump at the solid-liquid
interface ahead of the mushy zone. The temperature gradient in such a case is vertical upward, while the velocity
vector is in the opposite direction. The advective term writes:
\begin{align}
%\label{eq:}
& \rhol \vit \cdot \nabvec \hl = - \rhol C_p^l \norm{\vit} \norm{\nabvec T}
\end{align}
% FIGURE %%
\begin{figure}
\centering
\begin{subfigure}[h!]{0.3\textwidth}\centering % h! or H
	\def\svgwidth{100pt}
	\import{Chapter4/Graphics/}{Ingot_sl_1D_a.pdf_tex}
	\caption{Initial state}
	\label{fig:ingot_1d_a}
\end{subfigure}
\begin{subfigure}[h!]{0.3\textwidth}\centering % h! or H
	\centering
	\def\svgwidth{100pt}
	\import{Chapter4/Graphics/}{Ingot_sl_1D_c.pdf_tex}
	\caption{Solidification onset}
	\label{fig:ingot_1d_c}
\end{subfigure}
\begin{subfigure}[h!]{0.3\textwidth}\centering % h! or H
	\centering
	\def\svgwidth{100pt}
	\import{Chapter4/Graphics/}{Ingot_sl_1D_d.pdf_tex}
	\caption{Final state}
	\label{fig:ingot_1d_d}
\end{subfigure}
\caption{Effect of shrinkage flow on a solidifying ingot}
\end{figure}
%% FIGURE %%
We see that the second RHS term in equation \eqref{eq:energy_balance} acts as 
a heat source at the interface between the the phases, in this particular solidification
scenario. Another heat power (of unit $Wm^{-3}$) adds to the system within the mushy, 
that is the second term in the right-hand side of the same equation. This term is 
proportional to the solidification rate. Finally, the first RHS term accounts for thermal 
diffusion within the phases.
\newline
It should be emphasized that the assumption of a constant specific heat in the liquid in 
equation \eqref{eq:gradient_liquid_enthalpy} applies when no macrosegregation occurs. 
Nonetheless, when the latter is considered, the phases specific and latent heats become 
highly dependent on the local average composition. It then advisable to use the thermodynamic 
tabulation approach, where the enthalpies are directly tabulated as functions of temperature 
and composition. 
% ======================
\subsection{Species Conservation}
% ======================
The last conservation principle is applied to the chemical species or solutes. This principle allows predicting
macrosegregation when applied to a solidification system, along with the mass, momentum and energy balances.
However, the conservation equation should be reformulated in the case of a melt flow driven by shrinkage.
% ======================
\subsubsection{Assumptions}
% ======================
\begin{itemize}
\item The alloy is binary, i.e., it is composed from one solute, and hence the notation of the average composition
		without a solute index: $\wavg$ for the mass composition and $\avg{\rho w}$ for the volume composition
\item The solid fration is determined assuming complete mixing in both phases, hence the lever rule is applicable. It
		should be mentionned that the solidification path in the current approach is tabulated using thermodynamic data at 
		equilibrium
\item The solutal diffusion coefficient $D^s$ in the solid phase is neglected in the mass diffusive flux term. The
		remaining term, $D^l$, is a mass diffusion coefficient in the liquid phase, of unit $m^2 s^{-1}$
\item Consequence of the static solid phase: $\avg{\rho w \vec{v}} = \gl \rhol \wl \vl +  \cancel{\gs \rhos \ws \vs} = \gl \rhol \wl \vl$ 
\end{itemize}
% ======================
\subsubsection{Formulation}
% ======================
The species conservation is pretty similar the energy conservation formluted in the previous section. For a binary alloy, we can write:
\begin{subequations}
\begin{align}
	& \frac{\partial \avg{\rho w}}{\partial t} + \nabla \cdot \avg{\rho w \vec{v}} 
	= \nabla  \cdot \brac{\rhol \avg{D^l} \nabvec \wl } \\
	& \frac{\partial \avg{\rho w}}{\partial t} + \nabla \cdot \brac{\gl \rhol \wl \vl}
	= \nabla  \cdot \brac{\gl\rhol D^l \nabla \wl } \\ 
	& \frac{\partial \avg{\rho w}}{\partial t} 
		+ \rhol \wl  \nabla \cdot \vit
		+ \vit \cdot \nabvec \brac{\rhol \wl}
		= \nabla  \cdot \brac{\gl \rhol D^l \nabvec \wl } \\   
	& \frac{\partial \avg{\rho w}}{\partial t} 
		+ \cancel{\rhol} \wl  \frac{\rhol-\rhos}{\cancel{\rhol}} \frac{\partial  \gs }{\partial t}
		+ \vit \cdot \nabvec \brac{\rhol \wl}
		= \nabla  \cdot \brac{\gl \rhol D^l \nabvec \wl } \\ 
	& \frac{\partial \avg{\rho w}}{\partial t} 
		+ \brac{\rhol-\rhos} \wl \frac{\partial  \gs }{\partial t}
		+ \vit \cdot \nabvec \brac{\rhol \wl}
		= \nabla  \cdot \brac{\gl \rhol D^l \nabvec \wl }        
\end{align}
\end{subequations}
\begin{align}
\label{eq:solute_balance}
 \boxed{ \frac{\partial \avg{\rho w}}{\partial t} 
		+ \rhol \vit \cdot \nabvec \wl
		= \nabla  \cdot \brac{\rhol D^l \nabvec \wl }
		+ \brac{\rhos-\rhol} \wl \frac{\partial  \gs}{\partial t}}
\end{align}
% ======================
\subsubsection{Discussion}
% ======================
The species transport equation is usually derived with the volumetric average composition $\avg{\rho w}$, then
divided by the density, which is constant if no solidification shrinkage occurs. In the case where macrosegregation
is solely due to fluid flow generated by natural or forced convection, the overall
volume remains constant. It is thus convenient to compute composition variations using the mass variable $\wavg$.
However, in the current context, the volume is subject to changes, hence the formulation of equation 
\eqref{eq:solute_balance} with $\avg{\rho w}$.

%====================================================================
\section{FE model: Air}
In this section, we start from a the monodomain finite element model presented in \cref{sec:monodomain} relevant to metal only, 
then present the essential assumptions and formulations that allow predicting solidification shrinkage in a Eulerian context.

%-----------------------------------

\section{FE monolithic model}
%---------------------------------------
\subsection{Interface treatment}
%--------
How to transport level set using velocity from momentum conservation DIRECTLY or AVERAGED PER ELEMENTS, 
show examples of instability/stability when using false/nominal air properties \\ 
Validation of LS transport:
\comment{perform test case simulation of buoyancy driven air droplet in water by 2005Nagrath that I also have seen 
in Shyamprasad's masters report)}
\comment{ I read it quickly without noticing: what time step $\delta t$ did they use ? }
%---------------------------------------
\subsection{Permeability mixing}
%------
How to mix liquid fraction, best using harmonic or arithmetic, in order to replicate the effect the of non slip condition at top for example \\
Put the python plots from the presentations in "TEXUS monolithic" \\
Put video animations of PSEUDO SMACS 2D without and with LS ???
%---------------------------------------
\subsection{Model equations}
%-------
Rewrite air-metal monolithic conservation equations using mixture laws.
%--------------------------------
\section{Shrinkage without macrosegregation}
Explain how the flow and heat transfer in the air are not important \\ 
Give the strong form equations to be solved OR simply refer the previous section where the model was defined \\
Initial and boundary conditions for energy and momentum:  Initially we have liquid and air at rest. 
%--
\subsection{Al-7wt \% Si}
Present pseudo 1D case with results + discussion
%--
\subsection{Pb-3wt \% Sn}
Present 2D and 3D case with results + discussion
%--
%--------------------------------
\section{Shrinkage with macrosegregation}
Explain how the flow and heat transfer in the air are not important \\ 
Give the strong form equations to be solved OR simply refer the previous section where the model was defined \\
Initial and boundary conditions for energy and momentum:  Initially we have liquid and air at rest. 
%--
\subsection{Al-7wt \% Si}
Present pseudo 1D case with results + discussion
%--
\subsection{Pb-3wt \% Sn}
Present 2D and 3D case with results + discussion
%--

