\chapter{Macrosegregation with solidification shrinkage}
\minitoc
\newpage

this chapter discusses the following points:
\begin{itemize}
\item Density variation during solidification (industrial point of view then how to handle numerically)
\item Model equations
\item Application to solidification benchmark SMACS
\item Possible extension of application with grain structure (using Shijia's LS-air cells handling)
\item Multiscale Freckle predicition: FE + grain structure (in which lies a part about nucleation-growth and how numerically 
		we reach a smaller scale, the scale of the grain boundaries)
\end{itemize}


\section{Introduction}
\comment{ This report summarizes the derivation of conservation equations (mass, momentum, energy and species) relative 
to a specific solidification scenario where the surface of the solidifying alloy in contact with a surrounding 
gas (ambient air, argon gas ...) deforms due to solidification shrinkage, that is, the effect of relative change
in density between the liquid and the solid phases of the metal. } 

The shrinkage phenomenon is manifested in 2 
important locations, as depicted in \autoref{fig:ingot_sl_b_bis}. First, at the level of the liquid-solid interface
in the mushy zone, the solid forms with a density smaller than the liquid's. The volume difference tends to create 
voids with a big negative pressure, that systematically drains the liquid in its direction. Afterwards, we can 
see the consequence far away, at the surface ingot, where the initial metal(liquid)-air deforms gradually, forming
the so-called \emph{shrinkage pipe}. Since the mass of the alloy and its chemical species are conserved, a density
difference between the phases ($\rhos > \rhol$), leads to a different overall volume ($V^s<V^l$), once solidification
is complete, as shown in the following equations:
\begin{subequations}
\begin{align}
& \rhol V^l = \rhos V^s  \\ 
& V^s = \frac{\rhol}{\rhos} V^l
\end{align}
\end{subequations}
% FIGURE %%
\begin{figure}
\centering
\begin{subfigure}[h!]{0.3\textwidth}\centering % h! or H
	\def\svgwidth{100pt}
	\import{Chapter4/Graphics/}{Ingot_sl_1D_a.pdf_tex}
	\caption{Early stage}
	\label{fig:ingot_1d_a}
\end{subfigure}
%\begin{subfigure}[H!]{0.3\textwidth}\centering % h! or H
%	\centering
%	\def\svgwidth{100pt}
%	\import{Chapter4/Graphics/}{Ingot_sl_b.pdf_tex}
%	\caption{Intermediate stage}
%	\label{fig:ingot_sl_b}
%\end{subfigure}
\begin{subfigure}[h!]{0.3\textwidth}\centering % h! or H
	\centering
	\def\svgwidth{100pt}
	\import{Chapter4/Graphics/}{Ingot_sl_b_bis.pdf_tex}
	\caption{Intermediate stage}
	\label{fig:ingot_sl_b_bis}
\end{subfigure}
\caption{Schematic of a two cooling stages of an ingot against side and bottom mould walls (not shown). The top surface is free}
\end{figure}
%% FIGURE %%
%====================================================================
\section{Mass Conservation}
% ======================
\subsection{Assumptions}
% ======================
\begin{itemize}
\item Two phases are considered, liquid $l$ and solid $s$: $g^l+g^s =1 \implies \frac{dg^l}{dt} =- \frac{dg^s}{dt}$ 
\item The phase densities are constant but not equal: $ \rho^l=cst_1 $ and $ \rho^s=cst_2 $. Thermal and solutal expansion/contraction
is neglected
\item The solid phase is assumed static: $\vec{v^s}=\vec{0}$, which yields the following consequences:
\begin{enumerate}
\item $\avg{\vec{v}}= g^l \vec{v^l} + \cancelto{0}{g^s \vec{v}^s} = g^l \vec{v}^l $ 
\item $\avg{\rho \vec{v}} = g^l \rho^l \vec{v}^l + \cancelto{0}{g^s \rho^l \vec{v}^s} = g^l \rho^l \vec{v}^l $ 
\item $ \frac{\partial \rho^l }{\partial t} = \frac{\partial \rho^s }{\partial t} = 0 $
\item $ \nabvec \rho^l = \nabvec \rho^s = \vec{0} $  
\end{enumerate}
\end{itemize}
% ======================
\subsection{Formulation}
% ======================
The mass balance equation averaged over the two phases, is developed taking into account the assumptions made earlier.
\begin{subequations}
\begin{align}
& \frac{\partial \avg{\rho}}{\partial t} + \nabla \cdot \avg{\rho \vec{v}}  = 0 \\ 
& \frac{\partial }{\partial t} \brac{ g^l \rho^l + g^s \rho^s} + \nabla \cdot \brac{ g^l \rho^l \vec{v}^l} = 0 \\ 
& g^l \cancel{\frac{\partial \rho^l }{\partial t}} + \rho^l \frac{\partial  g^l }{\partial t} 
	+ g^s \cancel{\frac{\partial \rho^s }{\partial t}} + \rho^s \frac{\partial  g^s }{\partial t} 
	+ \rho^l \nabla \cdot \brac{ g^l \vec{v}^l} 
	+ g^l \vec{v}^l \cdot  \cancel{ \nabvec \rho^l }	 = 0 \\
& \brac{ \rho^l - \rho^s } \frac{\partial  g^l }{\partial t} + \rho^l \nabla \cdot \brac{ g^l \vec{v}^l}  = 0
\end{align}
\end{subequations}
\begin{align}
\label{eq:mass_balance}
 \boxed{\nabla \cdot \brac{ g^l \vec{v}^l} 
 	= \nabla \cdot \avg{\vec{v}^l} 
 	= \frac{\rho^l-\rho^s}{\rho^l} \frac{\partial  g^s }{\partial t}}
\end{align}
% ======================
\subsection{Discussion}
% ======================
With the assumptions of static solid phase and constant unequal phase densities, the average mass balance states that 
the divergence of the liquid velocity is proportional to the solidification rate, the proportionality constant being the
relative density change (which results in a relative volume change). This relation between the liquid velocity and the
temporal derivative of the solid fraction, explains the flow due to shrinkage. In metallic alloys, the solid density is
usually greater than the liquid density, therefore the first term in the RHS is negative. As for the second term, if we
neglect remelting, then it'll be positive in the solidifying areas of the alloy. A negative divergence term in these areas, 
means that a liquid feeding is necessary to compensate for the density difference, hence acting as a flow driving force in the melt.
In the case of constant densities, we can easily deduce that the divergence term is null, and therefore no flow is induced
by solidification.
\newline
Numerically speaking, a non-zero divergence term in the mass balance is equivalent to a compressible fluid behaviour. Additional 
terms should appear in the other conservation equations, balancing the volume change in the momentum, heat and species transport.

%====================================================================
\section{Momentum Conservation}
% ======================
In a typical volume averaging approach, one would write one momentum conservation 
equation for each phase. Nonetheless, only one equation will be present in our case, 
and that is the consequence of the assumption of the static solid, made in the previous 
section. It should be emphasized that, despite considering a single conservation equation, 
the effect of the solid movement with respect to the liquid's can still be incorporated 
through the interfacial fluxes in the momentum conservation of the liquid phase. 
% ======================
\subsection{Assumptions}
% ======================
\begin{itemize}
%%%
\item The interfacial momentum transfer between the solid and liquid phases is modelled 
by a momentum flux vector $\vec{\Gamma}^{l}$, consisting of hydrostatic and deviatoric 
parts, such that:
\begin{subequations}
\begin{align}
\label{eq:eq_interfacial_momentum_init}
		& \vec{\Gamma}^{l} =  \vec{\Gamma}_{p}^{l} + \vec{\Gamma}_{\mathbb{S}}^{l}						\\
		& \vec{\Gamma}_{p}^{l} = p^{l^{*}} \nabvec g^{l} = p^{l} \nabvec g^{l}			  \\
		& \vec{\Gamma}_{\mathbb{S}}^{l} = -{g^{l}}^{2} \mu^{l} \mathbb{K}^{-1} \brac{ \vl - \cancel{\vs} }  
\end{align}
\end{subequations}
where $p^{l^{*}}$ is the pressure at the interface, considered to be equal to the liquid hydrostatic 
pressure, $\mathbb{K}$ is the permeability computed by the Carman-Kozeny relation and $\mu^l$ is the 
liquid's dynamic viscosity.  For the solid phase, the interfacial terms are the opposite, which cancels 
them out with the liquid terms if the phase momentum equations are summed up.
%%%
\item The liquid is considered as a \emph{compressible} Newtonian fluid. It implies that the deviatoric part, $\mat{\mathbb{S}}^{l} $, of the Cauchy 
stress tensor is decomposed as follows: 
\comment{CAG: the liquid is incompressible, but the mixture is compressible, change it}
\begin{subequations} 
\begin{align}
\label{eq:stress_liq}
& \avg{ \mat{\sigma}^{l} } = - \avg{p^l} \mat{\mathbb{I}} + \avg{\mat{\mathbb{S}}^{l}}   \\
& \avg{ \mat{\sigma}^{l} } = - \avg{p^l} \mat{\mathbb{I}} + 2\mu^{l} \strainrate + \avg{\mat{\tau}^l}
\end{align}
\end{subequations}
where $ \avg{{\dot{\mat{\varepsilon}}}^l} $ is the strain rate tensor that depends on the average liquid velocity: 
\begin{align}
\label{eq:tensor_strainrate}
\strainrate = \mat{\nabla} \vit  +  \mat{\nabla}^\text{\textbf{t}} \vit 
\end{align}
and  $\mat{\tau}^{l}$ is the extra stress tensor in the liquid, given by:
\begin{subequations}
\begin{align}
\label{eq:tensor_extrastress}
 & \avg{\mat{\tau}^l} =  -\lambda \nabla \cdot \vit  \mat{\mathbb{I}} 
\end{align}
\end{subequations}
where $\lambda$ is a dilatational viscosity \textbf{CITE RAP2003}. For an incompressible flow, the divergence term vanishes, hence the classical Newtonian constitutive law is retrieved. In the literature, the coefficient $\lambda$ is taken proportional to the viscosity: $\lambda = \frac{2}{3} \mu^l $
\end{itemize}
% ======================
\subsection{Formulation}
% ======================
The momentum conservation equation in the liquid writes:
\begin{align}
\label{eq:momentum_liq}
& \temp{\rho^l g^l \vec{v}^l } + \nabvec \cdot \brac{\rho^l g^l \vl \times \vl} = 
	\nabvec \cdot \brac{g^l \mat{\sigma}^l} + g^l \vec{F}_\text{v} + \vec{\Gamma}^{l}
\end{align}
where $\vec{F}_\text{v}$ is an external volume force. The effect of the mass balance obtained in the previous section 
is incorporated by expanding the temporal and spatial derivatives in the momentum equation, taking firstly the left-hand side of equation \eqref{eq:momentum_liq}.
\begin{subequations}
\begin{align}
\label{eq:momentum_liq_LHS_a}
	 \text{LHS} &=
	 \rho^l \temp{ \gl \vl } + \gl \vl \cancel{\frac{\partial \rhol}{\partial t}} 
	+ \vl \nabla \cdot \brac{ \rhol \gl \vl } 
	+ \nabmat \vl \brac{ \rho^l g^l \vl } \\
\label{eq:momentum_liq_LHS_b}
	&= \rho^l \underbrace{ \temp{ g^l \vl } }_{\substack{\text{Unsteady} \\ \text{Acceleration}}}
	 + \brac{ \rhol - \rhos } \underbrace{ \frac{\partial  \gs }{\partial t} \vl}
	    _{\substack{\text{Shrinkage}\\ \text{Acceleration}}}
	+ \rhol  \underbrace{ \nabmat \vl \brac{ \gl \vl }}_{\substack{\text{Advective} \\ \text{Acceleration}}}     
\end{align}
\end{subequations}
The development in \eqref{eq:momentum_liq_LHS_b} shows that the origin of the flow, namely its acceleration, is 
attributed to three causes: 
 i) unsteady acceleration: a temporal change of a particle's velocity, ii) shrinkage-induced acceleration: a local "suction" effect at the solid-liquid interface (where $\frac{\partial  g^s }{\partial t} >0$) caused by the density jump $\brac{ \rhol - \rhos }$
and iii) convective acceleration: a spatial change in the velocity field.
The effect of the \emph{shrinkage-induced} flow is introduced using the mass balance in equation \eqref{eq:mass_balance}. The right-hand side of equation \eqref{eq:momentum_liq} is now expanded:
\begin{subequations}
\begin{align}
\label{eq:momentum_liq_RHS}
	\text{RHS} 
	&=
	 \nabvec \cdot \brac{\avg{p^l} \mat{\mathbb{I}} + 2\mu^{l} \strainrate + \avg{\mat{\tau}^l}} 
	 + \gl \rhol \vec{g} 
	 + \vec{\Gamma}_{p}^{l} + \vec{\Gamma}_{\mathbb{S}}^{l}	 \\
	&=  -\nabvec \brac{g^l p^l} 
		+ \nabvec \cdot \brac{ 2 \mu^l \strainrate }
		+ \nabvec \cdot \brac{-\frac{2}{3} \mu^l \nabla \cdot \avg{\vl} \mat{\mathbb{I}}}
		+ \gl \rhol \vec{g} 
	 	+ \vec{\Gamma}_{p}^{l} + \vec{\Gamma}_{\mathbb{S}}^{l}	 \\
	&=  \cancel{- p^l \nabvec \gl} - \gl \nabvec p^l
		+ \nabvec \cdot \brac{ 2 \mu^l \strainrate }
		+ \nabvec \cdot \brac{-\frac{2}{3} \mu^l \nabla \cdot \avg{\vec{v}^l} \mat{\mathbb{I}}}
		+ g^l \rho^l \vec{g} 
	 	+ \cancel{\vec{\Gamma}_{p}^{l}} + \vec{\Gamma}_{\mathbb{S}}^{l}	 \\
	&=   - g^l \nabvec p^l
		+ \nabvec \cdot \brac{ \mu^l \brac{\mat{\nabla} \avg{\vec{v}^l} + ^\text{\textbf{t}} \mat{\nabla} \avg{\vec{v}^l} }}
		+ \nabvec \brac{-\frac{2}{3} \mu^l \frac{\rho^l-\rho^s}{\rho^l} \frac{\partial  g^s }{\partial t} }
		+ g^l \rho^l \vec{g} 
	 	- {g^l}^{2} \mu^l \mathbb{K}^{-1} \vec{v}^{l}
\end{align}
\end{subequations}
The system thus consists of 3 equations (one for each of the components of $\vec{v}^l$) and 4 unknowns ($\vec{v}^l_x$, $\vec{v}^l_y$, $\vec{v}^l_z$ and $p$). An additional equation is provided by the mass continuity equation \eqref{eq:mass_balance}. For convenience, the superficial velocity $\vit$ will be chosen as a velocity unknown instead of the intrinsic average: $\vit = g^l \vec{v}^l$. The final system to 
solve, after grouping the unknowns in the LHS and the remaining terms in the RHS, is given by:
\comment{ I changed \emph{split} to \emph{align} here }
%\begin{empheq}[box=\widefbox, left=\empheqlbrace]{align}
\label{eq:momentum_liq_final}
%\label{eq:momentum_liq_final_vitesse}
\begin{align} % split
& \rho^l \frac{\partial \vit}{\partial t}
	 + \frac{ \rhol - \rhos }{g^l}\frac{\partial  \gs }{\partial t} \vit
	 + \rhol  \nabmat \vl \vit
	 + \nabvec \cdot \brac{2 \mu^l \strainrate}
	 + g^l \mu^l  \mathbb{K}^{-1} \vit  \\
	 &\qquad  = g^l \nabvec p^l
	 + \nabvec \brac{-\frac{2}{3} \mu^l \nabla \cdot \avg{\vec{v}^l}  }
	 +  g^l \rho^l \vec{g}  \\
%\label{eq:momentum_liq_final_pression}
&\nabla \cdot \vit= \frac{\rho^l-\rho^s}{\rho^l} \frac{\partial  g^s }{\partial t}
\end{align} % split
%\end{empheq}
%\end{subequations}

%====================================================================
\section{Energy Conservation}
We have seen the averaged energy conservation equation in the case of two phases: a solid phase and an incompressible liquid phase. However, with the incorporation of the shrinkage effect, new terms should appear
in the advective-diffusive heat transfer equation. 

\subsection{Assumptions}
\begin{itemize}
\item The thermal conductivity is constant for both phases: $\avg{\kappa} = \ks = \kl= \kappa $ 
%\item consideration of a fixed solid ($ \vec{v}^s=\vec{0} $).
\item Consequence of the static solid phase: $\avg{\rho h \vec{v}} = \gl \rhol \hl \vl +  \cancel{\gs \rhos \hs \vs} = \gl \rhol \hl \vl$ 
\item The system's enthalpy may thermodynamically evolve with pressure, knowing that $h=e+\frac{p}{\rho}$, where $e$ is the internal energy and $p$ is the pressure. It infers that the heat transport
equation may contain a contribution attributed to volume compression/expansion:
\begin{align}
			 \frac{\partial p}{\partial t}+\nabla \cdot \brac{p \vec{v}}
			 = \frac{\partial p}{\partial t}+ p \nabla \cdot \vec{v} + \vec{v} \cdot \nabvec p 
\end{align}
In the literature, this contribution has been always neglected, even when accounting for solidification
shrinkage, owing to the small variations of pressure.
\item Another contribution is also neglected in solidification problems, that is the heat generated by
mechanical deformation, $\mathbb{S}:\dot{\varepsilon}$ 
%\begin{enumerate}
%\item $\avg{\rho h}= \gl \rhol \hl + \gs \rhos \hs $
%\end{enumerate}
\end{itemize}

\subsection{Formulation}
The unknowns in the energy conservation are the average volumetric enthalpy $\avg{\rho h}$ and temperature $T$.The energy conservation equation writes:
\begin{subequations}
\begin{align}
	& \frac{\partial \avg{\rho h}}{\partial t} + \nabla \cdot \avg{\rho h \vec{v}} 
	= \nabla  \cdot \brac{\avg{\kappa} \nabvec T } \\
	& \frac{\partial \avg{\rho h}}{\partial t} + \nabla \cdot \brac{\gl \rhol \hl \vl}
	= \nabla  \cdot \brac{\kappa \nabvec T } \\ 
	& \frac{\partial \avg{\rho h}}{\partial t} 
		+ \rhol \hl  \nabla \cdot \vit
		+ \vit \cdot \nabvec \brac{\rhol \hl}
		= \nabla  \cdot \brac{\kappa \nabvec T } \\   
	& \frac{\partial \avg{\rho h}}{\partial t} 
		+ \cancel{\rhol} \hl  \frac{\rhol-\rhos}{\cancel{\rhol}} \frac{\partial  \gs }{\partial t}
		+ \vit \cdot \nabvec \brac{\rhol \hl}
		= \nabla  \cdot \brac{\kappa \nabvec T } \\ 
	& \frac{\partial \avg{\rho h}}{\partial t} 
		+ \brac{\rhol-\rhos} \hl \frac{\partial  \gs }{\partial t}
		+ \vit \cdot \nabvec \brac{\rhol \hl}
		= \nabla  \cdot \brac{\kappa \nabvec T }        
\end{align}
\end{subequations}
\begin{align}
\label{eq:energy_balance}
 \boxed{ \frac{\partial \avg{\rho h}}{\partial t} 
		+ \rhol \vit \cdot \nabvec \hl
		= \nabla  \cdot \brac{\kappa \nabvec T }
		+ \brac{\rhos-\rhol} \hl \frac{\partial  \gs}{\partial t}}
\end{align}

% ======================
\subsection{Discussion}
% ======================
In order to keep things simple, the term "enthalpy" will refer henceforth to "volume enthalpy",
otherwise, we will explicitly use the term "mass enthalpy". It is important to understand the 
meaning of the terms in equation \eqref{eq:energy_balance}.
The first term in the left-hand side is the temporal change in the system's average enthalpy,
i.e. a temporal change in the volume enthalpy of any of the phases in the course of solidification.
The second LHS term is a dot product between the superficial liquid velocity and the the gradient
of the liquid's enthalpy. Since phase densities are constant in our case, the gradient term reduces
to the liquid's mass enthalpy. If we consider a representative volume element (RVE) in the liquid
phase, far from the mushy zone, we can stipulate:
\begin{align}
\label{eq:gradient_liquid_enthalpy}
& \nabvec \hl = C_p^l \nabvec T
\end{align}
assuming that the phase mass specific heat, $ C_p^l $, is constant. Therefore, the liquid enthalpy
is advected in the case where the velocity vector is not orthogonal to the temperatre gradient.
The advection reaches its maximum when the two vectors have the same direction. Consider, for instance,
a filled ingot with a cooling flux applied to its bottom surface. If the density variation with temperature
were to be neglected, then the sole mechanical driving force in the melt is the density jump at the solid-liquid
interface ahead of the mushy zone. The temperature gradient in such a case is vertical upward, while the velocity
vector is in the opposite direction. The advective term writes:
\begin{align}
%\label{eq:}
& \rhol \vit \cdot \nabvec \hl = - \rhol C_p^l \norm{\vit} \norm{\nabvec T}
\end{align}
% FIGURE %%
\begin{figure}
\centering
\begin{subfigure}[h!]{0.3\textwidth}\centering % h! or H
	\def\svgwidth{100pt}
	\import{Chapter4/Graphics/}{Ingot_sl_1D_a.pdf_tex}
	\caption{Initial state}
	\label{fig:ingot_1d_a}
\end{subfigure}
\begin{subfigure}[h!]{0.3\textwidth}\centering % h! or H
	\centering
	\def\svgwidth{100pt}
	\import{Chapter4/Graphics/}{Ingot_sl_1D_c.pdf_tex}
	\caption{Solidification onset}
	\label{fig:ingot_1d_c}
\end{subfigure}
\begin{subfigure}[h!]{0.3\textwidth}\centering % h! or H
	\centering
	\def\svgwidth{100pt}
	\import{Chapter4/Graphics/}{Ingot_sl_1D_d.pdf_tex}
	\caption{Final state}
	\label{fig:ingot_1d_d}
\end{subfigure}
\caption{Effect of shrinkage flow on a solidifying ingot}
\end{figure}
%% FIGURE %%
We see that the second RHS term in equation \eqref{eq:energy_balance} acts as 
a heat source at the interface between the the phases, in this particular solidification
scenario. Another heat power (of unit $Wm^{-3}$) adds to the system within the mushy, 
that is the second term in the right-hand side of the same equation. This term is 
proportional to the solidification rate. Finally, the first RHS term accounts for thermal 
diffusion within the phases.
\newline
It should be emphasized that the assumption of a constant specific heat in the liquid in 
equation \eqref{eq:gradient_liquid_enthalpy} applies when no macrosegregation occurs. 
Nonetheless, when the latter is considered, the phases specific and latent heats become 
highly dependent on the local average composition. It then advisable to use the thermodynamic 
tabulation approach, where the enthalpies are directly tabulated as functions of temperature 
and composition. 
% ======================
\section{Species Conservation}
% ======================
The last conservation principle is applied to the chemical species or solutes. This principle allows predicting
macrosegregation when applied to a solidification system, along with the mass, momentum and energy balances.
However, the conservation equation should be reformulated in the case of a melt flow driven by shrinkage.
% ======================
\subsection{Assumptions}
% ======================
\begin{itemize}
\item The alloy is binary, i.e., it is composed from one solute, and hence the notation of the average composition
		without a solute index: $\wavg$ for the mass composition and $\avg{\rho w}$ for the volume composition
\item The solid fration is determined assuming complete mixing in both phases, hence the lever rule is applicable. It
		should be mentionned that the solidification path in the current approach is tabulated using thermodynamic data at 
		equilibrium
\item The solutal diffusion coefficient $D^s$ in the solid phase is neglected in the mass diffusive flux term. The
		remaining term, $D^l$, is a mass diffusion coefficient in the liquid phase, of unit $m^2 s^{-1}$
\item Consequence of the static solid phase: $\avg{\rho w \vec{v}} = \gl \rhol \wl \vl +  \cancel{\gs \rhos \ws \vs} = \gl \rhol \wl \vl$ 
\end{itemize}
% ======================
\subsection{Formulation}
% ======================
The species conservation is pretty similar the energy conservation formluted in the previous section. For a binary alloy, we can write:
\begin{subequations}
\begin{align}
	& \frac{\partial \avg{\rho w}}{\partial t} + \nabla \cdot \avg{\rho w \vec{v}} 
	= \nabla  \cdot \brac{\rhol \avg{D^l} \nabvec \wl } \\
	& \frac{\partial \avg{\rho w}}{\partial t} + \nabla \cdot \brac{\gl \rhol \wl \vl}
	= \nabla  \cdot \brac{\gl\rhol D^l \nabla \wl } \\ 
	& \frac{\partial \avg{\rho w}}{\partial t} 
		+ \rhol \wl  \nabla \cdot \vit
		+ \vit \cdot \nabvec \brac{\rhol \wl}
		= \nabla  \cdot \brac{\gl \rhol D^l \nabvec \wl } \\   
	& \frac{\partial \avg{\rho w}}{\partial t} 
		+ \cancel{\rhol} \wl  \frac{\rhol-\rhos}{\cancel{\rhol}} \frac{\partial  \gs }{\partial t}
		+ \vit \cdot \nabvec \brac{\rhol \wl}
		= \nabla  \cdot \brac{\gl \rhol D^l \nabvec \wl } \\ 
	& \frac{\partial \avg{\rho w}}{\partial t} 
		+ \brac{\rhol-\rhos} \wl \frac{\partial  \gs }{\partial t}
		+ \vit \cdot \nabvec \brac{\rhol \wl}
		= \nabla  \cdot \brac{\gl \rhol D^l \nabvec \wl }        
\end{align}
\end{subequations}
\begin{align}
\label{eq:solute_balance}
 \boxed{ \frac{\partial \avg{\rho w}}{\partial t} 
		+ \rhol \vit \cdot \nabvec \wl
		= \nabla  \cdot \brac{\rhol D^l \nabvec \wl }
		+ \brac{\rhos-\rhol} \wl \frac{\partial  \gs}{\partial t}}
\end{align}
% ======================
\subsection{Discussion}
% ======================
The species transport equation is usually derived with the volumetric average composition $\avg{\rho w}$, then
divided by the density, which is constant if no solidification shrinkage occurs. In the case where macrosegregation
is solely due to fluid flow generated by natural or forced convection, the overall
volume remains constant. It is thus convenient to compute composition variations using the mass variable $\wavg$.
However, in the current context, the volume is subject to changes, hence the formulation of equation 
\eqref{eq:solute_balance} with $\avg{\rho w}$.