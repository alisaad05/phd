\clearpage
\section*{Résumé chapitre 4}

Ce 4$^\circ$ chapitre est dédié à la macroségrgation induite par le mouvement de la phase liquide par convection thermosolutale,
à solide fixe et en absence de retrait à la solidification ($\rhos=\rhol$).
Pour cela, nous introduisons dans un premier temps les principaux schémas de résolution des équations Navier-Stokes selon la façon
dont ils répondent aux critères de stabilité de Babuška-Brezzi: les éléments finis
mixtes et la méthode mutli-échelles variationelle (Variational MultiScale). 

En choisissant la seconde méthode, nous donnons les détails de la formulation éléments finis correspondante qui régit les écoulements dans la phase liquide
loin du front de solidification, ainsi qu'au sein de la zone dendritique pâteuse. Le principal moteur de mouvement liquide est la convection thermosolutale.
Celle-ci est générée par la densité du liquide qui varie à la fois avec la température et la composition intrinsèque de la phase liquide, contribuant ainsi 
à la redistribution des éléments d'alliage.  
On s'intéresse à ce type de méso-macroségrégation en montrant une application de solidification dirigée, traitée dans le chapitre 3 en diffusion pure. 
Nous montrons qu'en fin de solidification, les écoulements créent des canaux à forte ségrégation positive en peau et dans le coeur de la pièce. 

L'investigation de ce défaut fait ensuite l'objet d'une confrontation qualitative entre la simulation et une expérience de solidification. Cette dernière consite en un 
banc de solidification dirigée d'un alliage d'indium-gallium à bas point de fusion. Un suivi en caméra rapide permet de suivre la formation de la microstructure
en fonction du temps. Par le biais de la simulation, on teste d'abord la performance du modèle purement macroscopique, i.e. avec suivi indirect 
des structures et phases via leur fraction volumique. Les résultats montrent que les canaux de ségrégation sont visibles mais sont moins nombreux et moins
stables que l'on prédit expérimentalement. 

Ensuite, on rajoute au modèle précedent une couche de modélisation à l'échelle mésoscopique pour suivre directement
les enveloppes des grains. Cette fois, la comparaison avec l'exprience montre que nous prédisons mieux qualitativement l'interaction complexe entre 
structure de solidification, l'écoulement au sein de la zone pâteuse et la ségrégation conséquente. Une étude paramétrique permet après d'étudier la sensibilité
de l'occurence et la forme des canaux ségrégés par rapport aux différents paramètres de contrôle du procédé.