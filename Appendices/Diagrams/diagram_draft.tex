\documentclass{article}

%%------------------------ SI units -------------------------------
\DeclareSIUnit\acceleration{\metre\per\square\second}
\DeclareSIUnit\velocity{\metre\per\second}
\DeclareSIUnit\diffusivity{\metre\squared\per\second}
\DeclareSIUnit\conductivity{\watt\per\metre\per\kelvin}
\DeclareSIUnit\composition{wt\%}
\DeclareSIUnit\masscapacity{\joule\per\kilo\gram\per\kelvin}
\DeclareSIUnit\massenergy{\joule\per\kilo\gram}
\DeclareSIUnit\volumecapacity{\joule\per\cubic\metre\per\kelvin}
\DeclareSIUnit\volumeenergy{\joule\per\cubic\metre}

\usepackage{mathtools}
\usepackage{tikz}
\usetikzlibrary{backgrounds,fit,calc,shadows,positioning}


\begin{document}

\newcommand{\vI}{\textbf{v}_{\text{I}}  }
\newcommand{\vII}{\textbf{v}_{\text{II}}  }
\newcommand{\capt}{\;\mathrm{capt}}
\newcommand{\deriv}[1]{\frac{\partial #1 }{\partial t}}
\newcommand{\diff}[2]{ \ensuremath{\nabla \cdot\left( #1 \nabla #2 \right)  }}
% This command is useful to have quick control over the format of all header titles
\newcommand{\name}[1]{\textbf{#1}}


\section{Monodomain model: without level set}

\begin{figure}[h!]
\newlength{\largeur}
\newlength{\llargeur}
\newlength{\rlargeur}

\setlength{\largeur}{4.5cm}
\setlength{\llargeur}{10cm}
\centering
\begin{tikzpicture}[node distance=0.4cm]

\tikzstyle{rect}=[rectangle,draw,text=black, fill=red!10, drop shadow, rounded corners]
\tikzstyle{test}=[diamond,aspect=3,draw,text=black]
\tikzstyle{fleche}=[->,>=stealth]
\tikzstyle{trait}=[]

\setlength{\rlargeur}{\largeur}
\addtolength{\rlargeur}{-1.\tabcolsep}
\addtolength{\rlargeur}{-1.\llargeur}

\node[rect,below=of mecaflu] (energy)
{
	\begin{tabular}{@{}p{\llargeur}p{\rlargeur}@{}}
	\name{Conservation of energy (Nonlinear Heat Transfer)} 
	\begin{equation*}
		\frac{\partial \avg{\rho h}}{\partial t} + \nabla \cdot \avg{\rho h \vec{v}} + \nabla \cdot \brac{\avg{\kappa} \nabvec T} = 0
	\end{equation*}
	\end{tabular}
};

% ===========================================================
% ===========================================================

\node[rect, below=of energy] (mecaflu)
{
	\begin{tabular}{@{}p{\llargeur}p{\rlargeur}@{}}
	\name{Mechanics Step II : Fluid-oriented Resolution}
	\begin{equation*}
	  \begin{array}{l l}
	    %{\rho}^{l}_{\text{ref}} 
		%\brac{\frac{\partial \brac{g^{l} \vII }}{\partial t} + 
		% \nabla \cdot\brac{g^{l} \vII  \times \vII  }} = \nabla\cdot \brac{g^{l}\mathbb{S}^{l}} 
		%-  g^{l} \nabla p^{l} + g^{l} {\rho}^{l} \vec{\mathbb{G}} - {g^{l}}^{2} \mu^{l} \mathbb{K}^{-1} \brac{ \vII  - \vI  } } \\ 
	    \nabla \cdot \vII = 0
	  \end{array} 	
	\end{equation*}
	\end{tabular}
};





\node[rect,below=of energy] (macroseg)
{
	\begin{tabular}{@{}p{\llargeur}p{\rlargeur}@{}}
	\name{Conservation of chemical species (Macrosegregation)} 
	\begin{equation*}
		\frac{\partial \langle w_i \rangle}{\partial t} + 	
		\avg{\vII}  \cdot   \nabla \langle w_i \rangle^{l} = 
		\nabla\cdot  ( \langle D^{l} \rangle \nabla \langle w_i \rangle^{l} )
	\end{equation*}
	\end{tabular}
};

\node[rect,below=of macroseg] (microseg)
{
	\begin{tabular}{@{}p{\llargeur}p{\rlargeur}@{}}
	\name{Microsegregation} 
	\begin{equation*}	   
	  \begin{array}{l l}
	    \brac{g^{\phi} , \avg{w_i^{\phi}}^{\phi} } = f\brac{\avg{w_i} , T }  \\ 
	    \frac{\partial \avg{\rho h} }{\partial T} = 
		 \frac{\partial}{\partial T} \brac{\sum_{\phi} g^{\phi}  \avg{\rho h}^{\phi} }
	  \end{array} 	
	\end{equation*}
	\end{tabular}
};


\draw[trait] (mecaflu) -- (energy);
\draw[trait] (energy) -- (macroseg);
\draw[trait] (macroseg) -- (microseg);

\end{tikzpicture}

%\caption{Coupling algorithm for structures i, without iterations. Solidification path is computed at nodes.}

\end{figure}


\end{document}