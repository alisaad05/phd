\chapter{Notes}

from \url{http://aerojet.engr.ucdavis.edu/fluenthelp/html/ug/node572.htm} \\

For many natural-convection flows, you can get faster convergence with the Boussinesq model than 
you can get by setting up the problem with fluid density as a function of temperature. This model 
treats density as a constant value in all solved equations, except for the buoyancy term in the momentum 
equation:
\begin{align}
 (\rho - \rho_0) g \approx -\rho_0 \beta (T - T_0) g 	
\end{align}

where  $\rho_0$ is the (constant) density of the flow,  $T_0$ is the operating temperature, and  $\beta$ is 
the thermal expansion coefficient. Equation  13.2-18 is obtained by using the Boussinesq approximation  
$\rho = \rho_0 (1 - \beta \Delta T)$ to eliminate  $\rho$ from the buoyancy term. This approximation is accurate as 
long as changes in actual density are small; specifically, the Boussinesq approximation is valid when  $\beta(T-T_0)\ll 1$.


\section{Useful Expressions}
\comment{ address a problem: attend to, apply oneself to, tackle, see to, deal with, confront, come to grips with, get down to, turn one's hand to, take in hand, undertake, concentrate on, focus on, devote oneself to
"the selectmen failed to address the issue of subsidies" }
\begin{itemize}
\item  the basic premise = the basic argument is that ...
\item A complication inherent in this approach is that
\item an important objective of the present study is to
\end{itemize}
