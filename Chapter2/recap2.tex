\clearpage
\section*{Résumé chapitre 2}

\begin{otherlanguage}{french}
{\small
Le second chapitre de ce manuscrit présente une revue de la littérature, concernant les aspects de modélisation utilisés dans ce travail.
Au début, on présente quelques approches connues pour modéliser l'effet de la microségrégation sur la variation des fractions de phases ainsi que leurs compositions.
On s'intéresse après à l'approche de prise de moyenne volumique. Celle-ci nous permet de faire des hypothèses sur des petits volumes, qualifiés de \emph{volumes élémentaires représentatifs},
permettant d'établir des relations pour l'ensemble des propriétés des phases. En utilisant ces relations, on présente la première brique numérique de ce travail: les équations
de conservation de masse, énergie, solutés et quantité de mouvement dans un contexte de solidification à volume constant, donc sans retrait. Le modèle est complété par une hypothèse de 
solide fixe et rigide qui permet de négliger tout movement des phases solides, et donc la thermomécanique de ces phases n'est pas traitée.
Dans la suite du chapitre, on présente les descriptions eulérienne et lagrangienne caractérisant l'écoulement de la phase liquide. 
Cela s'avère nécessaire dans un contexte de solidification avec changement de volume, où l'interface entre le métal et le milieu ambiant change au cours de la transformation.
Par conséquent, le choix de la méthode level set pour le suivi de cette interface est expliqué. On présente aussi des méthodes numériques pour prendre en compte le mouvement
de l'interface suivie par la méthode level set, dans le contexte eulérien. Finalement, deux méthodes de remaillage adaptatif, utilisé pour l'ensemble des calculs faits pendant la thèse,
sont présentés et expliqués, tout en montrant leurs avantages et leurs inconvénients.
}
\end{otherlanguage}