\chapter{Energy Resolution: Tsolver}
\minitoc
\newpage


When we speak about macrosegregation in solidification, we have to remember that the problem is one that involves phase change.


\section{State of the art}
\begin{itemize}
\item Use of enthalpy resolution in the majority of works 
\item motivation and advantages of TvsH without talking about resolution time
\item use article's introduction to fill this section (or improvise new things)
\end{itemize}


\section{Thermodynamic considerations}
\comment{ this section should be revised for missing symbols, equations and figures from the corresponding article }

\subsection{Volume averaging} 
A volume averaging technique was suggested to deal with the presence of multiple phases \cite{ni_volume-averaged_1991}. It locally considers a 
Representative Volume Element (RVE) that contains a single or several phases (these are not necessarily in thermodynamic equilibrium) 
at a mesoscopic scale. We represent, for each unknown $\psi$, an intrinsic volume average, $\avg{\psi}^{\phi}$ (also denoted $\avg{\psi^{\phi}}^{\phi}$ in the 
literature), corresponding to a phase $\phi$. The volume average $\avg{\psi}$ for this unknown in the RVE, hence averaged over all the present phases writes:
%%-------------
\begin{align}
\label{eq:volume_average_psi}
& \avg{\psi} = \sum_\phi \gphi \avg{\psi}^{\phi}
\end{align}
%%-------------
where $g^\phi$ denotes the volume fraction of phase $\phi$ in the RVE. 
It should be emphasized that the averaging technique applies to virtually all thermodynamic variables (enthalpy, density $\dots$). 
Among these variables, the temperature is also considered to be uniform in the RVE. Applying the volume averaging technique to the energy 
conservation principle along with interfacial balances between the phases, results in the following averaged equation \cite{rappaz_numerical_2003}:
%%-------------
\begin{align}
\label{eq:averaged_energy_eqn}
& \frac{\d \rh}{\d t} + \nabvec \cdot \rhv = \nabvec \cdot \brac{\avg{\kappa} \nabvec T} + \avg{\dot{Q}_V}
\end{align}
%%-------------
where $\rho$ stands for the density, $h$ the mass enthalpy, $\vec{v}$ the velocity field, $\kappa$ the thermal conductivity, $T$ the temperature 
and $\dot{Q}_V$ a possible volume heat source. \autoref{eq:averaged_energy_eqn} is the standard averaged form of the energy conservation equation used in non-stationary phase 
change problems. 
\comment{ I could elaborate more in this paragraph by showing the possible equations for the explicit formulation and maybe a figure to show the AlSi7
computation that i did with a v small time step }
 
Once the variational form has been discretized in space and time, two possible resolution schemes emerge: the first is an 
explicit forward Euler scheme which gives rise to a linear equation where the temperature is known at time $t$, $T^t$. This requires very small 
time steps in the current context, which limits the solver’s usability at the scale of industrial applications. The second scheme is the 
backward Euler or full implicit discretization where terms are function of $T^{t+\Delta t}$. It leads to a nonlinear equation with 2 interdependent 
unknowns, $\rh^{t+\Delta t}$ and  $T^{t+\Delta t}$. It is clear that the nature of the temperature-enthalpy relationship plays a central 
role when formulating the resolution strategy of this nonlinear equation. Generally, it is admitted that, depending on the resolution strategy, 
it is necessary to express enthalpy as a function of temperature or vice-versa, together with associated partial derivatives, 
$\frac{d \rh}{dT}$ or $\frac{dT }{d \rh}$.

\subsection{The temperature-enthalpy relationship} 
In solidification problems, additional variables are involved in \autoref{eq:volume_average_psi} and \autoref{eq:averaged_energy_eqn}, 
like the transformation path that defines the history of the phase fractions, as well as the average chemical composition $\wi$, 
i being the index of the chemical species (only the solutes are considered). The temperature-enthalpy relation averaged over the 
phases in a given RVE writes:
%%-------------
\begin{align}
\label{eq:volume_average_enthalpy}
& \rh = \sum_\phi \gphi_{\brac{T, \wi \dots}} \rphi_{\brac{T, \wi^{\phi} \dots}} \hphi_{\brac{T, \wi^{\phi} \dots}}
\end{align}
%%-------------
Note that the volume average enthalpy is approximated by the product $\avg{\rho h}^{\phi}=\avg{\rho}^{\phi}\avg{h}^{\phi}$ in the current work. As stated 
in the introduction, it becomes clear from \autoref{eq:volume_average_enthalpy} that phase properties, i.e. average phase density, , \rphi and enthalpy, \hphi, 
are temperature and composition dependent. This equation is the key to convert the average volume enthalpy to temperature (through a procedure named H2T) 
or vice-versa (T2H). The values of the different phase fractions $\gphi$ (solidification path) and phase enthalpies $\rh^{\phi}$ are thus needed 
to close the relation.

\subsection{Tabulation of properties}
The complexity of performing a thermodynamic conversion is directly linked to the simplicity of determining the alloy properties, namely the 
phase fractions and phase enthalpies. In the case of binary alloys and with several assumptions with respect to the system (e.g., linear monovariant 
temperature composition relationships, constant heat capacity of phases and constant latent heat of transformations, equilibrium approximations between 
phases) analytical calculations are often used to determine the properties. Nevertheless, analytical relations are more complex or even impossible to 
derive in the case of multicomponent alloys (i>1). To overcome this problem, one can resort to thermodynamic databases and phase equilibrium calculations 
to tabulate the transformation paths and the phase enthalpies for a given range of temperatures and average compositions. It is a handy solution for two 
main reasons: first, the conversion is merely a binary search in a table; secondly, it is a simple solution for coupling with macrosegregation. In this way, 
phase fractions $\gphi$ are tabulated as functions of temperature and average composition, while for each phase $\phi$ the mass enthalpy, $\hphi$, and 
the density, $\rphi$, are tabulated as functions of temperature and phase intrinsic average compositions HERE, as well as other possible parameters. 
Figure 1 summarizes the steps in order to perform a temperature-to-enthalpy (T2H) conversion using the predefined tabulation approach. In step 1, the 
transformation path is acquired for each average composition and temperature to determine the list of phases, their volume fractions $\gphi$ and their intrinsic 
compositions $\wi^{\phi}$. In step 2, the phase enthalpy $\hphi$  and density $\rphi$ are determined by searching for the temperature 
and the already known phase composition $\wi^{\phi}$. In step 3, the average volume enthalpy is computed from the volume fraction, density and mass enthalpy 
of phases using \autoref{eq:volume_average_enthalpy}.

\comment{ Figure 3 goes here }

The methodology to build the tabulations is straightforward. It is based on two main scans. On the one hand, intervals for the variation of the 
average composition $\wi$ are chosen from the known alloy composition. These variations have to cover the extreme values adopted during the 
simulation, which are not known a priori. An interval is also selected for the variation of temperature. The latter is easier to determine as it
usually starts from the initial melt temperature and goes down to the room temperature in a standard casting simulation. For these intervals, a 
systematic scan is made with chosen steps in each composition and T, during which a thermodynamic equilibrium is computed. The outputs are the 
number of phases encountered, together with their fraction and intrinsic composition. The minimum and maximum intrinsic composition for each phase 
could then be determined. On the other hand, for each phase, a scan of the intrinsic composition and temperature is made to compute the intrinsic 
properties. The same temperature interval and step as defined earlier are used.

\comment{ below paragraph should be re written and maybe stress LESS on the speed effect}
\comment{ I should change the superscript $k$ which may be confused with partition coefficient }
Regarding the enthalpy-to-temperature conversion (H2T), a backward iterative T2H search is performed. 
For a known composition $\wi$, denoting $k$ the iteration index to convert the enthalpy 
$\avg{\rho h}_{\text{input}}$, we start with an initial guess for temperature $\Tkinit$ then convert it to an 
enthalpy  $\Hkinit$ with the T2H conversion. Using an appropriate nonlinear algorithm (Brent is the most versatile 
in our case), we aim at minimizing the following residual: $\text{Residu}_{\rh} = \abval{\rh_{\text{input}} - \Hk }$. 
Once the algorithm has converged, the temperature $\Tk$ is the result of the H2T conversion. It is 
inferred that the first conversion (T2H) is a direct one whereas the latter (H2T) is indirect and requires 
a series of iterative steps; each step being a single T2H resolution. In other words, a H2T conversion is a 
backward search for a temperature, hence it’s slower. This conversion’s speed lag is exacerbated when tabulations 
increase in size (e.g. large number of temperature and composition steps) and complexity (e.g., multicomponent 
industrial alloys used in casting), since the search gets more complicated with the increasing number of input 
columns (one column for each alloying element).

\section{Formulation}
\comment{ Coming soon }

\section{Validation}

\subsection{Pure diffusion}
The two solvers are first tested in a purely diffusive case for a one-dimensional solidification configuration. 
Predictions with a 1D front tracking model \cite{gandin_constrained_2000} is used as a benchmark. It provides 
solutions for the temperature and solid fraction during directional solidification of a 10 cm long Al – 7 wt.\% Si 
ingot. The melt, with initial uniform temperature, is cooled with a heat exchange coefficient (assuming a Fourier 
boundary condition) from one side, the other side being adiabatic. All values for alloy properties, initial and 
boundary conditions and numerical parameters are listed in \autoref{table:data_case_alsi7}. For this simple test case, 
we use linear temperature dependence of the intrinsic phase enthalpies, that is $\rh^s= \rcp T$ and $\rh^l= \rcp T + \rho L$, 
where $\rcp$ is the heat capacity per unit volume and $\rho L$ is the latent heat per unit volume. Values for $\rcp$ 
and $\rho L$, as well as for the thermal conductivities, $\kappa = \kl = \ks$, are taken constant. Moreover, a 
Gulliver Scheil approximation is used to compute a single temperature – fraction of solid relationship in the 
absence of macrosegregation. This is done assuming a linear binary phase diagram and thus requires using the 
properties listed in \autoref{table:data_case_alsi7}, i.e. the segregation coefficient, $k$, the liquidus slope, $m_L$, the 
liquidus temperature, $T_L$, and the eutectic temperature, $T_E$. \textbf{Figure REF} shows the comparison with 
the Hsolver and Tsolver. The results are found superimposed to the front tacking solution, thus giving validation 
of the implementation as well as the iterative schemes presented above to solve the energy conservation.

\begin{table}
\centering
\begin{tabular}{llll}
\hline 
\textbf{Parameter} & \textbf{Symbol} & \textbf{Value} & \textbf{Unit} \\
\hline 
Nominal composition & $\avg{w}_0$ & \num{7} & • \\ 
Liquidus temperature & $T_L$ & \num{618} & • \\ 
Eutectic temperature & $T_E$ & \num{577} & • \\  
Segregation coefficient & $k$ & \num{0.13} & • \\  
Liquidus slope & $m_L$ & \num{-6.5} & • \\ 
Heat capacity (liquid and solid) & $\rho C_p$ & \num{2.6e6} & • \\  
Enthalpy of fusion & $\rho L$ & \num{9.5e8} & • \\ 
Thermal conductivity (liquid and solid) & $\kappa$ & \num{70} & • \\
\hline  
Heat transfer coefficient & $\hext$ & \num{500} & • \\ 
External temperature & $\Text$ & \num{100} & • \\ 
Initial temperature & $T_0$ & \num{800} & • \\ 
Ingot length &  & \num{0.1} & • \\ 
\hline 
FE mesh size &  & \num{e-3} & • \\ 
Time step & $\dt$ & \num{0.1} & • \\ 
Convergence criterion (residual) & $\varepsilon_R$ & \num{e-6} & • \\ 
Convergence criterion (temperature) & $\varepsilon_T$ & \num{e-2} & • \\ 
\hline 
\end{tabular} 
\caption{Parameters for the pure diffusion test case with alloy Al – 7 wt.\% Si presented in \textbf{FIGURE REF}}
\label{table:data_case_alsi7}
\end{table}

\subsection{Convection-diffusion with macrosegregation}
Conservation equations in \textbf{Table 2} are for mass, momentum and chemical species. 
As for energy, they are presented after the volume averaging technique has been applied 
\cite{ni_volume-averaged_1991} \cite{dantzig_solidification_2009}. Moreover, an assumption 
of a static and non deformable solid phase is made. Consequently, the mechanical model is 
reduced to the conservation of momentum in the liquid phase. This assumption also yields 
some other consequences on the mass balance and the liquid momentum conservation. In the 
latter, a Darcy term is added to take into account the dissipative interfacial stress in 
the porous-like mushy zone. Its main parameter is the permeability of the mushy zone, $K$. 
It is considered isotropic, hence reducing to a scalar which is given by the Carman-Kozeny 
relation, based on the secondary dendrite arm spacing $\lambda_2: K= \frac{g^{l^3}  \lambda_2^{2}
 }{180\left ( 1-g^l \right )^2}$. The liquid density being taken constant, its spatial variations 
as a function of temperature and average composition are still needed to compute thermosolutal 
convection forces. For that purpose, the Boussinesq approximation $\rl = \rref \brac{1-\betaT 
\brac{T-\Tref}-\betaWl \brac{\wl-\Wlref}}$ is used, considering the thermal $\betaT$ and solutal $\betaWl$) expansion coefficients 
and a reference density, $\rref$, defined at a reference temperature $\Tref$ and reference 
composition $\Wlref$. Values for the references are taken at the liquidus temperature and the nominal 
composition of the alloy, $\w_0$ \cite{carozzani_direct_2013}. More details about the FE formulation can be found in 
\cite{rivaux_simulation_2011, carozzani_developpement_2012}. It should be noted that the macroscopic 
solute diffusion coefficient in the solid phase is neglected in \textbf{REF Eq. 15c}. 

\comment{ in this table I use directly the simplified equations, but this was done only for
the article, now I have to go from the full conservation equations and state the hypothesis and 
methods that i used to reach this simplified form }

\begin{table}[h]
\centering
\begin{subequations}
\begin{align}
& \nabla \cdot \brac{\gl \vl}  = 0 \\ 
& \temp{\gl \rref \vl} + \nabvec \cdot \brac{\gl \rref \vl \times \vl} - \nabvec\cdot\avg{S}^l -\gl
   \nabla \pl +\mul \gl^2 K^{-1} \vl -\gl\rhol \gravity = 0 \\
& \temp{\wi} + \brac{\gl \vl} \cdot \nabvec \wil + \nabla \cdot \brac{\gl \Dl \nabvec \wil } = 0
\end{align}
\end{subequations}
\caption{Averaged conservation equations for the conservation of mass (a), momentum (b) and solute mass (c)}
\label{table:smacs_equations}
\end{table}
\comment{ Figure SMACS: Computed unidirectional heat diffusion during solidification of an Al – 7 wt.\% Si 
alloy using (orange) the enthalpy method and (black) the temperature method, comparison being made for (left) 
cooling curves and (right) time history of the liquid fraction. Each curve corresponds to a position along the 
sample, from \SI{0}{\centi \metre} (cooling side) to \SI{10}{\centi \metre} (insulated side), with \SI{2}{\centi 
\metre} spacing between the positions. }

The Tsolver’s ability to be coupled with various physical phenomena like macrosegregation and fluid flow 
in porous medium is displayed in this test case. It consists of a solidification benchmark where a \SI{10}{\centi \metre}
width × \SI{6}{\centi \metre} height × \SI{1}{\centi \metre} thick cavity containing a Sn – 3 wt.\% Pb melt is cooled down from its two 
narrowest vertical sides using heat exchangers (LHE: left heat exchanger, RHE: right heat exchanger). The 
experiment, inspired by Hebditch and Hunt’s \cite{hebditch_observations_1974} similar set up, has been 
revisited by Hachani et al. \cite{hachani_experimental_2012} who performed the solidification with better 
controlled conditions and using an increased number of samples for composition analysis. Recently, a successful 
attempt to simulate the experiment was carried out by Carozzani et al. relying on an enthalpy resolution 
\cite{carozzani_direct_2013}. All details regarding geometry, finite element discretization, material properties 
and boundary conditions can be found in the latter reference. 
\comment{ I could develop more here giving additional details }
For this computation, solidification paths, phase compositions and phase enthalpies were determined by a thermodynamic 
module dedicated to equilibrium calculations for binary alloys. The 3D simulation results in \textbf{REF Figure 4} show 
a satisfactory agreement with the experimental temperature measurements recorded at mid heights of the cavity and uniformly 
distributed along its width \cite{carozzani_direct_2013}. In fact, simulation results with the Tsolver and the Hsolver were 
found to be almost superimposed, as in \textbf{REF Figure 4}. Regarding the computation, the Tsolver resolution proves to be 
faster than the Hsolver used in \cite{carozzani_direct_2013}: a process time of 7000s required a computation time of 90 hours 
13 minutes compared to 114 hours 21 minutes spent by the enthalpy resolution with 32 cores on the same cluster. The gain factor 
is about 20\%.

\section{Application to multicomponent alloys}

The efficiency of the temperature-based resolution resides in its performance when combined with 
thermodynamic tabulations. A multicomponent alloy consists of at least two solute elements, and 
therefore the tabulation size increases, hence the number of search operations also increases. 
To demonstrate the speed-up ability of the temperature-based approach while predicting all phase 
transformations during macrosegregation, we consider the solidification of a ternary alloy, Fe–2 wt.\%C– 30 wt.\%Cr. 
As illustrated in \textbf{Figure 5a}, the alloy domain has a cylinder shape close to 3-inch height × 1-inch diameter. 
Exact values are reported in \textbf{Table 3} with all material properties, initial and boundary conditions, 
as well as numerical parameters for the simulations. The melt steel is initially at \SI{1395}{\udegC}. The 
temperature of the bottom surface is imposed with a constant decreasing rate of 0.1 K.s-1 starting 
with \SI{1380}{\udegC}, i.e. \SI{40}{\udegC} higher than the nominal liquidus temperature, as shown 
in \textbf{Figure 5b}. The other surfaces are kept adiabatic. The cylinder is held in a vertical position. 
In these conditions, and knowing that the carbon and chromium solutes have lightening effects on the liquid 
at nominal composition, the density inversion resulting from the composition gradient in the interdendritic 
liquid, may cause flow instability (segregation plumes) at the solidification front. While the selected alloy 
is a steel, this application is also representative of directional cooling in a single crystal casting, e.g. 
for nickel-base superalloys \cite{beckermann_development_2000}. \textbf{Figure 5c} also provides the 
transformation path of the alloy at nominal composition, i.e. assuming no macrosegregation and full 
thermodynamic equilibrium as computed with ThermoCalc and the TCFE6 database \cite{thermo-calc_andersson, tcfe6}. 
A total of 5 phases need to be handled, the characteristic temperature for their formation being reported 
in \textbf{Figure 5b}. 
\comment{Figure 5: Configurations for directional casting of (a) a 1 inch diameter × 3 inches height cylindrical
 domain for which (b) temperature-time conditions are imposed at its bottom surface. The alloy is Fe – 2 wt.\% C–
 30 wt.\% Cr, its computed transformation path [20], [21] at nominal composition being displayed in (c)}
 
\subsection{Tabulations}
Full thermodynamic equilibrium is considered in the present case. Due to macrosegregation, 
the average composition is expected to continuously vary in time and space during casting. 
Transformation paths are thus determined a priori for a set of average compositions around 
the nominal value. Hence, carbon content is arbitrarily varied in the interval [1.8 wt.\%, 2.2 wt.\%] 
while chromium content variation is in the interval [27 wt.\%, 33 wt.\%]. The offset of ±10\% with 
respect to the nominal composition value allows tabulating relatively small composition steps to 
ensure a good accuracy when compared to the corresponding ternary phase diagram. The average 
composition step is \bin{}{0.04}{} for carbon and \bin{}{0.6}{} for chromium, thus representing 2\% 
intervals with respect to the nominal composition. The temperature varies in the interval 
[\SI{100}{\udegC},\SI{1600}{\udegC}] by \SI{5}{\udegC} steps. For each triplet (carbon content 
in wt.\% C, \textbf{HERE} , chromium content in wt.\% Cr, \textbf{HERE}, temperature in \si{\udegK}) 
corresponds a phase fraction $g^\phi$ and a pair of intrinsic phase composition (\textbf{HERE}). For the 5 
phases listed in \textbf{Figure 5c} (LIQ$\equiv$liquid, BCC$\equiv$ferrite, FCC$\equiv$austenite, 
$\text{M}_7 \text{C}_3 \equiv$carbide, CEM$\equiv$cementite), the enthalpy $h^\phi$ and density $\rho^\phi$, are tabulated 
as functions of temperature and phase intrinsic composition. If this latter input lies between two tabulated 
values, a linear interpolation is performed to determine the output, i.e. phase enthalpy and density. With 
the advancement of solidification, the liquid is enriched with solute by macrosegregation, which enables new 
solidification paths. It means that the primary solidifying phase is not necessarily the same as when considering 
the nominal composition. For this reason, the tabulation approach is interesting inasmuch as it provides phase 
transformation paths and values of phase properties that are compatible with the system’s actual composition. 
\textbf{Figure 6} summarizes the tabulated thermodynamic data for two sets of average composition for the considered 
ternary system. Note that in the present test case, phase densities are taken constant ($\rhos=\rhol=$ \SI{6725}{\udensity}). 
Therefore they are not tabulated. With this assumption, no shrinkage occurs upon phase change.
\begin{table}[h]
\centering
\begin{tabular}{llll}
\hline  
\textbf{Parameter} & \textbf{Symbol} & \textbf{Value} & \textbf{Unit} \\
\hline 
Nominal composition & $\wC_0$ & 2 & \si{\ucomposition} \\ 
                    & $\wCR_0$ & 30 & \si{\ucomposition} \\ 
Characteristic temperatures & $T_\text{top},T_\text{bottom}$ & \textbf{FIGURE} & \si{\udegC} \\ 
Phase fraction & $g^\phi$ & Tabulations & $-$ \\ 
Phase enthalpy & $\h^\phi$ & Tabulations & $-$ \\ 
Phase composition & $\wC^\phi$ & Tabulations & \si{\ucomposition}  \\ 
                  & $\wCR^\phi$ & Tabulations & \si{\ucomposition}  \\ 
Diffusion coefficients & $\avg{D_\text{C}}^l$ & • & \si{\udiffusivity}  \\ 
                       & $\avg{D_\text{Cr}}^l$ & • & \si{\udiffusivity}  \\ 
Dynamic viscosity & $\mul$ & \num{2e-3} & \si{\uviscosity}  \\ 
Thermal expansion coefficient & \betaT & • & \si{\ubetaT}  \\ 
Solutal expansion coefficient & $\betaWlC$ & • & \si{\ubetaWl}  \\  
                              & $\betaWlCR$ & • & \si{\ubetaWl}  \\ 
Thermal conductivity in the solid & • & • & \si{\uconductivity}  \\ 
Thermal conductivity in the liquid & • & • & \si{\uconductivity}  \\ 
Dendrite arm spacing & • & • & \si{\metre}  \\ 
Density & • & • & \si{\udensity}  \\ 
\hline 
Initial temperature & • & • & \si{\udegC}  \\ 
Ingot diameter & • & • & \si{\metre}  \\ 
Ingot length & • & • & \si{\metre}  \\ 
\hline 
FE mesh size &  & \num{e-3} & \si{\metre}  \\ 
Time step & $\dt$ & \num{0.1} & \si{\second}  \\ 
Convergence criterion (residual) & $\varepsilon_R$ & \num{e-6} & $-$ \\ 
Convergence criterion (temperature) & $\varepsilon_T$ & \num{e-2} & \si{\udegK} \\ 
\hline 
\end{tabular} 
\caption{Parameters for solidification of alloy Fe – 2 wt.\% C – 30 wt.\% Cr }
\label{table:data_case_ternary}
\end{table}
