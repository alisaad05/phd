\chapter{Energy Resolution: Tsolver}
\minitoc
\newpage


When we speak about macrosegregation in solidification, we have to remember that the problem is one that involves phase change.


\section{State of the art}
\begin{itemize}
\item Use of enthalpy resolution in the majority of works 
\item motivation and advantages of TvsH without talking about resolution time
\item use article's introduction to fill this section (or improvise new things)
\end{itemize}


\section{Thermodynamic considerations}
\comment{ this section should be revised for missing symbols, equations and figures from the corresponding article }
\subsection{Volume averaging} 
A volume averaging technique was suggested to deal with the presence of multiple phases \cite{ni_volume-averaged_1991}. It locally considers a 
Representative Volume Element (RVE) that contains a single or several phases (these are not necessarily in thermodynamic equilibrium) 
at a mesoscopic scale. We represent, for each unknown $\psi$, an intrinsic volume average, HERE (also denoted HERE in the literature), 
corresponding to a phase HERE. The volume average HERE for this unknown in the RVE, hence averaged over all the present phases writes:
HERE	Eq. 1
where $g^\phi$ denotes the volume fraction of phase $\phi$ in the RVE. It should be emphasized that the averaging technique applies to 
virtually all thermodynamic variables (enthalpy, density $\dots$). Among these variables, the temperature is also considered to be uniform 
in the RVE. Applying the volume averaging technique to the energy conservation principle along with interfacial balances between the phases, 
results in the following averaged equation \cite{rappaz_numerical_2003}:
	HERE	Eq. 2
where $\rho$ stands for the density, $h$ the mass enthalpy, $v$ the velocity field, $\kappa$ the thermal conductivity, $T$ the temperature 
and $Q_V$ a possible volume heat source. Eq.2 is the standard averaged form of the energy conservation equation used in non-stationary phase 
change problems. Once the variational form has been discretized in space and time, two possible resolution schemes emerge: the first is an 
explicit forward Euler scheme which gives rise to a linear equation where the temperature is known at time $t$, $T^t$. This requires very small 
time steps in the current context, which limits the solver’s usability at the scale of industrial applications. The second scheme is the 
backward Euler or full implicit discretization where terms are function of $T^{t+\Delta t}$. It leads to a nonlinear equation with 2 interdependent 
unknowns, $\rhoh^{t+\Delta t}$ and  $T^{t+\Delta t}$. It is clear that the nature of the temperature-enthalpy relationship plays a central 
role when formulating the resolution strategy of this nonlinear equation. Generally, it is admitted that, depending on the resolution strategy, 
it is necessary to express enthalpy as a function of temperature or vice-versa, together with associated partial derivatives, 
$\frac{d \avg{\rho h}}{dT}$ or $\frac{dT }{d\avg{\rho h}}$.

\subsection{The temperature-enthalpy relationship} 
In solidification problems, additional variables are involved in Eq. 1 and Eq. 2, like the transformation path that defines the history of the phase 
fractions, as well as the average chemical composition $\wi$, i being the index of the chemical species (only the solutes are considered). The 
temperature-enthalpy relation averaged over the phases in a given RVE writes:
	HERE	Eq. 3
	
Note that the volume average enthalpy is approximated by the product $\avg{\rho h}^{\phi}=\avg{\rho}^{\phi}\avg{h}^{\phi}$ in the current work. As stated in the introduction, it becomes 
clear from Eq. 3 that phase properties, i.e. average phase density, , HERE and enthalpy, HERE, are temperature and composition dependent. This equation 
is the key to convert the average volume enthalpy to temperature (through a procedure named H2T) or vice-versa (T2H). The values of the different phase 
fractions $g^\phi$ (solidification path) and phase enthalpies $\avg{\rho h}^{\phi}$ are thus needed to close the relation.

\subsection{Tabulation of properties}
The complexity of performing a thermodynamic conversion is directly linked to the simplicity of determining the alloy properties, namely the 
phase fractions and phase enthalpies. In the case of binary alloys and with several assumptions with respect to the system (e.g., linear monovariant 
temperature composition relationships, constant heat capacity of phases and constant latent heat of transformations, equilibrium approximations between 
phases) analytical calculations are often used to determine the properties. Nevertheless, analytical relations are more complex or even impossible to 
derive in the case of multicomponent alloys (i>1). To overcome this problem, one can resort to thermodynamic databases and phase equilibrium calculations 
to tabulate the transformation paths and the phase enthalpies for a given range of temperatures and average compositions. It is a handy solution for two 
main reasons: first, the conversion is merely a binary search in a table; secondly, it is a simple solution for coupling with macrosegregation. In this way, 
phase fractions $g^\phi$ are tabulated as functions of temperature and average composition, while for each phase $\phi$ the mass enthalpy, $\avg{h}^{\phi}$, and 
the density, $\avg{\rho}^{\phi}$, are tabulated as functions of temperature and phase intrinsic average compositions HERE, as well as other possible parameters. 
Figure 1 summarizes the steps in order to perform a temperature-to-enthalpy (T2H) conversion using the predefined tabulation approach. In step 1, the 
transformation path is acquired for each average composition and temperature to determine the list of phases, their volume fractions $g^\phi$ and their intrinsic 
compositions $\avg{w_i}^{\phi}$. In step 2, the phase enthalpy $\avg{h}^{\phi}$  and density $\avg{\rho}^{\phi}$ are determined by searching for the temperature 
and the already known phase composition $\wi^\phi$. In step 3, the average volume enthalpy is computed from the volume fraction, density and mass enthalpy 
of phases using Eq. 3.

FIGURE 1 (Eq. 3)

The methodology to build the tabulations is straightforward. It is based on two main scans. On the one hand, intervals for the variation of the 
average composition $\wi$ are chosen from the known alloy composition. These variations have to cover the extreme values adopted during the 
simulation, which are not known a priori. An interval is also selected for the variation of temperature. The latter is easier to determine as it
usually starts from the initial melt temperature and goes down to the room temperature in a standard casting simulation. For these intervals, a 
systematic scan is made with chosen steps in each composition and T, during which a thermodynamic equilibrium is computed. The outputs are the 
number of phases encountered, together with their fraction and intrinsic composition. The minimum and maximum intrinsic composition for each phase 
could then be determined. On the other hand, for each phase, a scan of the intrinsic composition and temperature is made to compute the intrinsic 
properties. The same temperature interval and step as defined earlier are used.
 
Regarding the enthalpy-to-temperature conversion (H2T), a backward iterative T2H search is performed. For a known composition $\wi$, denoting $k$
the iteration index to convert the enthalpy $\avg{\rho h}_{\text{input}}$, we start with an initial guess for temperature $T^((k=0) )$ then convert it to an 
enthalpy  HERE with the T2H conversion. Using an appropriate nonlinear algorithm (Brent is the most versatile in our case), we aim at minimizing 
the following residual: HERE. Once the algorithm has converged, the temperature $T^((k) )$ is the result of the H2T conversion. 
It is inferred that the first conversion (T2H) is a direct one whereas the latter (H2T) is indirect and requires a series of iterative steps; each step 
being a single T2H resolution. In other words, a H2T conversion is a backward search for a temperature, hence it’s slower. This conversion’s speed lag is 
exacerbated when tabulations increase in size (e.g. large number of temperature and composition steps) and complexity (e.g., multicomponent industrial alloys 
used in casting), since the search gets more complicated with the increasing number of input columns (one column for each alloying element).

\section{Formulation}
from HvsT article

