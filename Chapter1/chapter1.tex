\chapter{Modelling Review}
\begin{nolinkcolors} 
\minitoc
\end{nolinkcolors}
\newpage

\section{Introduction}
Divide into 2 families of models: with and without level set. Regarding the second family of models,
the level set method has been applied on several occasions, but in a different way. Some references
apply it to track the solid-liquid interface, a situation more commonly known as the "Stefan problem".
The scope such applications mainly encompasses dendritic modelling and simulation \\
\blue{SOURCES:} \\
\url{http://www.sciencedirect.com/science/article/pii/S0021999105002603} \\
\url{http://physbam.stanford.edu/~fedkiw/papers/stanford2002-04.pdf} \\
Other references, in relevance to our scope, apply this method to track the surface of the metal
while going from the liquid state to the solid state, in contact with the surrounding gas which is usually air.
%---------------------------------------------------
%
%
\section{Modelling macrosegreation}
%-------------
%
%
\subsection{Dendritic growth}
%-------
In a casting process, the chill surface i.e. the contact between the molten alloy and relatively cold moulds, is the first area to solidify. 
Thermal gradient, $G$, and cooling rate $R$ are two crucial process parameters that define the interface speed $\vstar$, which in turn,
affects the initial microstructure. Although it may be not easy to control them, it remains important to understand their implication in solidification.

The solid-liquid interface fluctuates when solidifying, thus perturbations may appear on the front, locally destabilizing it. 
Two outcome scenarios are possible.
The first scenario is characterized by low values of $\vstar$ where the interface maintains a planar shape, hence we speak of \emph{planar growth}. 
With this kind of growth, a random protuberance appearing at the interface, has a low tip velocity (low driving force of solidification). As such,
the rest of the interface catches up, keeping the planar geometry.
In another scenario, where a real casting is considered, the interface speed is greater in general, due to high solidification rate.
The protuberance tip will be pulled into a liquid less rich in solute than the interface. The zone ahead of the solid-liquid interface is constitutionally undercooled \citep{tiller_redistribution_1953}, giving a greater driving force for the protuberance to grow in the direction
of the thermal gradient. As it has a tree-like shape, we speak of \emph{dendritic growth}. Near the chill surface, dendrites are columnar, with a 
favourable growth in the <100> direction for alloys with cubic lattices, but different orientations are also reported in the literature \citep[see][289]{dantzig_solidification_2009}.
If temperature is uniform, which the case usually far from mould walls, a similar dendritic growth phenomenon occurs, but with an equiaxed morphology.

Columnar dendrites are characterized by a primary spacing, $\pdas$, between the main trunks, and a secondary spacing, $\sdas$, for the arms that are perpendicular
to the trunks. It should be noted that $\sdas$, together with the grain size, are two important microstructural parameters in the as-cast microstructure \citep{easton_grain_2011}.
Further branching may occur but will not be discussed here.
%----------------------
\begin{figureth}
% textwidth 
{0.5}
%path 
{Chapter1/Graphics/dendritic3.png}
% caption
{Schematic of a) a protuberance growing on the solid-liquid interface with b) the corresponding composition profiles 
(Reproduced and adapted from \citet{doitpoms_dissemination_2000}, \doitpoms)}
% label
\label{fig:dendritic_growth}
\end{figureth}
%-----------------------------------
%
%
\subsection{Mush permeability}
The dendritic geometry is crucial in solidification theory as it exhibits lower solid fraction compared to a microstructure formed by planar growth.
This fact has consequences in the fluid-structure interaction in the mushy zone, namely the liquid flow through dendrites. At the chill surface,
the solid  grows gradually from dispersed growing nuclei, to a permeable solid skeleton until finally grains have fully grown with the end of phase change.
The intermediate state where liquid can flow in and out of the mushy zone through the dendrites is a key phenomenon from a rheological perspective.
%By defining a \emph{coherency temperature} $\Tcoh$, one can distinguish two behaviors. 
%For temperatures below $\Tcoh$ (usually at low solid fractions), the liquid flow is characterized by an intrinsic velocity. 
%As for temperature above $\Tcoh$,
%For the rest of this section, we will focus on the first behavior where liquid has a relative freedom to move inside the mushy zone.
The flow through the solid skeleton is damped by primary and secondary dendrites, resulting in momentum dissipation just like in saturated porous media. 
The famous \citet{darcy_les_1856} law relates the pressure gradient ($\nabvec p$) to the fluid velocity $\vec{v}$, through the following equation \citep{rappaz_numerical_2003}:
%----------------------
\begin{align}
\label{eq:darcy}
& \vec{v} = \frac{\K}{\mu} \nabvec p
\end{align}
%----------------------
where $\mu$ is the liquid dynamic viscosity and $\K$ is the permeability tensor. The latter parameter has been the subject of numerous studies that aimed
to predict it from various microstructural or morphological parameters.
Some of these studies has started even before the first attempts to model macrosegregation by \citet{flemings_macrosegregation:_1967, flemings_macrosegregation:_1968-1,flemings_macrosegregation:_1968}. Basically, all models include the solid fraction, $\gs$, as input 
to predict mush permeability along with empirical data. An instance of such models is the work of \citet{xu_gravity-_1991}.
Some models rely additionally on the primary dendrite arm spacing $\pdas$ like Blake-Kozeny \citep{ramirez_evaluation_2003}, or the secondary 
dendrite arm spacing $\sdas$ like Carman-Kozeny, as a meaningful parameter to determine an isotropic permeability. 
Other models like \citet{poirier_permeability_1987,felicelli_simulation_1991} derive an anisotropic permeability based on both $\pdas$ and $\sdas$.

The present work uses Carman-Kozeny as a constitutive model for the permeability scalar (zero order tensor):
%----------------------
\begin{align}
\label{eq:permeability}
& \K  = \frac{\sdas^2 \gl^3}{180\brac{1-\gl}^2}
\end{align}
%-----------------------------------
%
%
\subsection{Microsegregation}
Microsegregation is a fundamental phenomenon in solidification. The simplest definition would be 
an uneven distribution of solute between liquid and the herein growing solid, at the microscopic scale
of the interface separating these phases. If we consider a binary alloy, then the solubility limit is 
the key factor that dictates the composition at which a primary solid phase exists in equilibrium. 
The segregation (or partition) coefficient $\k$ determines the extent of solute rejection into the liquid during solidification:
%----------------------
\begin{align}
\label{eq:partition}
& \k = \frac{\Csstar}{\Clstar}
\end{align}
%----------------------
where $\Csstar$ and $\Clstar$ are the compositions of the solid and liquid respectively, at the interface. When the 
segregation coefficient is less than unity (such is the case for most alloys during dendritic solidification), 
the first solid forms with a composition $\k \Clstar=\k \Cnominal$ less than the liquid's 
composition $\Cnominal$, the latter being initially at the nominal composition, $\Cnominal$. \Cref{fig:binary_diag} illustrates a typical binary 
phase diagram where the real solidus and liquidus are represented by solid lines, while the corresponding linear approximations are in dashed lines.
For most binary alloys, this linearisation simplifies derivation of microsegregation models, as $\k$ becomes independent of temperature.

For each phase, the relationship between the composition at the interface and that in the bulk depends on the chemical homogenisation ability of the phase.
The more homogeneous a phase, the closer the concentrations between the interface and the bulk, hence closer to equilibrium.
%The higher the diffusion coefficient of the phase, $\Dphi$, the closer the concentrations between the interface and the bulk. 
It is thus essential to study the effect of homogenisation on the segregation behaviour and the subsequent effect on solidification, which leads the formalism of microsegregation models.
%----------------------
\begin{figureth}
% textwidth 
{0.25}
%path 
{dummy.pdf}
% caption
{Simplified binary phase diagram}
% label
\label{fig:binary_diag}
\end{figureth}
%-----------------------------------
%
%
\subsubsection{Microsegregation models}
Solid formation depends greatly on the ability of chemicals species to diffuse within each of the solid and liquid phases, but also across the
solid-liquid interface. Furthermore, chemical diffusion like all other diffusional process, is a time-dependent phenomenon. One can thus conclude that two factors
influence the amount of solid formation: cooling rate and diffusion coefficients. However, convection and other mechanical mixing sources, 
homogenise the composition much faster than atomic diffusion. As such, \emph{complete mixing} in the liquid is always an acceptable assumption, regardless of the 
solidification time. We may speak of infinite diffusion in the liquid. Nevertheless, diffusion in the solid also known as \emph{back diffusion}, is the only transport mechanism with very low diffusion coefficients. Therefore, chemical species require a long time, i.e. low cooling rate, to completely diffuse within the solid.
The difference in diffusional behaviour is summarized by two limiting segregation models of perfect equilibrium and nonequilibrium at the scale
of a secondary dendrite arm, which are the lever rule and Gulliver-Scheil models, respectively. Afterwards, models with finite back diffusion are presented. 
%------------
%
%
\subsubsection*{Lever rule}
The lever rule considers an ideal equilibrium in all phases, i.e. solidification is extremely slow, hence phase compositions are 
homogeneous ($ \Clstar = \Cl$ and $ \Csstar = \Cs$ ) at all times as a consequence of complete mixing. 
These compositions are given by:
%----------------------
\begin{align}
\label{eq:leverrule}
& \Cl= \Clstar = \k \Csstar = \k \Cs \\
& \Cs= \Csstar = \frac{\Cnominal}{\k(1-\fs) + \fs}
\end{align}
%----------------------
At the end of solidification, the composition of the solid phase is equal to the nominal composition, $\Cs = \Cnominal$
%------------
%
%
\subsubsection*{Gulliver-Scheil}
The other limiting case is the absence of diffusion in the solid. That includes also the diffusion at the interface, so nothing diffuses in or out. The consequence is a steady increase of the homogeneous liquid composition while the solid composition remains non-uniform.
Compared to a full equilibrium approach, higher fractions of liquid
will remain until eutectic composition is reached, triggering a eutectic solidification. The phase compositions are given by:
%----------------------
\begin{align}
\label{eq:scheil}
& \Cl= \Clstar = \k \Csstar \\
& \Cs= \k \Cnominal (1-\fs)^{1-\k}
\end{align}
%------------
%
%
\subsubsection*{Finite back diffusion}
It has been concluded that the assumption of a negligeable back diffusion overestimates the liquid composition
and the resulting eutectic fraction. Therefore, many models studied the limited diffusion in the solid. One of the earliest models is the Brody-Flemings models \citep{khan_influence_2014} that is a based on a differential solute balance equation for a parabolic growth rate, as follows:
%----------------------
\begin{align}
\label{eq:brodyflemings}
& \Cl= \Clstar = \k \Csstar \\
& \Cs= \k \Cnominal \crochet{ 1 - \brac{ 1 - 2 \Fos \k } \fs }^{\frac{\k-1}{1 - 2 \Fos \k }}
\end{align}
%------------
where $\Fos$ is the dimensionless \emph{Fourier number} for diffusion in the solid \citep{dantzig_solidification_2009}. It depends on the 
solid diffusion coefficient $\Ds$, solidification time $\tsolidif$ and the secondary dendrite arm spacing, as follows: 
%----------------------
\begin{align}
\label{eq:fouriersolid}
& \Fos = \frac{\Ds \tsolidif}{\brac{\sdas / 2}^2}
\end{align}
%------------
Several other models were since suggested and used. The interested reader is referred to the following non exhaustive list of publications: \citet{clyne_solute_1981,kobayashi_solute_1988,ni_volume-averaged_1991,wang_multiphase_1993,combeau_modeling_1996,martorano_solutal_2003,tourret_generalized_2009}.
%------------------------------------
%
%
\subsection{Macroscopic solidification model: monodomain}
In this section, we will present the macroscopic conservations equations that enable us to predict 
macrosegregation in the metal when the latter is the only domain in the system.
%-------------
%
%
\subsubsection{Volume averaging}
It is crucial for a solidification model to represent phenomena on the microscale, then scale up to predict 
macrscopic phenomena. Nevertheless, the characteristic length of a small scale in solidification may represent a dendrite arm spacing, for instance the mushy zone permeability, as it may also represent an atomic distance if one is interested, for instance, in the growth competition between diffusion and surface energy of the solid-liquid interface. Modelling infinitely small-scale phenomena could be prohibitively expensive in computation time, if the we target industrial scales. 

The volume averaging is a technique that allows bypassing this barrier by averaging
small-scale variations on a so-called \emph{representative volume element} (RVE) \citep{dantzig_solidification_2009} with the 
following dimensional constraints on its volume, \rev:
the element should be large enough to "see" and average microscopic fluctuations whilst being smaller than the scale of macroscopic variations.
Solid and liquid may exist simultaneously in the RVE, but no gas phase is considered (volume saturation: $\Vs+\Vl=\rev$). 
Moreover, temperature is assumed uniform and equal for all the phases.
The formalism, introduced by \citet{ni_volume-averaged_1991}, is summarized by the following equations for any physical quantity $\psi$:
%----------------------
\begin{align}
\label{eq:volumeaveraging1}
& \avg{\psi} = \frac{1}{\rev} \integral{\rev}{\psi}{\Omega} = \avg{\psi^s} + \avg{\psi^l}
\end{align}
%------------
where $\psis$ and $\psil$ are phase averages of $\psi$. Then, for any phase $\phi$, one can introduce the \emph{phase intrinsic average} of $\psi$, denoted $\avg{\psi}^\phi$, by writing:
%----------------------
\begin{align}
\label{eq:volumeaveraging2}
& \avg{\psi^\phi} = \frac{1}{\rev} \integral{\Vphi}{\psi}{\Omega} = \gphi \avg{\psi}^\phi
\end{align}
%------------
where $\gphi$ is the volume fraction of the phase. To finalize, the averaging is applied to temporal and spatial derivation operators \citep{rivaux_simulation_2011}:
\begin{align}
\label{eq:volumeaveraging3}
& \avg{ \frac{\partial \psi}{\partial t} ^\phi } = \frac{\partial \avg{\psi^\phi}}{\partial t} - \integral{\gammastar}{\psi^\phi \vstar \cdot \vec{n^\phi}}{\Gamma} \\
\label{eq:volumeaveraging4}
& \avg{ \nabvec \psi^\phi } = \nabvec \avg{\psi^\phi} + \integral{\gammastar}{\psi^\phi \vec{n^\phi}}{\Gamma}
\end{align}
%------------
where $\vstar$ is the local relative interface velocity and $\gammastar$ is the solid-liquid interface, 
while $n^\phi$ is the normal to $\gammastar$, directed outwards. The surface integral term in 
\cref{eq:volumeaveraging3,eq:volumeaveraging4} is an \emph{interfacial average} 
that expresses interfacial exchanges between the phases. The previous equations will 
be used to derive a set macroscopic conservation equations. 
It is noted that the intrinsic average $\avg{\psi}^\phi$ may be replaced by ${\psi}^\phi$ 
for notation simplicity, whenever the averaging technique applies.
%
%
%----------------------------------
\subsubsection{Macroscopic equations}
%-----------
A monodomain macroscopic model relies on four main conservation equations to predict 
macrosegregation in a single alloy domain, i.e. the latter is considered without 
any interaction with another alloy or ambient air. The general form of a conservation 
equation of any physical quantity $\psi$ is given by \citep{rappaz_numerical_2003}:
%-------
\begin{align}
\label{eq:conservationequation}
& \frac{\partial \psi}{\partial t} + \nabla \cdot \brac{\psi \vec{v}} - \nabla \cdot \vec{j_{\psi}} = Q_{\psi}
\end{align}
%-------
The first LHS term in \cref{eq:conservationequation} represents the time variation of $\psi$, the second term accounts for transport by advection while the third is the diffusive transport and the RHS term represents a volume source.
The considered equations are mass, energy, liquid momentum and species conservation, all summarized in \cref{tab:conservationeqs}. The solid momentum is not considered as we assume a fixed and rigid solid phase ($\vs=\vec{0}$).
%-----
\begin{tableth}
\centering

\caption{Summary of conservation equations with their variables}
{\tabulinesep=1.0mm
\begin{tabu}{ p{5cm} p{1cm} p{1cm} p{1cm}}
\tabucline[1pt]{-}
Conservation Equation & $\psi$ & $\vec{j_{\psi}}$ & $Q_{\psi}$ \\\tabucline[1pt]{-}
%-----------------------------
Mass				& 	$\avg{\rho}$  		& $-$ 						& $-$		\\
Energy				& 	$\avg{\rho h}$  	& $\avg{\vec{q}}$ 			& $-$		\\
Species				& 	$\avg{\rho w_i}$  	& $\avg{\vec{j_i}}$ 		& $-$		\\
Liquid momentum		&	$\avg{\rho \vl}$  	& $-\avg{\mat{\sigma^l}}$ 	& $\Fv$		\\\tabucline[1pt]{-} %\vec{\Gamma}^{l}
%-----------------------------
\end{tabu}}
\label{tab:conservationeqs}
\end{tableth}
%-----
We develop the ingredients of these equations using the averaging technique, as follows:
%-------
\begin{align}
\label{eq:mainingredients}
& \avg{\rho} = \gl \rhol + \gs \rhos \\
& \avg{\rho \vec{v}} = \gl \rhol \vl + \cancel{\gs \rhos \vs} \\
& \avg{\rho h} = \gl \rhol \hl + \gs \rhos \hs \\
& \avg{\rho h \vec{v}} = \gl \rhol \hl \vl + \cancel{\gs \rhos \hs \vs} \\
& \avg{\rho w_i} = \gl \rhol \wil + \gs \rhos \wis \\
& \avg{\rho w_i \vec{v}} = \gl \rhol \wil \vl + \cancel{\gs \rhos \wis \vs}
\end{align}
%-------
% $\mat{\sigma}$ 
%-------
Next we define the average diffusive fluxes, $\vec{q}$ for energy and $\vec{j_i}$ for solutes, using Fourier's conduction law and Fick's first law respectively:
%------
\begin{align}
\label{eq:fourierlaw}
& \avg{\vec{q}} = - \gl \kl \nabvec T -  \gs \ks \nabvec T 	\\
\label{eq:ficklaw}
& \avg{\vec{j_i}} = - \gl \Dl \nabvec \wil - \cancel{\gs \Ds \nabvec \wis}
\end{align}
%------------
In \cref{eq:ficklaw}, the solid diffusion coefficient is neglected, 
by considering that for macroscopic scales, the average composition of the alloy is much more influenced by advective 
and diffusive transport in the liquid.
%------------

%%-------------------------------------------------------------------------------------------
Macro models:
\begin{itemize}
\item Rivaux ?
\item Gu beckermann 1999 ?
\end{itemize}
%%----
MICRO MACRO: 
\begin{itemize}
\item Tommy Carozzani (direct)
\item P. Thévoz, J.-L. Desbiolles, M. Rappaz, Metallurgical and Materials TransactionsA 20 (2) (1989) 311–322
\item guo  beckermann 2003
\item Combeau 2009
\item Miha Zaloznik 2010 (indirect)
\end{itemize}
end by talking about taking air into account and the need for an interface capturing method
%%----
\section{Eulerian and Lagrangian motion description}
In mechanics, it is possible to describe motion using two well-known motion description: Eulerian and Lagrangian descrptions.
To start with the latter, it describes the motion of a particle by attributed a reference frame that moves with the particle.
In other words, the particle itself is the center of a reference frame moving at the same speed during time. As such, the total variation
of any physical quantity $\psi$ related to the particle can be found by deriving with respect to time:
%------
\begin{align}
\label{eq:lagrangian}
& \frac{\mathrm{d} \psi}{\mathrm{d} t} = \frac{\partial \psi}{\partial t}
\end{align}
%------------
In contrast to the Lagrangian description, the Eulerian description considers a 
fixed reference frame and independent of the particle's trajectory. The total variation of $\psi$
cannot be simply described by a temporal derivative, since the particle's velocity is not known to 
the reference frame, and thus the velocity effect, namely the advective transport of $\psi$ should also be considered as follows:
%------
\begin{align}
\label{eq:eulerian}
& \frac{\mathrm{d} \psi}{\mathrm{d} t} = \frac{\partial \psi}{\partial t} + \underbrace{\vec{v} \cdot \nabvec \psi}_{\substack{\text{Advective} \\ \text{Transport}}}  
\end{align}
%------------
The importance of these motion descriptions is essential to solve mechanics, whether for fluids or solids, using a numerical method like the finite 
element method (FEM). One of the main steps of this method is to spatially discretise a continuum into a grid of points (nodes, vertices ...), where any 
physical field shall be accordingly discretized. Now, if we focus on a node where velocity has a non zero value and following the previously made analysis,
two outcomes are possible: either the node would be fixed (Eulerian) or it would move by a distance proportional to the prescribed velocity (Lagrangian).
As a consequence, points located on the boundaries constantly require an update of the imposed boundary conditions.

From these explanations, one can deduce that a Eulerian Lagrangian is suited for fluid mechanics problems where 
velocities are high and may distort the mesh points, whereas the Lagrangian is better suited for solid mechanics 
problems where deformation velocities are relatively low and should well behave when predicting strains.

Another motion description has emerged some decades ago, \citet{hirt_arbitrary_1971} call it the Arbitrary Langrangian-Eulerian (ALE) method. 
ALE combines advantages from both previous descriptions. 


\section{Solidification models with level set}
Should I mention the use of level set in mould filling, which comes before solidification \\
Talk about the models used for welding processes. 

Pure MACRO models:
\begin{itemize}
\item Solidification: Du 2001 (double casting technique)
\item Welding: olivier desmaison
\item Welding: mickael from lorient 
\end{itemize}
Aside from the welding applications, check these articles
\url{http://www.tandfonline.com/doi/abs/10.1080/10407790050051137#.VF_gLvnF_kU} \\
\url{http://www.math.pku.edu.cn/pzhang/publication/2001_SDCTULSM.pdf}

MESO MACRO: Shijia Chen (CAFE+LS)

\section{The level set method (LSM)}
How it is defined, Heaviside, mixing laws, transport and reinitialization
\comment{ in the article 2004SunBeckermann, in the introduction there is a small discussion
about the importance of the diffuse interface thickness, check references 3 and 10}

\subsection{Transport and reiniliaztion}
Strong and weak form of transport \\
Numerical stability \\
\subsubsection{Convective reinitilization and Hamilton-Jacobi equations}
\subsubsection{Geometric reinitialization}

\subsection{Interface Remeshing}
Importance when using a static level set and more importantly when LS is transported,
influence of mixing area \emph{thickness} and \emph{resolution} (i.e. nb of nodes with the area),
Isotropic or anisotropic ? the first is more important to composition calculation while the second
is more relevant if we mean do thermohydraulics without macrosegregation

\subsection{Mixing Laws}