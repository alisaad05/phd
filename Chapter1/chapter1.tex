\chapter{Modelling Review}
\begin{nolinkcolors} 
\minitoc
\end{nolinkcolors}
\newpage

In this chapter the following points are discussed
\begin{itemize}
\item what does a typical solidification problem consist of ? heat - fluid - solid 
- chemical species
\item Presence of AIR requires a new problem definition : Lagrangian or Eulerian framework
\end{itemize}

\section{Introduction}
Divide into 2 families of models: with and without level set. Regarding the second family of models,
the level set method has been applied on several occasions, but in a different way. Some references
apply it to track the solid-liquid interface, a situation more commonly known as the "Stefan problem".
The scope such applications mainly encompasses dendritic modelling and simulation \\
\blue{SOURCES:} \\
\url{http://www.sciencedirect.com/science/article/pii/S0021999105002603} \\
\url{http://physbam.stanford.edu/~fedkiw/papers/stanford2002-04.pdf} \\
Other references, in relevance to our scope, apply this method to track the surface of the metal
while going from the liquid state to the solid state, in contact with the surrounding gas which is usually air. \\

%---------------------------------------------------
\section{Modelling macrosegreation}
%-------------
\subsection{Microsegregtion}

\subsubsection{What is microsegregation?}


\subsubsection{Dendritic growth}

\subsubsection{Microsegregation models}
\subsubsection*{Lever rule}
\subsubsection*{Gulliver Scheil}
\subsubsection*{Finite back diffusion}


\subsection{Volume averaging}


\subsection{Macroscopic model}
end by talking about taking air into account and the need for an interface capturing method


\section{Motion description}
\subsection{Langrangian description}
\subsection{Eulerian description}
\subsection{Arbitrary Langrangian-Eulerian}
a little history \citep{hirt_arbitrary_1971} \\ 
The ALE method combines advantages from both previous descriptions. Explain how the position mesh nodes can be
updated with velocity or fixed. However, this description will be more suitable for configuration with deformable solid. 
Further discussions are presented in the perspectives.

\section{Standard models: without level set}
A section presenting the main FE equations along with their weak formulations that will be solved 
in the metal being a single domain. I call it "standard" because it doesnt contain anything about 
levelsets, compressibility, ...

Pure MACRO (without LS)
\begin{itemize}
\item Energy (chapter 1)
\item Species mass (voller prakash) \comment{should I mention the tabulation approach that I couldnt finalize because of the equality between w and wl in liquid phase ?}
\item Fluid mechanics (vms: darcy model with boussinesq)
\end{itemize}
\comment{talking about Eulerian approach Air Metal will be presented in the next chapters, it should be the biblio section of another chapter}

MESO MACRO (without LS): 
\begin{itemize}
\item Tommy Carozzani
\item Miha Zaloznik
\end{itemize}

\section{Solidification models with level set}
Should I mention the use of level set in mould filling, which comes before solidification \\
Talk about the models used for welding processes. 

Pure MACRO models:
\begin{itemize}
\item Solidification: Du 2001 (double casting technique)
\item Welding: olivier desmaison
\item Welding: mickael from lorient 
\end{itemize}
Aside from the welding applications, check these articles
\url{http://www.tandfonline.com/doi/abs/10.1080/10407790050051137#.VF_gLvnF_kU} \\
\url{http://www.math.pku.edu.cn/pzhang/publication/2001_SDCTULSM.pdf}

MESO MACRO: Shijia Chen (CAFE+LS)

\section{The level set method (LSM)}
How it is defined, Heaviside, mixing laws, transport and reinitialization
\comment{ in the article 2004SunBeckermann, in the introduction there is a small discussion
about the importance of the diffuse interface thickness, check references 3 and 10}

\subsection{Transport and reiniliaztion}
Strong and weak form of transport \\
Numerical stability \\
\subsubsection{Convective reinitilization and Hamilton-Jacobi equations}
\subsubsection{Geometric reinitialization}

\subsection{Interface Remeshing}
Importance when using a static level set and more importantly when LS is transported,
influence of mixing area \emph{thickness} and \emph{resolution} (i.e. nb of nodes with the area),
Isotropic or anisotropic ? the first is more important to composition calculation while the second
is more relevant if we mean do thermohydraulics without macrosegregation

\subsection{Mixing Laws}