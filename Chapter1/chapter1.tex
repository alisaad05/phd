\chapter{Modelling Review}
\begin{nolinkcolors} 
\minitoc
\end{nolinkcolors}
\newpage

\section{Introduction}
Divide into 2 families of models: with and without level set. Regarding the second family of models,
the level set method has been applied on several occasions, but in a different way. Some references
apply it to track the solid-liquid interface, a situation more commonly known as the "Stefan problem".
The scope such applications mainly encompasses dendritic modelling and simulation \\
\blue{SOURCES:} \\
\url{http://www.sciencedirect.com/science/article/pii/S0021999105002603} \\
\url{http://physbam.stanford.edu/~fedkiw/papers/stanford2002-04.pdf} \\
Other references, in relevance to our scope, apply this method to track the surface of the metal
while going from the liquid state to the solid state, in contact with the surrounding gas which is usually air.
%---------------------------------------------------
\section{Modelling macrosegreation}
%-------------
\subsection{Dendritic growth}
In a casting process, the chill surface i.e. the contact between the molten alloy and relatively cold moulds, is the first area to solidify. 
Thermal gradient, $G$, and cooling rate $R$ are two crucial process parameters that define the interface speed $\vstar$, which in turn,
affects the initial microstructure. Although it may be not easy to control them, it remains important to understand their implication in solidification.

The solid-liquid interface fluctuates when solidifying, thus perturbations may appear on the front, locally destabilizing it. 
Two outcome scenarios are possible.
The first scenario is characterized by low values of $\vstar$ where the interface maintains a planar shape, hence we speak of \emph{planar growth}. 
With this kind of growth, a random protuberance appearing at the interface, has a low tip velocity (low driving force of solidification). As such,
the rest of the interface catches up, keeping the planar geometry.
In another scenario, where a real casting is considered, the interface speed is greater in general, due to high solidification rate.
The protuberance tip will be pulled into a liquid less rich in solute than the interface. The zone ahead of the solid-liquid interface is constitutionally undercooled \citep{tiller_redistribution_1953}, giving a greater driving force for the protuberance to grow in the direction
of the thermal gradient. As it has a tree-like shape, we speak of \emph{dendritic growth}. Near the chill surface, dendrites are columnar, with a 
favourable growth in the <100> direction for alloys with cubic lattices, but different orientations are also reported in the literature (TODO SOURCE). %TODO 
If temperature is uniform, which the case usually far from mould walls, a similar dendritic growth phenomenon occurs, but with an equiaxed morphology.

Columnar dendrites are characterized by a primary spacing, $\pdas$, between the main trunks, and a secondary spacing, $\sdas$, for the arms that are perpendicular
to the trunks. It should be noted that $\sdas$, together with the grain size, are two important microstructural parameters in the as-cast microstructure \citep{easton_grain_2011}.
Further branching may occur but will not be discussed here.
%----------------------
\begin{figureth}
% textwidth 
{0.5}
%path 
{Chapter1/Graphics/dendritic3.png}
% caption
{Schematic of a) a protuberance growing on the solid-liquid interface with b) the corresponding composition profiles 
(Reproduced and adapted from \citet{doitpoms_dissemination_2000}, \doitpoms)}
% label
\label{fig:dendritic_growth}
\end{figureth}
%----------------------
%-----------------------------------
\subsection{Mush permeability}
The dendritic geometry is crucial in solidification theory as it exhibits lower solid fraction compared to a microstructure formed by planar growth.
This fact has consequences in the fluid-structure interaction in the mushy zone, namely the liquid flow through dendrites. At the chill surface,
the solid  grows gradually from dispersed growing nuclei, to a permeable solid skeleton until finally grains have fully grown with the end of phase change.
The intermediate state where liquid can flow in and out of the mushy zone through the dendrites is a key phenomenon from a rheological perspective.
%By defining a \emph{coherency temperature} $\Tcoh$, one can distinguish two behaviors. 
%For temperatures below $\Tcoh$ (usually at low solid fractions), the liquid flow is characterized by an intrinsic velocity. 
%As for temperature above $\Tcoh$,
%For the rest of this section, we will focus on the first behavior where liquid has a relative freedom to move inside the mushy zone.
The flow through the solid skeleton is damped by primary and secondary dendrites, resulting in momentum dissipation just like in saturated porous media. 
The famous \citet{darcy_les_1856} law relates the pressure gradient ($\nabvec p$) to the fluid velocity $\vec{v}$, through the following equation \citep{rappaz_numerical_2003}:
%----------------------
\begin{align}
\label{eq:darcy}
& \vec{v} = \frac{\K}{\mu} \nabvec p
\end{align}
%----------------------
where $\mu$ is the liquid dynamic viscosity and $\K$ is the permeability tensor. The latter parameter has been the subject of numerous studies that aimed
to predict it from various microstructural or morphological parameters.
Some of these studies has started even before the first attempts to model macrosegregation by \citet{flemings_macrosegregation:_1967, flemings_macrosegregation:_1968-1,flemings_macrosegregation:_1968}. Basically, all models include the solid fraction, $\gs$, as input 
to predict mush permeability along with empirical data. An instance of such models is the work of \citet{xu_gravity-_1991}.
Some models rely additionally the primary dendrite spacing $\pdas$ like Blake-Kozeny \citep{ramirez_evaluation_2003}, or the secondary 
dendrite spacing $\sdas$ like Carman-Kozeny, as a meaningful parameter to determine an isotropic permeability. 
Other models like \citet{poirier_permeability_1987,felicelli_simulation_1991} derive an anisotropic permeability based on both $\pdas$ and $\sdas$.

The present work uses Carman-Kozeny as a constitutive model for the permeability scalar (zero order tensor):
%----------------------
\begin{align}
\label{eq:permeability}
& \K  = \frac{\sdas^2 \gl^3}{180\brac{1-\gl}^2}
\end{align}
%----------------------
%-----------------------------------
\subsection{Microsegregation}
Microsegregation is a fundamental phenomenon in solidification. The simplest definition would be 
an uneven distribution of solute between liquid and the herein growing solid, at the microscopic scale
of the interface separating these phases. If we consider a binary alloy, then the solubility limit is 
the key factor that dictates the composition at which a primary solid phase exists in equilibrium. 
The segregation (or partition) coefficient $\k$ determines the extent of solute rejection into the liquid during solidification:
%----------------------
\begin{align}
\label{eq:partition}
& \k = \frac{\Csstar}{\Clstar}
\end{align}
%----------------------
where $\Csstar$ and $\Clstar$ are the compositions of the solid and liquid respectively, at the interface. When the 
segregation coefficient is less than unity (such is the case for most alloys during dendritic solidification), 
the first solid forms with a composition $\k \Clstar=\k \Cnominal$ less than the liquid's 
composition $\Cnominal$, the latter being initially at the nominal composition, $\Cnominal$. \Cref{fig:binary_diag} illustrates a typical binary 
phase diagram where the real solidus and liquidus are represented by solid lines, while the corresponding linear approximations are in dashed lines.
For most binary alloys, this linearisation simplifies derivation of microsegregation models, as $\k$ becomes independent of temperature.

For each phase, the relationship between the composition at the interface and that in the bulk depends on the chemical homogenisation ability of the phase.
The more homogeneous a phase, the closer the concentrations between the interface and the bulk, hence closer to equilibrium.
%The higher the diffusion coefficient of the phase, $\Dphi$, the closer the concentrations between the interface and the bulk. 
It is thus essential to study the effect of homogenisation on the segregation behaviour and the subsequent effect on solidification, which leads the formalism of microsegregation models.
%----------------------
\begin{figureth}
% textwidth 
{0.25}
%path 
{dummy.pdf}
% caption
{Simplified binary phase diagram}
% label
\label{fig:binary_diag}
\end{figureth}
%-----------------------------------
\subsubsection{Microsegregation models}
Solid formation depends greatly on the ability of chemicals species to diffuse within each of the solid and liquid phases, but also across the
solid-liquid interface. Furthermore, chemical diffusion like all other diffusional process, is a time-dependent phenomenon. One can thus conclude that two factors
influence the amount of solid formation: cooling rate and diffusion coefficients. However, convection and other mechanical mixing sources, 
homogenise the composition much faster than by diffusion. As such, \emph{complete mixing} is always an acceptable assumption, regardless of the diffusion time 
or coefficient...
\subsubsection*{Lever rule}
\subsubsection*{Gulliver Scheil}
\subsubsection*{Finite back diffusion}
brody-flemings, 
wang beckermann, 
martorano 2003 (CET) , 
hamoudaGandin (multicomponent), 
combeau2006 (check gouttebroze) H. Combeau, J.-M. Drezet, A. Mo, M. Rappaz, Metallurgical and MaterialsTransactions A 27 (8) (1996) 2314–2327, 
damien tourret (predicition of peritectic with undercooling: aslan u cannot predict without undercooling 3al arja7) 

\subsection{Volume averaging}

\subsection{Macroscopic model}
\begin{itemize}
\item Energy (chapter 1)
\item Species mass (voller prakash) \comment{should I mention the tabulation approach that I couldnt finalize because of the equality between w and wl in liquid phase ?}
\item Fluid mechanics (vms: darcy model with boussinesq)
\end{itemize}

Macro models:
\begin{itemize}
\item Rivaux ?
\item Gu beckermann 1999 ?
\end{itemize}

MICRO MACRO: 
\begin{itemize}
\item Tommy Carozzani (direct)

\item P. Thévoz, J.-L. Desbiolles, M. Rappaz, Metallurgical and Materials TransactionsA 20 (2) (1989) 311–322
\item guo  beckermann 2003
\item Combeau 2009
\item Miha Zaloznik 2010 (indirect)
\end{itemize}



end by talking about taking air into account and the need for an interface capturing method


\section{Motion description}
\subsection{Langrangian description}
\subsection{Eulerian description}
\subsection{Arbitrary Langrangian-Eulerian}
a little history \citep{hirt_arbitrary_1971} \\ 
The ALE method combines advantages from both previous descriptions. Explain how the position mesh nodes can be
updated with velocity or fixed. However, this description will be more suitable for configuration with deformable solid. 
Further discussions are presented in the perspectives.

\section{Standard models: without level set}
A section presenting the main FE equations along with their weak formulations that will be solved 
in the metal being a single domain. I call it "standard" because it doesnt contain anything about 
levelsets, compressibility, ...

\section{Solidification models with level set}
Should I mention the use of level set in mould filling, which comes before solidification \\
Talk about the models used for welding processes. 

Pure MACRO models:
\begin{itemize}
\item Solidification: Du 2001 (double casting technique)
\item Welding: olivier desmaison
\item Welding: mickael from lorient 
\end{itemize}
Aside from the welding applications, check these articles
\url{http://www.tandfonline.com/doi/abs/10.1080/10407790050051137#.VF_gLvnF_kU} \\
\url{http://www.math.pku.edu.cn/pzhang/publication/2001_SDCTULSM.pdf}

MESO MACRO: Shijia Chen (CAFE+LS)

\section{The level set method (LSM)}
How it is defined, Heaviside, mixing laws, transport and reinitialization
\comment{ in the article 2004SunBeckermann, in the introduction there is a small discussion
about the importance of the diffuse interface thickness, check references 3 and 10}

\subsection{Transport and reiniliaztion}
Strong and weak form of transport \\
Numerical stability \\
\subsubsection{Convective reinitilization and Hamilton-Jacobi equations}
\subsubsection{Geometric reinitialization}

\subsection{Interface Remeshing}
Importance when using a static level set and more importantly when LS is transported,
influence of mixing area \emph{thickness} and \emph{resolution} (i.e. nb of nodes with the area),
Isotropic or anisotropic ? the first is more important to composition calculation while the second
is more relevant if we mean do thermohydraulics without macrosegregation

\subsection{Mixing Laws}