\chapter{Modelling Review}
\begin{nolinkcolors} 
\minitoc
\end{nolinkcolors}
\newpage

%***********************************************
\section{Modelling macrosegreation}
%***********************************************
%
%***********************************************
\subsection{Dendritic growth}
%***********************************************
In a casting process, the chill surface i.e. the contact between the molten alloy and relatively cold moulds, is the first area to solidify. 
Thermal gradient, $G$, and cooling rate $R$ are two crucial process parameters that define the interface speed $\vstar$, which in turn,
affects the initial microstructure. Although it may be not easy to control them, it remains important to understand their implication in solidification.

The solid-liquid interface fluctuates when solidifying, thus perturbations may appear on the front, locally destabilizing it. 
Two outcome scenarios are possible.
The first scenario is characterized by low values of $\vstar$ where the interface maintains a planar shape, hence we speak of \emph{planar growth}. 
With this kind of growth, a random protuberance appearing at the interface, has a low tip velocity (low driving force of solidification). As such,
the rest of the interface catches up, keeping the planar geometry.
In another scenario, where a real casting is considered, the interface speed is greater in general, due to high solidification rate.
The protuberance tip will be pulled into a liquid less rich in solute than the interface. The zone ahead of the solid-liquid interface is constitutionally undercooled \citep{tiller_redistribution_1953}, giving a greater driving force for the protuberance to grow in the direction
of the thermal gradient. As it has a tree-like shape, we speak of \emph{dendritic growth}. Near the chill surface, dendrites are columnar, with a 
favourable growth in the <100> direction for alloys with cubic lattices, but different orientations are also reported in the literature \citep[see][289]{dantzig_solidification_2009}.
If temperature is uniform, which is the case usually far from mould walls, a similar dendritic growth phenomenon occurs, but with an equiaxed morphology.

Columnar dendrites are characterized by a primary spacing, $\pdas$, between the main trunks, and a secondary spacing, $\sdas$, for the arms that are perpendicular
to the trunks. It should be noted that $\sdas$, together with the grain size, are two important microstructural parameters in the as-cast microstructure \citep{easton_grain_2011}.
Further branching may occur but will not be discussed here.
%----------------------
\begin{figureth}
% textwidth 
{0.75}
%path 
{Chapter1/Graphics/ProtuberanceReal.pdf} % Protuberance % ProtuberanceReal
% caption
{Schematic of a) a protuberance growing on the solid-liquid interface with b) the corresponding composition profiles 
(Reproduced and adapted from \citet{doitpoms_dissemination_2000}, \doitpoms).}
% label
\label{fig:dendritic_growth}
\end{figureth}
%-----------------------------------
%***********************************************
\subsection{Mush permeability}
%***********************************************
The dendritic geometry is crucial in solidification theory as it exhibits lower solid fraction compared to a microstructure formed by planar growth.
This fact has consequences on the fluid-structure interaction in the mushy zone, namely the liquid flow through dendrites. At the chill surface,
the solid  grows gradually from dispersed growing nuclei to a permeable solid skeleton, until finally grains have fully grown at the end of phase change.
The intermediate state where liquid can flow in and out of the mushy zone through the dendrites is a key phenomenon from a rheological perspective.
%By defining a \emph{coherency temperature} $\Tcoh$, one can distinguish two behaviors. 
%For temperatures below $\Tcoh$ (usually at low solid fractions), the liquid flow is characterized by an intrinsic velocity. 
%As for temperature above $\Tcoh$,
%For the rest of this section, we will focus on the first behavior where liquid has a relative freedom to move inside the mushy zone.
The flow through the solid skeleton is damped by primary and secondary dendrites, resulting in momentum dissipation just like in saturated porous media. 
The famous \citet{darcy_les_1856} law relates the pressure gradient ($\nabvec p$) to the fluid velocity $\vec{v}$, through the following equation \citep{rappaz_numerical_2003}:
%----------------------
\begin{align}
\label{eq:darcy}
& \vec{v} = \frac{\K}{\mul} \nabvec p
\end{align}
%----------------------
where $\mul$ is the liquid dynamic viscosity and $\K$ is the permeability tensor. The latter parameter has been the subject of numerous studies that aimed
to predict it from various microstructural or morphological parameters.
Some of these studies has started even before the first attempts to model macrosegregation by \citet{flemings_macrosegregation:_1967, flemings_macrosegregation:_1968-1,flemings_macrosegregation:_1968}. Basically, all models include the solid fraction, $\gs$, as input 
to predict mush permeability along with empirical data. An instance of such models is the work of \citet{xu_gravity-_1991}.
Some models rely additionally on the primary dendrite arm spacing $\pdas$ like Blake-Kozeny \citep{ramirez_evaluation_2003}, or the secondary 
dendrite arm spacing $\sdas$ like Carman-Kozeny, as a meaningful parameter to determine an isotropic permeability. 
Other models like \citet{poirier_permeability_1987,felicelli_simulation_1991} derive an anisotropic permeability based on both $\pdas$ and $\sdas$.

The present work uses Carman-Kozeny as a constitutive model for the isotropic permeability scalar (zero order tensor):
%----------------------
\begin{align}
\label{eq:permeability}
& \K  = \frac{\sdas^2 \gl^3}{180\brac{1-\gl}^2}
\end{align}
%-----------------------------------
%
%***********************************************
\subsection{Microsegregation}
%***********************************************
Microsegregation is a fundamental phenomenon in solidification. The simplest definition is 
an uneven distribution of solute between the liquid and the herein growing solid, at the microscopic scale
of the interface separating these phases. If we consider a binary alloy, then the solubility limit is 
the key factor that dictates the composition at which a primary solid phase exists in equilibrium. 
The segregation (or partition) coefficient $\k$ determines the extent of solute rejection into the liquid during solidification:
%----------------------
\begin{align}
\label{eq:partition}
& \k = \frac{\Csstar}{\Clstar}
\end{align}
%----------------------
where $\Csstar$ and $\Clstar$ are the compositions of the solid and liquid respectively, at the interface. When the 
segregation coefficient is less than unity (such is the case for most alloys during dendritic solidification), 
the first solid forms with a composition $\Csstar=\k \Clstar=\k \Cnominal$ less than the liquid's 
composition $\Cnominal$, the latter being initially at the nominal composition, $\Cnominal$. \Cref{fig:binary_diag} illustrates a typical binary 
phase diagram where the real solidus and liquidus are represented by solid lines, while the corresponding linear approximations are in grey dashed lines.
For most binary alloys, this linearisation simplifies derivation of microsegregation models, as $\k$ becomes independent of temperature.

For each phase, the relationship between the composition at the interface and that in the bulk depends on the chemical homogenisation ability of the phase.
The more homogeneous a phase, the closer the concentrations between the interface and the bulk, hence closer to equilibrium.
%The higher the diffusion coefficient of the phase, $\Dphi$, the closer the concentrations between the interface and the bulk. 
It is thus essential to study the effect of homogenisation on the segregation behaviour and the subsequent effect on solidification, which leads the formalism of microsegregation models.
%----------------------
\begin{figureth}
% textwidth 
{0.5}
%path 
{Chapter1/Graphics/phasediagram.pdf}
% caption
{Typical eutectic phase diagram of a binary alloy showing the real solidus and liquidus at full equilibrium,
with the corresponding linear approximations (grey dashed lines). $T_m$ and $T_E$ are respectively the melting point
of the solvent and the eutectic temperature.}
% label
\label{fig:binary_diag}
\end{figureth}
%-----------------------------------
%
%***********************************************
\subsubsection{Microsegregation models}
%***********************************************
Solid formation depends greatly on the ability of chemicals species to diffuse within each of 
the solid and liquid phases, but also across the solid-liquid interface. Furthermore, chemical diffusion like all other 
diffusional process, is a time-dependent phenomenon. One can thus conclude that two factors
influence the amount of solid formation: cooling rate and diffusion coefficients. However, 
convection and other mechanical mixing sources, homogenise the composition much faster than atomic diffusion. 
As such, \emph{complete mixing} in the liquid is always an acceptable assumption, regardless of the 
solidification time. We may speak of infinite diffusion in the liquid. Nevertheless, diffusion in the solid, 
also known as \emph{back diffusion}, is the only transport mechanism with very low diffusion coefficients. 
Therefore, chemical species require a long time, i.e. low cooling rate, to completely diffuse within the solid.
The difference in diffusional behaviour at the scale of a secondary dendrite arms, is summarized by two limiting 
segregation models of perfect equilibrium  and nonequilibrium, which are the lever rule and Gulliver-Scheil models, respectively. 
Afterwards, models with finite back diffusion are presented. 
%------------
%
%***********************************************
\subsubsection*{Lever rule}
%***********************************************
The lever rule considers an ideal equilibrium in all phases, i.e. solidification is extremely slow, hence phase compositions are 
homogeneous ($ \Clstar = \Cl$ and $ \Csstar = \Cs$ ) at all times as a consequence of complete mixing. 
These compositions are given by:
%----------------------
\begin{align}
\label{eq:leverrule}
& \Cl= \Clstar = \k \Csstar = \k \Cs \\
& \Cs= \Csstar = \frac{\Cnominal}{\k(1-\fs) + \fs}
\end{align}
%----------------------
At the end of solidification, the composition of the solid phase is equal to the nominal composition, $\Cs = \Cnominal$
%------------
%
%***********************************************
\subsubsection*{Gulliver-Scheil}
%***********************************************
The other limiting case is the absence of diffusion in the solid. That includes also the diffusion at the interface, so nothing diffuses in or out. The consequence is a steady increase of the homogeneous liquid composition while the solid composition remains non-uniform.
Compared to a full equilibrium approach, higher fractions of liquid
will remain until eutectic composition is reached, triggering a eutectic solidification. The phase compositions are given by:
%----------------------
\begin{align}
\label{eq:scheil}
& \Cl= \Clstar = \k \Csstar \\
& \Cs= \k \Cnominal (1-\fs)^{1-\k}
\end{align}
%------------
%
%***********************************************
\subsubsection*{Finite back diffusion}
%***********************************************
It has been concluded that the assumption of a negligeable back diffusion overestimates the liquid composition
and the resulting eutectic fraction. Therefore, many models studied the limited diffusion in the solid. One of the earliest models is the Brody-Flemings models \citep{khan_influence_2014} that is a based on a differential solute balance equation for a parabolic growth rate, as follows:
%----------------------
\begin{align}
\label{eq:brodyflemings}
& \Cl= \Clstar = \k \Csstar \\
& \Cs= \k \Cnominal \crochet{ 1 - \brac{ 1 - 2 \Fos \k } \fs }^{\frac{\k-1}{1 - 2 \Fos \k }}
\end{align}
%------------
where $\Fos$ is the dimensionless \emph{Fourier number} for diffusion in the solid \citep{dantzig_solidification_2009}. It depends on the 
solid diffusion coefficient $\Ds$, solidification time $\tsolidif$ and the secondary dendrite arm spacing, as follows: 
%----------------------
\begin{align}
\label{eq:fouriersolid}
& \Fos = \frac{\Ds \tsolidif}{\brac{\sdas / 2}^2}
\end{align}
%------------
Several other models were since suggested and used. The interested reader is referred to the following non 
exhaustive list of publications: \citet{clyne_solute_1981,kobayashi_solute_1988,ni_volume-averaged_1991,wang_multiphase_1993,
combeau_modeling_1996,martorano_solutal_2003,tourret_generalized_2009}. It is noted that some of these publications consider 
also a finite diffusion in the liquid phase.
%------------------------------------
%
%
%***********************************************
\subsection{Macroscopic solidification model: monodomain} \label{sec:monodomain}
%***********************************************
In this section, we will present the macroscopic conservations equations that enable us to predict 
macrosegregation in the metal when the latter is the only domain in the system.
%-------------
%
%***********************************************
\subsubsection{Volume averaging} \label{sec:volumeavg}
%***********************************************
It is crucial for a solidification model to represent phenomena on the microscale, then scale up to predict 
macrscopic phenomena. Nevertheless, the characteristic length of a small scale in solidification may represent a dendrite arm spacing, for instance the mushy zone permeability, as it may also represent an atomic distance if one is interested, for instance, in the growth competition between diffusion and surface energy of the solid-liquid interface. Modelling infinitely small-scale phenomena could be prohibitively expensive in computation time, if we target industrial scales. 

The volume averaging is a technique that allows bypassing this barrier by averaging
small-scale variations on a so-called \emph{representative volume element} (RVE) \citep{dantzig_solidification_2009} of volume \rev, with the 
following dimensional constraints:
the element should be large enough to "see" and average microscopic fluctuations whilst being smaller than the scale of macroscopic variations.
Solid and liquid may exist simultaneously in the RVE, but no gas phase is considered (volume saturation: $\Vs+\Vl=\rev$). 
Moreover, temperature is assumed uniform and equal for all the phases.
The formalism, introduced by \citet{ni_volume-averaged_1991}, is summarized by the following equations for any physical quantity $\psi$:
%----------------------
\begin{align}
\label{eq:volumeaveraging1}
& \avg{\psi} = \frac{1}{\rev} \integral{\rev}{\psi}{\Omega} = \avg{\psi^s} + \avg{\psi^l}
\end{align}
%------------
where $\psis$ and $\psil$ are phase averages of $\psi$. Then, for any phase $\phi$, one can introduce the \emph{phase intrinsic average} of $\psi$, denoted $\avg{\psi}^\phi$, by writing:
%----------------------
\begin{align}
\label{eq:volumeaveraging2}
& \avg{\psi^\phi} = \frac{1}{\rev} \integral{\Vphi}{\psi}{\Omega} = \gphi \avg{\psi}^\phi
\end{align}
%------------
where $\gphi$ is the volume fraction of the phase. To finalize, the averaging is applied to temporal and spatial derivation operators \citep{rivaux_simulation_2011}:
\begin{align}
\label{eq:volumeaveraging3}
& \avg{ \frac{\partial \psi}{\partial t} ^\phi } = \frac{\partial \avg{\psi^\phi}}{\partial t} - \integral{\gammastar}{\psi^\phi \vstar \cdot \vec{n^\phi}}{A} \\
\label{eq:volumeaveraging4}
& \avg{ \nabvec \psi^\phi } = \nabvec \avg{\psi^\phi} + \integral{\gammastar}{\psi^\phi \vec{n^\phi}}{A}
\end{align}
%------------
where $\vstar$ is the local relative interface velocity and $\gammastar$ is the solid-liquid interface, 
while $n^\phi$ is the normal to $\gammastar$, directed outwards. The surface integral term in 
\cref{eq:volumeaveraging3,eq:volumeaveraging4} is an \emph{interfacial average} 
that expresses exchanges between the phases across the interface. The previous equations will 
be used to derive a set macroscopic conservation equations. 
It is noted that the intrinsic average $\avg{\psi}^\phi$ may be replaced by ${\psi}^\phi$ 
for notation simplicity, whenever the averaging technique applies.
%
%
%***********************************************
\subsubsection{Macroscopic equations}
%***********************************************
A monodomain macroscopic model relies on four main conservation equations to predict 
macrosegregation in a single alloy domain, i.e. the latter is considered without 
any interaction with another alloy or ambient air. The general form of a conservation 
equation of any physical quantity $\psi$ is given by \citep{rappaz_numerical_2003}:
%-------
\begin{align}
\label{eq:conservationequation}
& \frac{\partial \psi}{\partial t} + \nabla \cdot \brac{\psi \vec{v}} - \nabla \cdot \vec{j_{\psi}} = Q_{\psi}
\end{align}
%-------
The first LHS term in \cref{eq:conservationequation} represents the time variation of $\psi$, the second term accounts for transport by advection while the third is the diffusive transport and the RHS term represents a volume source.
The considered equations are mass, energy, liquid momentum and species conservation, all summarized in \cref{tab:conservationeqs}. The solid momentum is not considered as we assume a fixed and rigid solid phase ($\vs=\vec{0}$).
%-----
\begin{tableth}
\centering

\caption{Summary of conservation equations with their variables.}
{\tabulinesep=1.5mm
\begin{tabu}{ p{5cm} p{1cm} p{1cm} p{1cm}}
\tabucline[1pt]{-}
Conservation Equation & $\psi$ & $\vec{j_{\psi}}$ & $Q_{\psi}$ \\\tabucline[1pt]{-}
%-----------------------------
Mass				& 	$\avg{\rho}$  		& $-$ 						& $-$		\\
Energy				& 	$\avg{\rho h}$  	& $\avg{\vec{q}}$ 			& $-$		\\
Species				& 	$\avg{\rho w_i}$  	& $\avg{\vec{j_i}}$ 		& $-$		\\
Liquid momentum		&	$\avg{\rho \vl}$  	& $-\avg{\sigmal}$ 			& $\Fv$		\\\tabucline[1pt]{-} %\vec{\Gamma}^{l}
%-----------------------------
\end{tabu}}
\label{tab:conservationeqs}
\end{tableth}
%-----
We develop the ingredients of these equations using the averaging technique, as follows:
%-------
\begin{align}
\label{eq:mainingredients1}
& \avg{\rho} = \gl \rhol + \gs \rhos \\
\label{eq:mainingredients2}
& \avg{\rho \vec{v}} = \gl \rhol \vl + \cancel{\gs \rhos \vs} \\
\label{eq:mainingredients3}
& \avg{\rho h} = \gl \rhol \hl + \gs \rhos \hs \\
\label{eq:mainingredients4}
& \avg{\rho h \vec{v}} = \gl \rhol \hl \vl + \cancel{\gs \rhos \hs \vs} \\
\label{eq:mainingredients5}
& \avg{\rho w_i} = \gl \rhol \wil + \gs \rhos \wis \\
\label{eq:mainingredients6}
& \avg{\rho w_i \vec{v}} = \gl \rhol \wil \vl + \cancel{\gs \rhos \wis \vs}
\end{align}
%-------
% $\mat{\sigma}$ 
%-------
Next, we define the average diffusive fluxes, $\vec{q}$ for energy and $\vec{j_i}$ for solutes, using Fourier's conduction law and Fick's first law, respectively:
%------
\begin{align}
\label{eq:fourierlaw}
& \avg{\vec{q}} = - \gl \kl \nabvec T -  \gs \ks \nabvec T 	\\
\label{eq:ficklaw}
& \avg{\vec{j_i}} = - \gl \Dl \nabvec \wil - \cancel{\gs \Ds \nabvec \wis}
\end{align}
%------------
In \cref{eq:ficklaw}, the solid diffusion coefficient is neglected, 
by considering that for macroscopic scales, the average composition of the alloy is much more influenced by advective 
and diffusive transport in the liquid.
In \cref{eq:fourierlaw}, we assumed that phases are are thermal equilibrium, that is, temperature is uniform in the RVE.

Now that the main conservation equations ingredients are properly defined, we may write each averaged conservation
equations as the sum of two local conservation equations for each phase in the RVE, hence introducing also interfacial average terms.
For instance, the local mass balance in each phase is given by:
%----
\begin{subequations}
\begin{align}
\label{eq:mass_liquid}
& \temp{\gl \rhol} + \nabla \cdot \brac{\gl \rhol \vl} = S_V \langle \rhol \vlstar \cdot \vec{n} \rangle^* - S_V \langle \rhol \vstar \cdot \vec{n} \rangle^* \\
\label{eq:mass_solid}
& \temp{\gs \rhos} + \nabla \cdot \brac{\gs \rhos \vs} = - S_V \langle \rhos \vsstar \cdot \vec{n} \rangle^* + S_V \langle \rhos \vstar \cdot \vec{n} \rangle^*
\end{align}
\end{subequations}
%----
where $S_V= A_{sl}/\rev$ is the specific surface area, $\vec{v}^{l^*}$ and $\vec{v}^{s^*}$ are respectively, the liquid 
and solid phase velocity at the interface and $\vec{v}^*$ is the previously introduced solid-liquid interface velocity. 
For instance, the first interfacial exchange term in the RHS of \cref{eq:mass_liquid} is expanded as follows \citep{dantzig_solidification_2009}:
%----
\begin{subequations}
\begin{align}
S_V \langle \rhol \vlstar \cdot \vec{n} \rangle^*
&= \frac{A_{sl}}{\rev} \brac{ \frac{1}{A_{sl}} \integral{A_{sl}}{\rhol \vlstar \cdot \vec{n}}{A}} \\
&= \frac{1}{\rev}  \integral{A_{sl}}{\rhol \vlstar \cdot \vec{n}}{A}
\end{align}
\end{subequations}
%----
Summing equations \eqref{eq:mass_liquid} and \eqref{eq:mass_solid}, results in the overall mass balance
in the RVE:
%----
\begin{multline}
\label{eq:conservation_mass_local}
 \temp{\gl \rhol + \gs \rhos}  +  \nabla \cdot \brac{\gl \rhol \vl + \gs \rhos \vs} = \\  
	  S_V \avg{\rhol \brac{\vlstar-\vstar} \cdot \vec{n}}^*  
   -  S_V \avg{\rhos \brac{\vsstar-\vstar} \cdot \vec{n}}^*
\end{multline}
%----
where the RHS cancels to zero as shown by \citet{ni_volume-averaged_1991}. Moreover, the authors show that with their averaging technique, 
interfacial exchanges for energy, chemical species and momentum cancel out as they are equal in absolute value but opposite in sign. 
Using \cref{eq:mainingredients1,eq:mainingredients2,eq:mainingredients3,eq:mainingredients4,eq:mainingredients5,eq:mainingredients6,eq:fourierlaw,eq:ficklaw}
and following the same procedure done in \cref{eq:conservation_mass_local}, the averaged mass balance hence writes:
%------
\begin{align}
\label{eq:conservation_mass}
& \frac{\partial \avg{\rho}}{\partial t} + \nabla \cdot \avg{\rho \vec{v}} = 0
\end{align}
%------------
whereas the averaged energy balance writes:
%------
\begin{align}
\label{eq:conservation_energy_init}
& \frac{\partial \avg{\rho h}}{\partial t} + \nabla \cdot \avg{\rho h \vec{v}} - \nabla \cdot \brac{\avg{\kappa} \nabvec T} = 0
\end{align}
%------------
and finally the species balance writes:
%------
\begin{align}
\label{eq:conservation_solute_init}
& \frac{\partial \avg{\rho w_i}}{\partial t} + \nabla \cdot \avg{\rho w_i \vec{v}} - \nabla \cdot \brac{\gl \Dl \nabvec \brac{\rhol \wil}}= 0
\end{align}
%------------
As stated previously, the momentum balance in the solid phase is not taken into consideration, hence we do not sum the corresponding 
local conservation equations. This has consequences on the advection terms in energy and species conservation, and later on we will show
the consequences on the momentum conservation in the liquid. First, the advection terms in \cref{eq:conservation_energy_init,eq:conservation_solute_init}
shall be redefined by considering that the fluid is incompressible $\brac{\nabla \cdot \vit = 0}$, which yields:
%------
\begin{align}
\label{eq:advection_no_solid}
& \nabla \cdot \avg{\rho h \vec{v}} = \vit \cdot \nabvec \brac{\rhol \hl} \\
& \nabla \cdot \avg{\rho w_i \vec{v}} = \vit \cdot \nabvec \brac{\rhol \wil}
\end{align}
%------------
As for the liquid momentum balance, we write:
%------
\begin{align}
\label{eq:conservation_momentumliq_init}
& \temp{ \rhol \gl \vl } + \nabvec \cdot \brac{\rhol \gl \vl \times \vl} = 
	\nabvec \cdot \brac{\gl \sigmal} +\gl \Fv + \vec{\Gamma}^{l}
\end{align}
%------------
where $\Fv$ is the vector of external body forces exerted on the liquid phase. In our case, it accounts for the fluid's weight:
%------
\begin{align}
\label{eq:fluid_weight}
& \Fv = \rhol \gravity
\end{align}
%------------
The interfacial momentum transfer between the solid and liquid phases in \cref{eq:conservation_momentumliq_init} is modelled 
by a momentum flux vector $\vec{\Gamma}^{l}$, consisting of hydrostatic and deviatoric parts, such that:
%------------
%\begin{subequations}
\begin{align}
	\label{eq:interfacial_momentum_total}
	& \vec{\Gamma}^{l} =  \vec{\Gamma}_{p}^{l} + \vec{\Gamma}_{\mathbb{S}}^{l}	\\
	\label{eq:interfacial_momentum_hydrostatic}
	& \vec{\Gamma}_{p}^{l} = p^{l^{*}} \nabvec g^{l} = p^{l} \nabvec g^{l}		\\
	\label{eq:interfacial_momentum_deviatoric}
	& \vec{\Gamma}_{\mathbb{S}}^{l} = - \gl^2 \mul \K^{-1} \brac{ \vl - \cancel{\vs} }  
\end{align}
%\end{subequations}
%------------
where $p^{l^{*}}$ is the pressure at the interface, considered to be equal to the liquid hydrostatic 
pressure $\pl$, $\K$ is a permeability scalar (isotropic) computed using \cref{eq:darcy} and $\mul$ is the liquid's 
dynamic viscosity. The general form of the Cauchy liquid stress tensor in \cref{eq:conservation_momentumliq_init} 
is decomposed as follows: 
%------------
%\begin{subequations} 
\begin{align}
\label{eq:stress_liq}
& \avg{\sigmal} = \gl \sigmal = - \brac{\avg{\pl} -\lambda \nabla \cdot \vit} \I + \avg{\mat{\mathbb{S}}^{l}}
\end{align}
%\end{subequations}
%------------
where $\lambda$ is a dilatational viscosity \citep{dantzig_solidification_2009} and $\mat{\mathbb{S}}^{l}$ is the
liquid strain deviator tensor. In the literature, the coefficient $\lambda$ is taken proportional to the 
viscosity: $\lambda = \frac{2}{3} \mu^l $. However, as we consider an incompressible flow, the divergence 
term vanishes, thus rewriting \cref{eq:stress_liq} as follows:
%------------
\begin{subequations} 
\begin{align}
\label{eq:stress_liq_incompressible1}
& \avg{\sigmal} = - \avg{\pl} \I + \avg{\mat{\mathbb{S}}^{l}} \\
\label{eq:stress_liq_incompressible2}
& \avg{\sigmal} = - \avg{\pl} \I + 2 \mul \strainrate 
\end{align}
\end{subequations}
%------------
where the transition from \cref{eq:stress_liq_incompressible1} to \cref{eq:stress_liq_incompressible2} is made
possible by assuming a Newtonian behaviour for the liquid phase. The strain rate tensor, $\strainrate$, depends on 
the average liquid velocity:
%------------
\begin{align}
\label{eq:tensor_strainrate}
\strainrate = \frac{1}{2} \brac{\nabmat \vit  +  \nabmattransp \vit}
\end{align}
%-------------------------------------------------------------------------------------------
Finally, we obtain the final form of momentum conservation in the liquid phase coupled with the averaged mass balance, by injecting \cref{eq:fluid_weight,eq:interfacial_momentum_hydrostatic,eq:interfacial_momentum_deviatoric,eq:stress_liq_incompressible2,eq:tensor_strainrate}
in \cref{eq:conservation_momentumliq_init}:
%------
% \begin{align}
\begin{multline}
\label{eq:conservation_momentumliq}
 \temp{ \rhol \vit } + \frac{1}{\gl} \nabvec \cdot \brac{\rhol \vit \times \vit} = \\
	  - \gl\nabvec \pl - 2 \mul \nabvec \cdot \brac{\nabmat \vit + \nabmattransp \vit}
	  - \gl \mul \K^{-1} \vit + \gl \rhol \gravity
\end{multline}
% \end{align}
%------------
where we intentionally employed the \emph{superficial velocity}, $\vit = \gl \vl$, as the main unknown, together with the liquid pressure $\pl$.
This system, when modelled in 3D, has a total of 4 unknowns (velocity vector and pressure) and 3 equations ($X$,$Y$ and $Z$ projections for the velocity vector).
A fourth equation provided by the mass balance (\cref{eq:conservation_mass})is therefore added for closure, giving the following system of equations :
%------
\begin{equation}
\label{eq:Navier-Stokes1}
   \left\{
   \begin{aligned}
      & \temp{ \rhol \vit } + \frac{1}{\gl} \nabvec \cdot \brac{\rhol \vit \times \vit} = \\
	  &- \gl\nabvec \pl - 2 \mul \nabvec \cdot \brac{\nabmat \vit + \nabmattransp \vit}
	  - \gl \mul \K^{-1} \vit + \gl \rhol \gravity\\ \\
      & \nabla \cdot \vit =0
    \end{aligned}
    \right.
\end{equation}
%------------
Last, the Boussinesq approximation allows taking a constant density in the inertial terms of \cref{eq:Navier-Stokes1} while the variations responsible for buoyancy forces can be computed
using \cref{eq:rholiq}, if the system is incompressible. Hence, the final set of equations is better known as the incompressible \emph{Navier-Stokes} equations, applied to a solidifying melt:
%------
\begin{equation}
\label{eq:Navier-Stokes2}
   \left\{
   \begin{aligned}
      & \rholref \brac{\tempup{ \vit } + \frac{1}{\gl} \nabvec \cdot \brac{\vit \times \vit}} = \\
	  &- \gl\nabvec \pl - 2 \mul \nabvec \cdot \brac{\nabmat \vit + \nabmattransp \vit}
	  - \gl \mul \K^{-1} \vit + \gl \rhol \gravity\\ \\
      & \nabla \cdot \vit =0
    \end{aligned}
    \right.
\end{equation}
%------------
Since all conservation equations were presented and simplified by the main assumption of a static solid phase, 
we may include them in a graphical summary in \cref{fig:algo_monodomain}
\comment{correct definitions of $\Cl$ and $\Cs$}
%---------------
\begin{figure}[htbp]
\label{fig:algo_monodomain}
%=========
\setlength{\largeur}{2cm}
\setlength{\llargeur}{11cm}
\setlength{\llargeurOut}{2.5cm}
\setlength{\rlargeurOut}{0.25cm}
\centering
%=========
\begin{tikzpicture}[node distance=0.5cm]

\tikzstyle{rect}=[rectangle,draw,text=black, fill=red!10, drop shadow, rounded corners,minimum height=0.01cm]
\tikzstyle{output}=[rectangle,draw,text=black, fill=blue!15, drop shadow, rounded corners,minimum height=0.01cm]
%\tikzstyle{test}=[diamond,aspect=3,draw,text=black]
%\tikzstyle{fleche}=[->,>=stealth]
\tikzstyle{trait}=[]
%=========
\setlength{\rlargeur}{\largeur}
\addtolength{\rlargeur}{-1.\tabcolsep}
\addtolength{\rlargeur}{-1.\llargeur}
%=========
\node[rect] (energy)
{
	\begin{tabular}{p{\llargeur}}
	\textbf{Conservation of energy (Nonlinear Heat Transfer)} 
	\begin{equation*}
		 \frac{\partial \avg{\rho h}}{\partial t} + \vit \cdot \nabla \brac{\rhol \hl} - \nabla \cdot \brac{\avg{\kappa} \nabvec T} = 0
	\end{equation*}
	\end{tabular}
};
%==========
\node[rect,below=of energy] (microseg)
{
	\begin{tabular}{p{\llargeur}}
	\textbf{Microsegregation}
	
	\vspace{3mm}
	Discrete mapping:  $\brac{\gphi , \avg{w_i^{\phi}}^{\phi} } = f\brac{\wi , T }$ \\
	Direct lever rule:	
	\begin{equation*}
   		\left.
		\begin{aligned}	   
 			& \Cl= \Clstar = \k \Csstar = \k \Cs   \\
  			& \Cs= \Csstar = \frac{\Cnominal}{\k(1-\fs) + \fs}
		\end{aligned}
		\right.
	\end{equation*}
	\end{tabular}
};
%==========
\node[rect,below=of microseg] (macroseg)
{
	\begin{tabular}{p{\llargeur}}
	\textbf{Conservation of chemical species (Macrosegregation)} 
	\begin{equation*}
		\frac{\partial \avg{\rho w_i}}{\partial t} + \vit \cdot \nabla \brac{\rhol \wil} - \nabla \cdot \brac{\gl \Dl \nabvec \brac{\rhol \wil}}= 0
	\end{equation*}
	\end{tabular}
};
%==========
\node[rect, below=of macroseg] (mecaflu)
{
	\begin{tabular}{p{\llargeur}}
	\textbf{Conservation of liquid momentum (Navier Stokes)}
	\begin{equation*}
   \left\{
   \begin{aligned}
      & \rholref \brac{\tempup{ \vit } + \frac{1}{\gl} \nabvec \cdot \brac{\vit \times \vit}} = \\
	  &- \gl\nabvec \pl - 2 \mul \nabvec \cdot \brac{\nabmat \vit + \nabmattransp \vit}
	  - \gl \mul \K^{-1} \vit + \gl \rhol \gravity\\ \\
      & \nabla \cdot \vit =0
    \end{aligned}
    \right.
\end{equation*}
	\end{tabular}
};
%%%%%% OUTPUT NODES %%%%%%%%%%%%%%%%
\node[output,right=of energy] (energyOut)
{
	\begin{tabular}{p{\llargeurOut}}
	\begin{equation*}
		 T^t
	\end{equation*}
	\end{tabular}
};
%==========
\node[output,right=of mecaflu] (mecafluOut)
{
	\begin{tabular}{p{\llargeurOut}}
	\begin{equation*}
		 \vit^t , (\pl)^t
	\end{equation*}
	\end{tabular}
};
%==========
\node[output,right=of macroseg] (macrosegOut)
{
	\begin{tabular}{p{\llargeurOut}}
	\begin{equation*}
		 \avg{\rho w_i}^t
	\end{equation*}
	\end{tabular}
};
%==========
\node[output,right=of microseg] (microsegOut)
{
	\begin{tabular}{p{\llargeurOut}}
	\begin{equation*}
   		\left.
		\begin{aligned}	   
 			& (\gphi)^t  \\
  			& (\wil)^t , (\wis)^t
		\end{aligned}
		\right.
	\end{equation*}
	\end{tabular}
};
%==========
%Connectivity
\draw[trait] (energy) -- (microseg);
\draw[trait] (microseg) -- (macroseg);
\draw[trait] (macroseg) -- (mecaflu);
%-----
\draw[trait] (energy) -- (energyOut);
\draw[trait] (mecaflu) -- (mecafluOut);
\draw[trait] (macroseg) -- (macrosegOut);
\draw[trait] (microseg) -- (microsegOut);
%==========
\end{tikzpicture}
\caption{Graphical resolution algorithm of the conservation equations used in a 
monodomain macroscopic model to predict macrosegregation. The blue boxes represent the output of each equation at a time step $t$.}
\end{figure}
%%------------------------------------------------------------------------------------------
%%------------------------------------------------------------------------------------------
%%------------------------------------------------------------------------------------------
\comment{Macro models: Rivaux ? Gu beckermann 1999}

\comment{Micro macro: Tommy Carozzani2013, guo  beckermann 2003,  Combeau 2009, Miha Zaloznik 2010 (indirect), 
P. Thévoz, J.-L. Desbiolles, M. Rappaz, Metallurgical and Materials TransactionsA 20 (2) (1989) 311–322}

\comment{end by talking about taking air into account and the need for an interface capturing method}

%***********************************************
\section{Eulerian and Lagrangian motion description}
%%***********************************************
%
%***********************************************
\subsection{Overview}
%***********************************************
In mechanics, it is possible to describe motion using two well-known motion description: Eulerian and Lagrangian descrptions.
To start with the latter, it describes the motion of a particle by attributing a reference frame that moves with the particle.
In other words, the particle itself is the center of a reference frame moving at the same speed during time. 
The position vector, denoted by $\vec{x}$, is hence updated as follows:
%------
\begin{align}
\label{eq:lagrangian}
& \vec{x}^{(t+1)}  = \vec{x}^{(t)} + \vec{v} \Delta t
\end{align}
%------------
As such, the total variation of any physical quantity $\psi$ related to the particle 
can be found by deriving with respect to time, $\frac{\mathrm{d} \psi}{\mathrm{d} t}$.
%
In contrast to the Lagrangian description, the Eulerian description considers a 
fixed reference frame and independent of the particle's trajectory. The total variation of $\psi$
cannot be simply described by a temporal derivative, since the particle's velocity is not known to 
the reference frame, and thus the velocity effect, namely the advective transport of $\psi$, should also be considered as follows:
%------
\begin{align}
\label{eq:eulerian}
& \frac{\mathrm{d} \psi}{\mathrm{d} t} = \frac{\partial \psi}{\partial t} 
   + \underbrace{\vec{v} \cdot \nabvec \psi}_{\substack{\text{Advective} \\ \text{Transport}}}  
\end{align}
%------------
In this case, the LHS term is also known as \emph{total} or \emph{material derivative}.
The importance of these motion descriptions is essential to solve mechanics, whether for fluids or solids, using a numerical method like the finite 
element method (FEM). One of the main steps of this method is to spatially discretise a continuum into a grid of points (nodes, vertices ...), where any 
physical field shall be accordingly discretized. Now, if we focus on a node where velocity has a non zero value and following the previously made analysis,
two outcomes are possible: either the node would be fixed (Eulerian) or it would move by a distance proportional to the prescribed velocity (Lagrangian).
In the latter case, points located on the boundaries constantly require an update of the imposed boundary conditions.

From these explanations, one can deduce that an Eulerian framework is suited for fluid mechanics problems where 
velocities are high and may distort the mesh points, whereas the Lagrangian framework is better suited for solid mechanics 
problems where deformation velocities are relatively low and should well behave when predicting strains.

Another motion description has emerged some decades ago, \citet{hirt_arbitrary_1971} call it the Arbitrary Langrangian-Eulerian (ALE) method. 
ALE combines advantages from both previous descriptions as it dictates a Lagrangian behavior at "solid" nodes where solid is deforming, and
an Eulerian behavior at "fluid" nodes.
%
%
%***********************************************
\subsection{Interface capturing}
%***********************************************
As no solid deformation is considered in this work, the Eulerian framework is a convenient choice. Although solidification shrinkage is 
to be considered in the current scope, it will deform the alloy's outter surface in contact with the air.
We intend to track this interface and its motion over time via a numerical method. A wide variety of methods accomplish this 
task while they yield different advantages and disadvantages. Such methods fall into two main classes, either interface tracking
or interface capturing, among which we cite: marker-and-cell (MAC) \citep{harlow_numerical_1965}, volume of fluid (VOF) \citep{hirt_volume_1981}, 
phase field methods (PF), level set method (LSM) \citep{osher_fronts_1988}, coupled level set - VOF method and others. 
The interested reader may refer to quick references by \citet{prosperetti_navier-stokes_2002,maitre_review_2006} about these methods.

In the past years, the level set method received a considerable attention in many computational fields, specifically in solidification.
For this reason, we will focus on this method henceforth, giving a brief literature review and technical details in the next sections.
%
%***********************************************
\section{Solidification models with level set}
%***********************************************
In classic solidification problems, the need to track an interface occurs usually at the solid-liquid interface, that is why the phase field
method \citep{karma_phase-field_1996,boettinger_phase-field_2002} and the level set method \citep{chen_simple_1997,gibou_level_2003,tan_level_2007} 
were applied at a microscale to follow mainly the dendritic growth of a single crystal in an undercooled melt.
In our case however, when we mention "solidification models using LSM", we 
do not mean the solid-liquid interface inside the alloy, but it is the alloy(liquid)-air interface that is tracked, assuming that microscale
phenomena between the phases within the alloy, are averaged using the previously defined technique in \cref{sec:volumeavg}.

Very few models were found in the literature, combining solidification and level set as stated previously. 
\citet{du_simulating_2001} applied it to track the interface between two molten alloys in a double casting technique. 
Welding research, on another hand, has been more active adapting the level set methodology to corresponding applications. 
In CEMEF, two projects use the metal-air level set methodology in welding simulations and showed promising results. 
\comment{two sentences not more about the contribution of OD and SC}
Firstly, \citet{desmaison_level_2014} ... \red{TODO}\\ %TODO
Later, \citet{chen_three_2014} applied it to gas metal arc welding (GMAW) to predict 
the grain structure in the heat affected zone essentially.
More recently, \citet{courtois_complete_2014} used the same methodology but this time to predict keyhole defect formation
in spot laser welding. The tracked interface in this case was that between the molten alloy and the corresponding vapor phase.
%----------------------------------
%
%
%***********************************************
\section{The level set method}
%%***********************************************
Firstly introduced by \citet{osher_fronts_1988}, this method became very popular in studying multiphase flows.
It is reminded that the term \emph{multiphase} in computational domains usually refers to multiple fluids, and thus
should not be mixed with definition of a phase in the current solidification context. For disambiguation, we shall
use \emph{multifluid flow} when needed.
The great advantage lies in the way the interface between two fluids, $F_1$ and $F_2$ is implicitly captured, unlike 
other methods where the exact interface position is needed. In a discrete domain, the concept is to assign for each 
mesh node of position vector$\vec{x}$, the minimum distance $d_\Gamma(\vec{x})$ separating it from an interface $\Gamma$. 
The distance function, denoted $\alpha$ and defined in \cref{eq:levelset_defintion}, is then signed positive or negative, based on the fluid or domain 
to which the node belongs.
%------------
\begin{align}
\label{eq:levelset_defintion}
\levelsetx = 
\begin{cases}
  d_\Gamma(\vec{x}) 		& \text{ if } \vec{x} \in F_1 \\ 
 -d_\Gamma(\vec{x})			& \text{ if } \vec{x} \in F_2 \\ 
  0 						& \text{ if } \vec{x} \in \Gamma_{F1, F2} 
\end{cases}
\end{align}
%------------
%----------------------
\begin{figureth}
% textwidth 
{0.5}
%path 
{Chapter1/Graphics/distancefunction.pdf}
% caption
{Schematic of the interface $\Gamma$ (thick black line) of a rising air bubble ($\Omega$) in water. The other contours represent isovalues
of the distance function around and inside the interface contour. Those outside are signed negative whereas inside they are signed
positive.}
% label
\label{fig:distance_function}
\end{figureth}
%-----------------------------------
%
%***********************************************
\subsection{Diffuse interface}
\label{sec:heaviside}
%***********************************************
The level set has many attractive properties that allows seamless implementation in 2D and 3D models. 
It is a continuously differentiable $C^1$-function. For instance, a \emph{Heaviside} function, 
denoted $\heaviside$, can be obtained by first order derivation of the level set function. The 
Heaviside function is continuous but non differentiable, with an abrupt transition from $0$ to $1$ 
across the sharp interface, as follows:
%----------
\begin{align}
\label{eq:no_smoothing}
\heaviside = \heaviside \brac{\levelsetx} = 
\begin{cases}
	0  & \text{ if } \levelsetx < 0 \\
    1  & \text{ if } \levelsetx \geqslant 0 \\  
\end{cases}
\end{align}
%----------
With the help of \cref{eq:no_smoothing}, we can define the geometric 
"presence" of a domain with respect to the interface. As such, material properties 
depend upon this function, which will be discussed later in \cref{sec:mixinglaws}.
It is established that a steep transition can lead to numerical problems, 
so the Heaviside function should be smoothed in a fixed thickness.
Sinusoidal smoothing in \cref{eq:sinusoidal_smoothing} is widely used with level set formulations.
%------------
\begin{align}
\label{eq:sinusoidal_smoothing}
\heaviside= 
\begin{cases}
	0  & \text{ if } \levelsetx < -\varepsilon \\
    1  & \text{ if } \levelsetx > \varepsilon \\  
    \frac{1}{2} \brac{1+ \frac{\levelsetx}{\varepsilon} 
    + \frac{1}{\pi}\sin \brac{\frac{\pi \levelsetx}{\varepsilon }} } & \text{ if } - \varepsilon \leq \levelsetx \leq \varepsilon
\end{cases}
\end{align}
%-----------
where the interval $\crochet{-\varepsilon; +\varepsilon}$ is an artificial interface thickness around the zero distance.
Defining a diffuse interface rather than a sharp one, is also a common approach 
in phase field methods \citep{beckermann_modeling_1999,sun_diffuse_2004}.
It is emphasized that the latter methods give physically meaningful 
analysis of a diffuse interface and the optimal thickness by thoroughly studying the 
intricate phenomena happening at the scale of the interface. However, for level set methods, 
there has not been a formal work leading the same type of analysis. For this reason, 
many aspects of the level set method lack physical meanings but still computationally useful.
In a recent paper by \citet{gada_derivation_2009}, the authors respond partially to this problem 
by analysing and deriving conservation equations using a level set in a more meaningful way, but 
do not discuss the diffuse interface aspect.

The dirac delta function is also an important property to convert surface integrals to volume terms, which could
turn useful when modelling surface tension effects for instance, using 
the \emph{continuum surface force} method (CSF) \citep{brackbill_continuum_1992}.
The dirac function, plotted in \cref{fig:heaviside_dirac} along with the Heaviside function within an interface thickness 
of $\crochet{-\varepsilon;+\varepsilon}$, is derived from the Heaviside as follows:
%--- --------------
\begin{align}
\label{eq:dirac}
\dirac = \delta \brac{\levelsetx} = \frac{\partial \heaviside  }{\partial \levelsetx}=
\begin{cases}
\frac{1}{2\varepsilon} \brac{ 1 + \cos \brac{\frac{\pi \levelsetx}{\varepsilon }} }  & \text{ if } \abval{\levelsetx} \leqslant  \varepsilon \\
    0  & \text{ if } \abval{\levelsetx} > \varepsilon
\end{cases}
\end{align}
%-------------
The Heaviside and delta dirac functions can be readily processed to obtain other geometric properties from the level set,
which are extremely useful. We mention the most relevant ones \citep{peng_pde-based_1999}:
%------------
\begin{align}
\label{eq:levelset_normal}
\text{normal vector}: \vec{n} &= \frac{\nabvec \levelset}{\norm{\nabvec \levelset} } \\
%
\label{eq:levelset_curvature}
\text{curvature}: \zeta &= - \nabla \cdot \vec{n} \\
%
\label{eq:levelset_interface}
\text{surface area of the air-metal interface }: A^\Gamma &= \integral{\Omega}{\dirac \norm{\nabvec \levelset}}{\Omega} \\
%
\label{eq:levelset_volume}
\text{metal volume}: V^M &= \integral{\Omega}{\heaviside^M}{\Omega}
%
\end{align}
%-----------
where, for the last two equations, we considered a three-dimensional domain $\Omega$ containing two subdomains, metal and air, seperated by an interface $\Gamma$.
It is reminded that for a 2D case, \cref{eq:levelset_interface} evaluates a length instead of the area while \cref{eq:levelset_volume} gives the area instead of volume.
Finally, within the diffuse interface, fluids properties 
may vary linearly or not, depending on the mixing law, which is presented in the next section.
%----------------------
\begin{figureth}
% textwidth 
{0.6}
%path 
{Chapter1/Graphics/HeavisideDirac.pdf}
% caption
{Schematic of two level properties inside the diffuse interface: Heaviside (lower x-axis) and Dirac delta (upper x-axis) functions.
Note that the peak of the dirac function depends on the interface thickness to ensure a unity integral of the delta function over $\Omega$. }
%In this example, the interface thickness is 800 $\mu m$.
% label
\label{fig:heaviside_dirac}
\end{figureth}
%-----------------------------------
%
%***********************************************
\subsection{Mixing Laws} 
\label{sec:mixinglaws}
%**********************************************
A \emph{monolithic} resolution style, as opposed to a \emph{partitioned} resolution, is based on solving a single set of equations
for both fluids separated by an interface, as if a single fluid were considered. 
Level set is one many methods that use the monolithic style to derive a single set of conservation equations
for both fluids. The switch from one material to the other is implicitly taken care of by using the Heaviside function as well mixing laws. These laws
are crucial to define how properties vary across the diffuse interface in view of a more accurate resolution.
The most frequently used mixing law in the literature is the arithmetic law. Other transitions are less known such as 
the harmonic and logarithmic mixing. The first law is maybe the most intuitive and most used for properties mixture as it emanates
from VOF-based methods. 
If we consider any property $\psi$, for instance the fluid's dynamic viscosity $\mu$, then the arithmetic 
law will give a mixed property $\mix{\psi}$ as follows:
%---
\begin{align}
\label{eq:arithmetic}
\mix{\psi} = \heaviside^{F_1} \psi^{F_1} + \heaviside^{F_2} \psi^{F_2}
\end{align}
%---
Basically, the result is an average property that follows the same trend as the Heaviside function. As for the harmonic law, it writes:
%---
\begin{align}
\label{eq:harmonic}
\mix{\psi} = \brac{\frac{\heaviside^{F_1}}{\psi^{F_1}} + \frac{\heaviside^{F_2}}{\psi^{F_2}} }^{-1}
\end{align}
%---
and last, the logarithmic law writes:
%---
\begin{align}
\label{eq:logarithmic}
\mix{\psi} =  n^{\brac{\heaviside^{F_1} \log_n \psi^{F_1} + \heaviside^{F_2} \log_n \psi^{F_2}}}
\end{align}
%---
where $n$ is any real number serving as a logarithm base, which often is either the exponential $e$ or $10$.
The mixture result with this law is the same, regardless of the value of $n$.
By looking to \cref{fig:mixinglaws}, we clearly see that the difference between all three approaches is the property weight given to each side 
of the level set in the mixture. The arithmetic law, being symmetric, has equal weights, $\psi^{F_1}$ and $\psi^{F_2}$, in the final mixture. 
Nevertheless, the asymmetric harmonic mixing varies inside the diffuse interface with a dominant weight of one property 
over the other. As for the logarithmic mixture, it can be seen as an intermediate transition between the preceding laws.
%----------------------
\begin{figureth}
% textwidth 
{0.6}
%path
{Chapter1/Graphics/mixinglaws.pdf}
% caption
{Three mixing laws, arithmetic, logarithmic and harmonic commonly used in monolithic formulations.}
% label
\label{fig:mixinglaws}
\end{figureth}
%-----------------------------------
As long as the interface thickness is small enough, the choice of a mixing law should not drastically change the result, 
inasmuch as it depends on the discretisation resolution of the interface. This fact made the arithmetic mixing
the most applied one, because it is symmetric and easy to implement (no handling of potential division problems like harmonic laws for instance). 
However, \citet{strotos_numerical_2008} claim that the harmonic law proves to conserve better diffusive fluxes at the interface.
More recently, an interesting study made by \citet{ettrich_modelling_2014} focused on mixing thermal properties using a phase field method.
They define a diffuse interface in which they separately mix the thermal conductivity, $\kappa$, and the heat capacity, $C_\text{v}$,
then compute the thermal diffusivity as the ratio of these properties. Later, the authors compare the temperature field obtained by diffusion
to a reference case in order to decide which combination of mixing laws gives the best result. Despite not being directly related to a level set 
method, this work gives an insight of the mixture possibilities and their effect on a pure thermal diffusion.
Otherwise, little work has been found in the literature on the broad effects of mixture types on simulation results in a level set context.
%----------------------------------
%
%*****************************************
\section{Interface motion} % reinitialisation % regularisation
%****************************************
When a physical interface needs to have topology changes because
of fluid structure interaction or surface tension for instance, the level set model 
can easily follow these changes by a transport step. The idea is the advect the signed distance
function, its zero isovalue representing the interface and all other distant isovalues, with the
velocity field as input. The motion of the interface is thus expressed by:
%------
\begin{align}
\label{eq:transport_strong}
& \frac{d \levelset}{d t}  = \frac{\partial \levelset}{\partial t} + \vec{v} \cdot \nabvec \levelset = 0
\end{align}
%------------
\comment{Associated boundary conditions}
%
%
\subsection{Level set transport}
The finite element method gives the fully discretised weak form of \cref{eq:transport_strong} 
by using a convenient set of test functions $\levelsettest$ belonging the hilbertian \emph{Sobolev} space:
%-----------
\begin{align}
\label{eq:transport_weak1}
&  \integral{\Omega}{\levelsettest \frac{\partial \levelset}{\partial t} }{\Omega} + \integral{\Omega}{\levelsettest \vec{v} \cdot \nabvec \levelset}{\Omega} = 0
\qquad \forall \levelsettest \in \hilbert \brac{\Omega}   
\end{align}
%------------
The spatial discretisation of $\levelset$ assigns, for each of the total $N$ nodes of a simplex, the following values:
%-----------
\begin{align}
\label{eq:transport_weak2}
&  \levelset = \sum_N  P_j \levelset_j
\end{align}
%------------
Furthermore, with the Galerkin method, we replace test functions by the interpolation functions $P_j$, 
then we apply a temporal discretisation for the main unknowns by a forward finite difference in time.
Consequently, \cref{eq:transport_weak1} can be recast as follows:
%-----------
\begin{subequations}
\begin{align}
\label{eq:transport_weak3}
  \frac{1}{\dt} \brac{\levelset_j^t - \levelset_j^{t-\dt}}  \integral{\Omega}{P_i P_j}{\Omega} 
	+ \levelset_j^t   \integral{\Omega}{\vec{v^t} \cdot \nabvec P_j }{\Omega} &=  0 \\
\crochet{\frac{1}{\dt} \integral{\Omega}{P_i P_j }{\Omega} + \integral{\Omega}{\vec{v^t} \cdot \nabvec P_j }{\Omega}} \levelset_j^t &= 
\frac{1}{\dt}  \integral{\Omega}{ \levelset^{t-\dt} P_i }{\Omega} \\
\label{eq:weakform_transport}
\crochet{\Mij+\Aij} \levelset_j^t &= \Fi
\end{align}
\end{subequations}
%------------
where $\Mij$ and $\Aij$ are respectively the mass (or capacity) matrix and advection matrix, both written within a local 
finite element, whereas $\Fi$ is a local vector of known quantities from the previous time step. The solution of the linear
system in \cref{eq:weakform_transport} is the transported distance function.
%
%
%==============================
\subsection{Level set regularisation}
%==============================
%
Upon transport the distance function field, a crucial property of the level set may be partially or totally 
lost over the domain, which is:
%------
\begin{align}
\label{eq:transport_property}
\begin{cases}
\norm{\nabvec \levelset} = 1 \\ 
\levelset (x,t) = 0 	\qquad  \text{ if } x \in \Gamma(t)
\end{cases}
\end{align}
%------------
The closer this $L^2$-norm to one, the more regular the level set. An irregular distance function induces 
cumulative numerical erros in the transport step (\cref{eq:transport_strong}) and thus results in non 
conservation of mass, because of spurious distance information. 
When the transport equation in \cref{eq:transport_strong} is discretised in time then solved, a \emph{regularisation} 
(also known as \emph{reinitialisation}) is necessary to conserve as much as 
possible the property in \cref{eq:transport_property}.

\Cref{fig:reinit_influence} shows the need of regularisation in two different 
simulations of the same phenomenon: rising air bubble inside
water. The importance of this well studied case \citep{sussman_level_1994,hysing_quantitative_2009}
is that the interface between two fluids is highly deformable
as the bubble rises because of buoyancy, and therefore the task of tracking the 
dynamic interface while maintaining an accurate distance function
is a considerable numerical task. In the first simulation, the distance contours 
are squeezed against the zero-distance contour marked by the 
thick black line. A closer look to the interface reveals undesired distortions, with a 
"wavy" shape at some points. This effect is evidently an artefact
of a level set transport lacking subsequent reinitialisation, inasmuch 
as the surface tension tends to minimise the total surface area and make it as smooth
as possible. Nevertheless, the second simulation unveils much better results, especially 
how the interface shows no sign of destabilisation. We also note the regular spacing between contours,
which is a consequence of conserving the property defined in \cref{eq:transport_property}.
We attribute this improvement to the regularization done at each time step after the transport.
In the forthcoming sections, we present two regularisation methods, then show their strong and weak points.
%----------------------
\begin{figureth}
% textwidth 
{0.6}
%path 
{Chapter1/Graphics/regularization.pdf}
% caption
{Schematic of the influence of level set regularisation on the distance function at the same time frame: 
a) without any regularisation step, the isovalue contours are distorted in the wake of the rising air 
bubble while being squeezed ahead of it, 
b) in contrast to regularising the distance function, where the contours maintain their spacing and geometric 
properties with respect to the tracked interface.}
% label
\label{fig:reinit_influence}
\end{figureth}
%-----------------------------------
%
%---------------------------------------------------------------------
\subsubsection{Classic Hamilton-Jacobi reinitialisation}
%
In order to repair a distance function impaired by convective transport, \citet{sussman_level_1994} proposed 
solving a classic \emph{Hamilton-Jacobi} equation, given in its most general form:
%\citep{vigneaux_methodes_2007}:
%------
\begin{align}
\label{eq:hamilton_jacobi}
& \frac{\partial \levelset}{\partial t} + \mathbb{H} \brac{\levelset,x,t} = 0 \qquad x \in \Omega, t>0
\end{align}
%------------
where $\levelset (x,t=0) = \levelset _0$ is the initial value of the distance function. 
The term $\mathbb{H}$ is known as the \emph{Hamiltonian}. When the sign of the level set and its metric 
property ($\norm{\nabvec \levelset}=1$) are considered, \cref{eq:hamilton_jacobi} reduces to:
%------
\begin{align}
\label{eq:reinit_classic}
& \frac{\partial \levelset}{\partial t}  + S(\levelset) \brac{\norm{\nabvec \levelset} - 1} = 0
\end{align}
%------------
where $S(\levelset)$ is a step function giving the sign of the level set as follows:
%------------
\begin{align}
\label{eq:levelset_sign}
& S(\levelset) = \frac{\levelset}{\abval{\levelset}} = 
\begin{cases}
 -1 	& \text{ if } \levelset < 0 \\ 
 0	 	& \text{ if } \levelset = 0 \\ 
 +1 	& \text{ if } \levelset > 0 
\end{cases}
\end{align}
%------------
The sign function defined in \cref{eq:levelset_sign} is often smoothed to avoid numerical problems, 
as proposed for instance by \citet{sussman_level_1994}:
%------
\begin{align}
\label{eq:levelset_sign_smoothed}
& S(\levelset) = S_\varepsilon (\levelset) = \frac{\levelset}{\sqrt{\levelset^2 + \varepsilon^2}}
\end{align}
%------------
where $\varepsilon$ is a smoothing parameter that depends on the mesh size around the interface. 
However, one should be aware that within the smoothing thickness, the regularised function may suffer 
from local oscillations because of the reciprocal reinitialisation taking place at each side of the level 
set. \citet{peng_pde-based_1999} states that this problem is more likely to happen if the initial level set
shows very weak or very steep gradients, and therefore is not regular enough. The authors eventually propose
a new sign function which would reinitialise the distance function, as close as possible to the interface
without modifying the latter, as follows:
%------
\begin{align}
\label{eq:levelset_sign_smoothed2}
& S(\levelset) = S_\varepsilon (\levelset) = \frac{\levelset}{\sqrt{\levelset^2 + \norm{\nabvec \levelset}^2 \varepsilon^2}}
\end{align}
%------------
%
\subsubsection{Convective reinitialization}
%
A recent work by \citet{ville_convected_2011} introduced another concept for reinialisation 
called the \emph{convective reinitialisation}. The idea lies in combining both level set advection 
and regularisation in a single equation, saving resolution time. The key components of their method
starts by defining a pseudo time step, $\Delta \tau$, that is linked to the main time variable through
a numerical parameter $\lambda_\tau$, as follows:
%------
\begin{align}
\label{eq:pseudotimestep}
& \lambda_\tau = \frac{\partial \tau}{ \partial t}
\end{align}
%------------
The order of magnitude of $\lambda_\tau$, which can be seen as a relaxation parameter \citep[see][89]{vigneaux_methodes_2007}, 
is close to the ratio $h/\dt$. Then, the classic Hamilton-Jacobi 
\cref{eq:hamilton_jacobi} is combined into the convection step by writing:
%------
\begin{align}
\label{eq:leveller}
& \frac{\partial \levelset}{ \partial t} + \brac{\vec{v} + \lambda_\tau \vec{U}} \cdot \nabvec \levelset = \lambda_\tau S(\levelset)
\end{align}
%------------
where $\vec{U}$ is a velocity vector in the normal direction to the interface, defined by $\vec{U}= S(\levelset) \vec{n}$.
the normal vector $\vec{n}$ being previously defined in \cref{eq:levelset_normal}.
The obvious shortcoming of convective reinitialisation is that it depends on a numerical parameter $\lambda_\tau$.
Another limitation of the method is the use of a sinusoidal filter to modify the distance function by truncating
its values beyond a thickness threshold, which is also another parameter to calibrate the resolution. The drawback
of truncating the level set is the loss of information far from the interface and the inability to fully reconstruct
the distance function.
If we denote this threshold by $E$ and the modified level set by $\widetilde\levelset$ inside the thickness, 
then \cref{eq:leveller} is recast as:
%------
\begin{align}
\label{eq:leveller_s}
& \frac{\partial \levelset}{ \partial t} + \brac{\vec{v} + 
\lambda_\tau \vec{U}} \cdot \nabvec \levelset = \lambda_\tau S(\levelset)
\sqrt{1-\brac{\frac{\pi}{2 E} \widetilde\levelset}^2}
\end{align}
%------------
\Cref{eq:leveller_s} describes the transport and partial reconstruction of the distance function $\levelset$,
knowing its value $\widetilde\levelset$ inside the thickness $E$. 
%----------------------------------------
%
%-----------------------------------------
\subsubsection{Geometric reinitialization}
%
This category of methods go from the level set's basic geometric principle to construct a distance function,
instead of solving a partial differential system of equations as in the classic Hamilton-Jacobi reinitialisation.
A widely known instance of this category is the \emph{fast marching method} developed by \citet{sethian_fast_1996}
and influenced by the \citet{dijkstra_note_1959}'s method to compute the shortest path in a network of nodes. 
The method aims to solve the eikonal equation in \cref{eq:transport_property} to propagate the distance function in a 
single direction by \emph{upwinding}, i.e. going from low to high values of the distance function, 
while preserving a unitary distance gradient.

\emph{Direct reinitialisation} is another interesting method in the geometric reinitialise category. However,
it has not gained noticeable attention in the literature given the terribly cost in terms of computation time and efficiency.
The main idea is very simple: reconstruct the distance function over $\Omega$ ar a subset of $\Omega$, by computing the 
minimum distance between each mesh node and the interface. It means that, for any point $\vec{x} \in \Omega$, the following 
constraint should be satisfied \citet{osher_signed_2003}:
%------
\begin{align}
\label{eq:min_distance}
& d_\Gamma(\vec{x}) = \min \norm{\vec{x}-\vec{x_\Gamma}} \qquad \forall \vec{x_\Gamma} \in \partial \Omega = \Gamma,
\end{align}
%------------
A efficient implementation of this method is done by \citet{shakoor_efficient_2015} making use of \emph{k-d} trees
to limit the search operations of elements and the subsequent distance evaluations in each of these elements.
Moreover, the authors give a comparison of the previously stated methods on 2D and 3D cases, showing the great 
performance of direction reinitialisation when used with k-d trees algorithm, hence we use it in the present work.
%
%--------------------------------------
\section*{To do ?}
\comment{Interface Remeshing: Importance when using a static level set and more importantly when LS is transported,
influence of mixing area \emph{thickness} and \emph{resolution} (i.e. nb of nodes with the area),
Isotropic or anisotropic ? the first is more important to composition calculation while the second
is more relevant if we mean do thermohydraulics without macrosegregation}

















