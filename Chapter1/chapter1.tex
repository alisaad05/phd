\chapter{Modelling Review}
\begin{nolinkcolors} 
\minitoc
\end{nolinkcolors}
\newpage

%
\section{Modelling macrosegreation}
%-------------
%
%
\subsection{Dendritic growth}
%-------
In a casting process, the chill surface i.e. the contact between the molten alloy and relatively cold moulds, is the first area to solidify. 
Thermal gradient, $G$, and cooling rate $R$ are two crucial process parameters that define the interface speed $\vstar$, which in turn,
affects the initial microstructure. Although it may be not easy to control them, it remains important to understand their implication in solidification.

The solid-liquid interface fluctuates when solidifying, thus perturbations may appear on the front, locally destabilizing it. 
Two outcome scenarios are possible.
The first scenario is characterized by low values of $\vstar$ where the interface maintains a planar shape, hence we speak of \emph{planar growth}. 
With this kind of growth, a random protuberance appearing at the interface, has a low tip velocity (low driving force of solidification). As such,
the rest of the interface catches up, keeping the planar geometry.
In another scenario, where a real casting is considered, the interface speed is greater in general, due to high solidification rate.
The protuberance tip will be pulled into a liquid less rich in solute than the interface. The zone ahead of the solid-liquid interface is constitutionally undercooled \citep{tiller_redistribution_1953}, giving a greater driving force for the protuberance to grow in the direction
of the thermal gradient. As it has a tree-like shape, we speak of \emph{dendritic growth}. Near the chill surface, dendrites are columnar, with a 
favourable growth in the <100> direction for alloys with cubic lattices, but different orientations are also reported in the literature \citep[see][289]{dantzig_solidification_2009}.
If temperature is uniform, which the case usually far from mould walls, a similar dendritic growth phenomenon occurs, but with an equiaxed morphology.

Columnar dendrites are characterized by a primary spacing, $\pdas$, between the main trunks, and a secondary spacing, $\sdas$, for the arms that are perpendicular
to the trunks. It should be noted that $\sdas$, together with the grain size, are two important microstructural parameters in the as-cast microstructure \citep{easton_grain_2011}.
Further branching may occur but will not be discussed here.
%----------------------
\begin{figureth}
% textwidth 
{0.5}
%path 
{Chapter1/Graphics/dendritic3.png}
% caption
{Schematic of a) a protuberance growing on the solid-liquid interface with b) the corresponding composition profiles 
(Reproduced and adapted from \citet{doitpoms_dissemination_2000}, \doitpoms)}
% label
\label{fig:dendritic_growth}
\end{figureth}
%-----------------------------------
%
%
\subsection{Mush permeability}
The dendritic geometry is crucial in solidification theory as it exhibits lower solid fraction compared to a microstructure formed by planar growth.
This fact has consequences in the fluid-structure interaction in the mushy zone, namely the liquid flow through dendrites. At the chill surface,
the solid  grows gradually from dispersed growing nuclei, to a permeable solid skeleton until finally grains have fully grown with the end of phase change.
The intermediate state where liquid can flow in and out of the mushy zone through the dendrites is a key phenomenon from a rheological perspective.
%By defining a \emph{coherency temperature} $\Tcoh$, one can distinguish two behaviors. 
%For temperatures below $\Tcoh$ (usually at low solid fractions), the liquid flow is characterized by an intrinsic velocity. 
%As for temperature above $\Tcoh$,
%For the rest of this section, we will focus on the first behavior where liquid has a relative freedom to move inside the mushy zone.
The flow through the solid skeleton is damped by primary and secondary dendrites, resulting in momentum dissipation just like in saturated porous media. 
The famous \citet{darcy_les_1856} law relates the pressure gradient ($\nabvec p$) to the fluid velocity $\vec{v}$, through the following equation \citep{rappaz_numerical_2003}:
%----------------------
\begin{align}
\label{eq:darcy}
& \vec{v} = \frac{\K}{\mu} \nabvec p
\end{align}
%----------------------
where $\mu$ is the liquid dynamic viscosity and $\K$ is the permeability tensor. The latter parameter has been the subject of numerous studies that aimed
to predict it from various microstructural or morphological parameters.
Some of these studies has started even before the first attempts to model macrosegregation by \citet{flemings_macrosegregation:_1967, flemings_macrosegregation:_1968-1,flemings_macrosegregation:_1968}. Basically, all models include the solid fraction, $\gs$, as input 
to predict mush permeability along with empirical data. An instance of such models is the work of \citet{xu_gravity-_1991}.
Some models rely additionally on the primary dendrite arm spacing $\pdas$ like Blake-Kozeny \citep{ramirez_evaluation_2003}, or the secondary 
dendrite arm spacing $\sdas$ like Carman-Kozeny, as a meaningful parameter to determine an isotropic permeability. 
Other models like \citet{poirier_permeability_1987,felicelli_simulation_1991} derive an anisotropic permeability based on both $\pdas$ and $\sdas$.

The present work uses Carman-Kozeny as a constitutive model for the permeability scalar (zero order tensor):
%----------------------
\begin{align}
\label{eq:permeability}
& \K  = \frac{\sdas^2 \gl^3}{180\brac{1-\gl}^2}
\end{align}
%-----------------------------------
%
%
\subsection{Microsegregation}
Microsegregation is a fundamental phenomenon in solidification. The simplest definition would be 
an uneven distribution of solute between liquid and the herein growing solid, at the microscopic scale
of the interface separating these phases. If we consider a binary alloy, then the solubility limit is 
the key factor that dictates the composition at which a primary solid phase exists in equilibrium. 
The segregation (or partition) coefficient $\k$ determines the extent of solute rejection into the liquid during solidification:
%----------------------
\begin{align}
\label{eq:partition}
& \k = \frac{\Csstar}{\Clstar}
\end{align}
%----------------------
where $\Csstar$ and $\Clstar$ are the compositions of the solid and liquid respectively, at the interface. When the 
segregation coefficient is less than unity (such is the case for most alloys during dendritic solidification), 
the first solid forms with a composition $\k \Clstar=\k \Cnominal$ less than the liquid's 
composition $\Cnominal$, the latter being initially at the nominal composition, $\Cnominal$. \Cref{fig:binary_diag} illustrates a typical binary 
phase diagram where the real solidus and liquidus are represented by solid lines, while the corresponding linear approximations are in dashed lines.
For most binary alloys, this linearisation simplifies derivation of microsegregation models, as $\k$ becomes independent of temperature.

For each phase, the relationship between the composition at the interface and that in the bulk depends on the chemical homogenisation ability of the phase.
The more homogeneous a phase, the closer the concentrations between the interface and the bulk, hence closer to equilibrium.
%The higher the diffusion coefficient of the phase, $\Dphi$, the closer the concentrations between the interface and the bulk. 
It is thus essential to study the effect of homogenisation on the segregation behaviour and the subsequent effect on solidification, which leads the formalism of microsegregation models.
%----------------------
\begin{figureth}
% textwidth 
{0.25}
%path 
{Misc/dummy.pdf}
% caption
{Simplified binary phase diagram}
% label
\label{fig:binary_diag}
\end{figureth}
%-----------------------------------
%
%
\subsubsection{Microsegregation models}
Solid formation depends greatly on the ability of chemicals species to diffuse within each of the solid and liquid phases, but also across the
solid-liquid interface. Furthermore, chemical diffusion like all other diffusional process, is a time-dependent phenomenon. One can thus conclude that two factors
influence the amount of solid formation: cooling rate and diffusion coefficients. However, convection and other mechanical mixing sources, 
homogenise the composition much faster than atomic diffusion. As such, \emph{complete mixing} in the liquid is always an acceptable assumption, regardless of the 
solidification time. We may speak of infinite diffusion in the liquid. Nevertheless, diffusion in the solid also known as \emph{back diffusion}, is the only transport mechanism with very low diffusion coefficients. Therefore, chemical species require a long time, i.e. low cooling rate, to completely diffuse within the solid.
The difference in diffusional behaviour is summarized by two limiting segregation models of perfect equilibrium and nonequilibrium at the scale
of a secondary dendrite arm, which are the lever rule and Gulliver-Scheil models, respectively. Afterwards, models with finite back diffusion are presented. 
%------------
%
%
\subsubsection*{Lever rule}
The lever rule considers an ideal equilibrium in all phases, i.e. solidification is extremely slow, hence phase compositions are 
homogeneous ($ \Clstar = \Cl$ and $ \Csstar = \Cs$ ) at all times as a consequence of complete mixing. 
These compositions are given by:
%----------------------
\begin{align}
\label{eq:leverrule}
& \Cl= \Clstar = \k \Csstar = \k \Cs \\
& \Cs= \Csstar = \frac{\Cnominal}{\k(1-\fs) + \fs}
\end{align}
%----------------------
At the end of solidification, the composition of the solid phase is equal to the nominal composition, $\Cs = \Cnominal$
%------------
%
%
\subsubsection*{Gulliver-Scheil}
The other limiting case is the absence of diffusion in the solid. That includes also the diffusion at the interface, so nothing diffuses in or out. The consequence is a steady increase of the homogeneous liquid composition while the solid composition remains non-uniform.
Compared to a full equilibrium approach, higher fractions of liquid
will remain until eutectic composition is reached, triggering a eutectic solidification. The phase compositions are given by:
%----------------------
\begin{align}
\label{eq:scheil}
& \Cl= \Clstar = \k \Csstar \\
& \Cs= \k \Cnominal (1-\fs)^{1-\k}
\end{align}
%------------
%
%
\subsubsection*{Finite back diffusion}
It has been concluded that the assumption of a negligeable back diffusion overestimates the liquid composition
and the resulting eutectic fraction. Therefore, many models studied the limited diffusion in the solid. One of the earliest models is the Brody-Flemings models \citep{khan_influence_2014} that is a based on a differential solute balance equation for a parabolic growth rate, as follows:
%----------------------
\begin{align}
\label{eq:brodyflemings}
& \Cl= \Clstar = \k \Csstar \\
& \Cs= \k \Cnominal \crochet{ 1 - \brac{ 1 - 2 \Fos \k } \fs }^{\frac{\k-1}{1 - 2 \Fos \k }}
\end{align}
%------------
where $\Fos$ is the dimensionless \emph{Fourier number} for diffusion in the solid \citep{dantzig_solidification_2009}. It depends on the 
solid diffusion coefficient $\Ds$, solidification time $\tsolidif$ and the secondary dendrite arm spacing, as follows: 
%----------------------
\begin{align}
\label{eq:fouriersolid}
& \Fos = \frac{\Ds \tsolidif}{\brac{\sdas / 2}^2}
\end{align}
%------------
Several other models were since suggested and used. The interested reader is referred to the following non exhaustive list of publications: \citet{clyne_solute_1981,kobayashi_solute_1988,ni_volume-averaged_1991,wang_multiphase_1993,combeau_modeling_1996,martorano_solutal_2003,tourret_generalized_2009}.
%------------------------------------
%
%
\subsection{Macroscopic solidification model: monodomain}
In this section, we will present the macroscopic conservations equations that enable us to predict 
macrosegregation in the metal when the latter is the only domain in the system.
%-------------
%
%
\subsubsection{Volume averaging} \label{sec:volumeavg}
It is crucial for a solidification model to represent phenomena on the microscale, then scale up to predict 
macrscopic phenomena. Nevertheless, the characteristic length of a small scale in solidification may represent a dendrite arm spacing, for instance the mushy zone permeability, as it may also represent an atomic distance if one is interested, for instance, in the growth competition between diffusion and surface energy of the solid-liquid interface. Modelling infinitely small-scale phenomena could be prohibitively expensive in computation time, if the we target industrial scales. 

The volume averaging is a technique that allows bypassing this barrier by averaging
small-scale variations on a so-called \emph{representative volume element} (RVE) \citep{dantzig_solidification_2009} with the 
following dimensional constraints on its volume, \rev:
the element should be large enough to "see" and average microscopic fluctuations whilst being smaller than the scale of macroscopic variations.
Solid and liquid may exist simultaneously in the RVE, but no gas phase is considered (volume saturation: $\Vs+\Vl=\rev$). 
Moreover, temperature is assumed uniform and equal for all the phases.
The formalism, introduced by \citet{ni_volume-averaged_1991}, is summarized by the following equations for any physical quantity $\psi$:
%----------------------
\begin{align}
\label{eq:volumeaveraging1}
& \avg{\psi} = \frac{1}{\rev} \integral{\rev}{\psi}{\Omega} = \avg{\psi^s} + \avg{\psi^l}
\end{align}
%------------
where $\psis$ and $\psil$ are phase averages of $\psi$. Then, for any phase $\phi$, one can introduce the \emph{phase intrinsic average} of $\psi$, denoted $\avg{\psi}^\phi$, by writing:
%----------------------
\begin{align}
\label{eq:volumeaveraging2}
& \avg{\psi^\phi} = \frac{1}{\rev} \integral{\Vphi}{\psi}{\Omega} = \gphi \avg{\psi}^\phi
\end{align}
%------------
where $\gphi$ is the volume fraction of the phase. To finalize, the averaging is applied to temporal and spatial derivation operators \citep{rivaux_simulation_2011}:
\begin{align}
\label{eq:volumeaveraging3}
& \avg{ \frac{\partial \psi}{\partial t} ^\phi } = \frac{\partial \avg{\psi^\phi}}{\partial t} - \integral{\gammastar}{\psi^\phi \vstar \cdot \vec{n^\phi}}{\Gamma} \\
\label{eq:volumeaveraging4}
& \avg{ \nabvec \psi^\phi } = \nabvec \avg{\psi^\phi} + \integral{\gammastar}{\psi^\phi \vec{n^\phi}}{\Gamma}
\end{align}
%------------
where $\vstar$ is the local relative interface velocity and $\gammastar$ is the solid-liquid interface, 
while $n^\phi$ is the normal to $\gammastar$, directed outwards. The surface integral term in 
\cref{eq:volumeaveraging3,eq:volumeaveraging4} is an \emph{interfacial average} 
that expresses interfacial exchanges between the phases. The previous equations will 
be used to derive a set macroscopic conservation equations. 
It is noted that the intrinsic average $\avg{\psi}^\phi$ may be replaced by ${\psi}^\phi$ 
for notation simplicity, whenever the averaging technique applies.
%
%
%----------------------------------
\subsubsection{Macroscopic equations}
%-----------
A monodomain macroscopic model relies on four main conservation equations to predict 
macrosegregation in a single alloy domain, i.e. the latter is considered without 
any interaction with another alloy or ambient air. The general form of a conservation 
equation of any physical quantity $\psi$ is given by \citep{rappaz_numerical_2003}:
%-------
\begin{align}
\label{eq:conservationequation}
& \frac{\partial \psi}{\partial t} + \nabla \cdot \brac{\psi \vec{v}} - \nabla \cdot \vec{j_{\psi}} = Q_{\psi}
\end{align}
%-------
The first LHS term in \cref{eq:conservationequation} represents the time variation of $\psi$, the second term accounts for transport by advection while the third is the diffusive transport and the RHS term represents a volume source.
The considered equations are mass, energy, liquid momentum and species conservation, all summarized in \cref{tab:conservationeqs}. The solid momentum is not considered as we assume a fixed and rigid solid phase ($\vs=\vec{0}$).
%-----
\begin{tableth}
\centering

\caption{Summary of conservation equations with their variables}
{\tabulinesep=1.5mm
\begin{tabu}{ p{5cm} p{1cm} p{1cm} p{1cm}}
\tabucline[1pt]{-}
Conservation Equation & $\psi$ & $\vec{j_{\psi}}$ & $Q_{\psi}$ \\\tabucline[1pt]{-}
%-----------------------------
Mass				& 	$\avg{\rho}$  		& $-$ 						& $-$		\\
Energy				& 	$\avg{\rho h}$  	& $\avg{\vec{q}}$ 			& $-$		\\
Species				& 	$\avg{\rho w_i}$  	& $\avg{\vec{j_i}}$ 		& $-$		\\
Liquid momentum		&	$\avg{\rho \vl}$  	& $-\avg{\sigmal}$ 			& $\Fv$		\\\tabucline[1pt]{-} %\vec{\Gamma}^{l}
%-----------------------------
\end{tabu}}
\label{tab:conservationeqs}
\end{tableth}
%-----
We develop the ingredients of these equations using the averaging technique, as follows:
%-------
\begin{align}
\label{eq:mainingredients1}
& \avg{\rho} = \gl \rhol + \gs \rhos \\
\label{eq:mainingredients2}
& \avg{\rho \vec{v}} = \gl \rhol \vl + \cancel{\gs \rhos \vs} \\
\label{eq:mainingredients3}
& \avg{\rho h} = \gl \rhol \hl + \gs \rhos \hs \\
\label{eq:mainingredients4}
& \avg{\rho h \vec{v}} = \gl \rhol \hl \vl + \cancel{\gs \rhos \hs \vs} \\
\label{eq:mainingredients5}
& \avg{\rho w_i} = \gl \rhol \wil + \gs \rhos \wis \\
\label{eq:mainingredients6}
& \avg{\rho w_i \vec{v}} = \gl \rhol \wil \vl + \cancel{\gs \rhos \wis \vs}
\end{align}
%-------
% $\mat{\sigma}$ 
%-------
Next we define the average diffusive fluxes, $\vec{q}$ for energy and $\vec{j_i}$ for solutes, using Fourier's conduction law and Fick's first law respectively:
%------
\begin{align}
\label{eq:fourierlaw}
& \avg{\vec{q}} = - \gl \kl \nabvec T -  \gs \ks \nabvec T 	\\
\label{eq:ficklaw}
& \avg{\vec{j_i}} = - \gl \Dl \nabvec \wil - \cancel{\gs \Ds \nabvec \wis}
\end{align}
%------------
In \cref{eq:ficklaw}, the solid diffusion coefficient is neglected, 
by considering that for macroscopic scales, the average composition of the alloy is much more influenced by advective 
and diffusive transport in the liquid.
In \cref{eq:fourierlaw}, we assumed that phases are are thermal equilibrium, that is, temperature is uniform in the RVE.

Now that the main conservation equations ingredients are properly defined, we may write each averaged conservation
equations as the sum of two local conservation equations for each phase in the RVE, hence introducing also interfacial average terms.
For instance, the local mass balance in each phase is given by:
%----
\begin{subequations}
\begin{align}
\label{eq:mass_liquid}
& \temp{\gl \rhol} + \nabla \cdot \brac{\gl \rhol \vl} = S_V \langle \rhol \vlstar \cdot \vec{n} \rangle^* - S_V \langle \rhol \vstar \cdot \vec{n} \rangle^* \\
\label{eq:mass_solid}
& \temp{\gs \rhos} + \nabla \cdot \brac{\gs \rhos \vs} = - S_V \langle \rhos \vsstar \cdot \vec{n} \rangle^* + S_V \langle \rhos \vstar \cdot \vec{n} \rangle^*
\end{align}
\end{subequations}
%----
where $S_V= A_{sl}/\rev$ is the specific surface area, $\vec{v}^{l^*}$ and $\vec{v}^{s^*}$ are respectively, the liquid 
and solid phase velocity at the interface and $\vec{v}^*$ is the previously introduced solid-liquid interface velocity. 
For instance, the first interfacial exchange term in the RHS is expanded as follows \citep{dantzig_solidification_2009}:
%----
\begin{subequations}
\begin{align}
S_V \langle \rhol \vlstar \cdot \vec{n} \rangle^*
  &= \frac{A_{sl}}{\rev} \brac{ \frac{1}{A_{sl}} \int_{A_{sl}} \rhol \vlstar \cdot \vec{n} dA} \\
&=\frac{1}{\rev} \int_{A_{sl}} \rhol \vlstar \cdot \vec{n} dA
\end{align}
\end{subequations}
%----
Summing equations \eqref{eq:mass_liquid} and \eqref{eq:mass_solid}, results in the overall mass balance
in the RVE:
%----
\begin{align}
\label{eq:conservation_mass_local}
& \temp{\gl \rhol + \gs \rhos}  +  \nabla \cdot \brac{\gl \rhol \vl + \gs \rhos \vs}
   =  S_V \avg{\rhol \brac{\vlstar-\vstar} \cdot \vec{n}}^*  
   -  S_V \avg{\rhos \brac{\vsstar-\vstar} \cdot \vec{n}}^*
\end{align}
%----
where the RHS cancels to zero as shown by \citet{ni_volume-averaged_1991}. Moreover, the authors show that with their averaging technique, 
interfacial exchanges for energy, chemical species and momentum cancel out as they are equal in absolute value but opposite in sign. 
Using \cref{eq:mainingredients1,eq:mainingredients2,eq:mainingredients3,eq:mainingredients4,eq:mainingredients5,eq:mainingredients6,eq:fourierlaw,eq:ficklaw}
and following the same procedure done in \cref{eq:conservation_mass_local}, the averaged mass balance hence writes:
%------
\begin{align}
\label{eq:conservation_mass}
& \frac{\partial \avg{\rho}}{\partial t} + \nabla \cdot \avg{\rho \vec{v}} = 0
\end{align}
%------------
whereas the averaged energy balance writes:
%------
\begin{align}
\label{eq:conservation_energy}
& \frac{\partial \avg{\rho h}}{\partial t} + \nabla \cdot \avg{\rho h \vec{v}} + \nabla \cdot \brac{\avg{\kappa} \nabvec T} = 0
\end{align}
%------------
and finally the species balance writes:
%------
\begin{align}
\label{eq:conservation_solute}
& \frac{\partial \avg{\rho w_i}}{\partial t} + \nabla \cdot \avg{\rho w_i \vec{v}} + \nabla \cdot \brac{\gl \Dl \nabvec \wil}= 0
\end{align}
%------------
As stated previously, the momentum balance in the solid phase is not taken into consideration, hence no need to sum the corresponding 
local conservation equations. The liquid momentum balance writes:
%------
\begin{align}
\label{eq:conservation_momentumliq}
& \temp{ \rhol \gl \vl } + \nabvec \cdot \brac{\rhol \gl \vl \times \vl} = 
	\nabvec \cdot \brac{\gl \sigmal} +\gl \Fv + \vec{\Gamma}^{l}
\end{align}
%------------
The interfacial momentum transfer between the solid and liquid phases is modelled by a momentum flux vector $\vec{\Gamma}^{l}$, consisting of
hydrostatic and deviatoric parts, such that:
%------------
%\begin{subequations}
\begin{align}
\label{eq:eq_interfacial_momentum_init}
		& \vec{\Gamma}^{l} =  \vec{\Gamma}_{p}^{l} + \vec{\Gamma}_{\mathbb{S}}^{l}						\\
		& \vec{\Gamma}_{p}^{l} = p^{l^{*}} \nabvec g^{l} = p^{l} \nabvec g^{l}			  \\
		& \vec{\Gamma}_{\mathbb{S}}^{l} = -{g^{l}}^{2} \mu^{l} \mathbb{K}^{-1} \brac{ \vl - \cancel{\vs} }  
\end{align}
%\end{subequations}
%------------
where $p^{l^{*}}$ is the pressure at the interface, considered to be equal to the liquid hydrostatic 
pressure $\pl$, $\mathbb{K}$ is a permeability scalar (isotropic) computed using \cref{eq:darcy} and $\mu^l$ is the liquid's 
dynamic viscosity. 
The general form of the Cauchy liquid stress tensor in \cref{eq:conservation_momentumliq} is decomposed as follows: 
%------------
%\begin{subequations} 
\begin{align}
\label{eq:stress_liq}
& \avg{\sigmal} = \gl \sigmal = - \brac{\avg{\pl} -\lambda \nabla \cdot \vit} \I + \avg{\mat{\mathbb{S}}^{l}}
\end{align}
%\end{subequations}
%------------
where $\lambda$ is a dilatational viscosity \citep{dantzig_solidification_2009} and $\mat{\mathbb{S}}^{l}$ is the
liquid strain deviator tensor. In the literature, the coefficient $\lambda$ is taken proportional to the 
viscosity: $\lambda = \frac{2}{3} \mu^l $. However, as we consider an incompressible flow, the divergence 
term vanishes, thus rewriting \cref{eq:stress_liq} as follows:
%------------
\begin{subequations} 
\begin{align}
\label{eq:stress_liq_incompressible1}
& \avg{\sigmal} = - \avg{\pl} \I + \avg{\mat{\mathbb{S}}^{l}} \\
\label{eq:stress_liq_incompressible2}
& \avg{\sigmal} = - \avg{\pl} \I + 2 \mul \strainrate 
\end{align}
\end{subequations}
%------------
where the transition from \cref{eq:stress_liq_incompressible1} to \cref{eq:stress_liq_incompressible2} is made
possible by assuming a Newtonian behaviour for the liquid phase. The strain rate tensor, $\strainrate$, depends on 
the average liquid velocity:
%------------
\begin{align}
\label{eq:tensor_strainrate}
\strainrate = \nabmat \vit  +  \nabmattransp \vit 
\end{align}

%%-------------------------------------------------------------------------------------------
\subsection{garbage}
Macro models:
\begin{itemize}
\item Rivaux ?
\item Gu beckermann 1999 ?
\end{itemize}
%%----
MICRO MACRO: 
\begin{itemize}
\item Tommy Carozzani (direct)
\item P. Thévoz, J.-L. Desbiolles, M. Rappaz, Metallurgical and Materials TransactionsA 20 (2) (1989) 311–322
\item guo  beckermann 2003
\item Combeau 2009
\item Miha Zaloznik 2010 (indirect)
\end{itemize}
end by talking about taking air into account and the need for an interface capturing method
%
%
%----------------------------------
\section{Eulerian and Lagrangian motion description}
%
%----------------------------------
\subsection{Overview}
%
In mechanics, it is possible to describe motion using two well-known motion description: Eulerian and Lagrangian descrptions.
To start with the latter, it describes the motion of a particle by attributed a reference frame that moves with the particle.
In other words, the particle itself is the center of a reference frame moving at the same speed during time. The particle
position is hence updated as follows:
%------
\begin{align}
\label{eq:lagrangian}
& x^{(t)} = x^{(t+1)} + \vec{v} \Delta t
\end{align}
%------------
As such, the total variation of any physical quantity $\psi$ related to the particle 
can be found by deriving with respect to time, $\frac{\mathrm{d} \psi}{\mathrm{d} t}$.
%
In contrast to the Lagrangian description, the Eulerian description considers a 
fixed reference frame and independent of the particle's trajectory The total variation of $\psi$
cannot be simply described by a temporal derivative, since the particle's velocity is not known to 
the reference frame, and thus the velocity effect, namely the advective transport of $\psi$, should also be considered as follows:
%------
\begin{align}
\label{eq:eulerian}
& \frac{\mathrm{d} \psi}{\mathrm{d} t} = \frac{\partial \psi}{\partial t} + \underbrace{\vec{v} \cdot \nabvec \psi}_{\substack{\text{Advective} \\ \text{Transport}}}  
\end{align}
%------------
In this case, the LHS term is also known as \emph{total} or \emph{material derivative}.
The importance of these motion descriptions is essential to solve mechanics, whether for fluids or solids, using a numerical method like the finite 
element method (FEM). One of the main steps of this method is to spatially discretise a continuum into a grid of points (nodes, vertices ...), where any 
physical field shall be accordingly discretized. Now, if we focus on a node where velocity has a non zero value and following the previously made analysis,
two outcomes are possible: either the node would be fixed (Eulerian) or it would move by a distance proportional to the prescribed velocity (Lagrangian).
As a consequence, points located on the boundaries constantly require an update of the imposed boundary conditions.

From these explanations, one can deduce that an Eulerian framework is suited for fluid mechanics problems where 
velocities are high and may distort the mesh points, whereas the Lagrangian framework is better suited for solid mechanics 
problems where deformation velocities are relatively low and should well behave when predicting strains.

Another motion description has emerged some decades ago, \citet{hirt_arbitrary_1971} call it the Arbitrary Langrangian-Eulerian (ALE) method. 
ALE combines advantages from both previous descriptions as it dictates a Lagrangian behavior at "solid" nodes where solid is deforming, and
an Eulerian behavior at "fluid" nodes.
%
%----------------------------------
\subsection{Interface capturing}
%
As no solid deformation is considered in this work, the Eulerian framework is a convenient choice. Although solidification shrinkage is 
to be considered in the current scope, it will deform the alloy's outter surface in contact with the air.
We intend to track this interface and its motion over time via a numerical method. A wide variety of methods accomplish this 
task while they yield different advantages and disadvantages. Such methods fall into two main classes, either interface tracking
or interface capturing, among which we cite: marker-and-cell (MAC) \citep{harlow_numerical_1965}, volume of fluid (VOF) \citep{hirt_volume_1981}, 
phase field methods (PF), level set method (LSM) \citep{osher_fronts_1988}, coupled level set - VOF method and others. 
The interested reader may refer to quick references by \citet{prosperetti_navier-stokes_2002,maitre_review_2006} about these methods.
In the past years, the level set method received a considerable attention in many computational fields, specifically in solidification.
For this reason, we will focus on this method henceforth, giving a brief literature review and technical details in the next sections.
%
%----------------------------------
\section{Solidification models with level set}
In classic solidification problems, the need to track an interface occurs usually at the solid-liquid interface, that is why the phase field
method \citep{karma_phase-field_1996,boettinger_phase-field_2002} and the level set method \citep{chen_simple_1997,gibou_level_2003,tan_level_2007} 
were applied at a microscale to follow mainly the dendritic growth of a single crystal in an undercooled melt.
In our case however, when we mention "solidification models using LSM", we 
do not mean the solid-liquid interface inside the alloy, but it is the alloy(liquid)-air interface that is tracked, assuming that microscale
phenomena between the phases within the alloy, are averaged using the previously defined technique in \cref{sec:volumeavg}.

Very few models were found in the literature, combining solidification and level set as stated previously. 
\citet{du_simulating_2001} applied it to track the interface between two motlen alloys in a double casting technique. In welding research, on another hand, has been more
active adapting the level set methodology to corresponding applications. In Cemef, two projects were already based on the metal-air level set
in welding simulations and showed promising results. Firstly, \citet{desmaison_level_2014} ... \red{TODO}\\ %TODO
Later, \citet{chen_three_2014} applied to gas metal arc welding (GMAW) to predict 
the grain structure in the heat affected zone essentially.
More recently, in a related context, \citet{courtois_complete_2014} used the same methodology but this time to predict keyhole defect formation
in spot laser welding. The tracked interface in this case was that between the molten alloy and the corresponding vapor phase.
%----------------------------------
\section{The level set method}
Firstly introduced by \citet{osher_fronts_1988}, this method became very popular in studying multiphase flows.
It is reminded that the term \emph{multiphase} in computational domains usually refers to multiple fluids, and thus
should not be mixed with definition of a phase in the current solidification context. For disambiguation, we shall
use \emph{multifluid flow} when needed.
The great advantage lies in the way the interface between two fluids, $F_1$ and $F_2$ is implicitly captured, unlike 
other methods where the exact interface position is needed. In a discrete domain, the concept is to assign for each mesh node of position $x$, 
the minimum distance $d_{\Gamma} \brac{x}$ separating it from an interface $\Gamma$. 
The distance function, denoted $\alpha$ and defined in \cref{eq:levelset_defintion}, is then positive or negative based on the fluid or domain 
to which the node belongs.
%------------
\begin{align}
\label{eq:levelset_defintion}
\levelset = 
\begin{cases}
 d_{\Gamma} \brac{x} 		& \text{ if } x \in F_1 \\ 
 -d_{\Gamma} \brac{x}	 	& \text{ if } x \in F_2 \\ 
 0 							& \text{ if } x \in \Gamma_{F1, F2} 
\end{cases}
\end{align}
%------------
%----------------------
\begin{figureth}
% textwidth 
{0.5}
%path 
{Misc/dummy.pdf}
% caption
{Schematic of a distance function}
% label
\label{fig:distance_function}
\end{figureth}
%-----------------------------------
%
\subsection{Diffuse interface}
The level set has many attractive properties that allows seamless implementation in 2D and 3D models. It is a continuously differentiable
$C^1$-function. For instance, a \emph{Heaviside} function, denoted $\heaviside$, can be obtained by first order derivation of the level set function. The 
Heaviside function is continuous but non differentiable, with an abrupt transition from 0 to 1 across the sharp interface, as follows:
%----------
\begin{align}
\label{eq:no_smoothing}
\heaviside = 
\begin{cases}
	0  & \text{ if } \levelset < 0 \\
    1  & \text{ if } \levelset \geqslant 0 \\  
\end{cases}
\end{align}
%----------
The Heaviside function is crucial in the level set methodology as it is useful to define the geometric 
"presence" of a domain with respect to the interface. As such, material properties 
depend upon this function, which will be discussed later in section \red{TODO}\\ %TODO
It is established that a steep transition, as shown in \cref{fig:heaviside}, can lead to numerical problems, 
so the Heaviside function should be smoothed in a fixed thickness.
Sinusoidal smoothing in \cref{eq:sinusoidal_smoothing} is widely used with level set formulations.
%------------
\begin{align}
\label{eq:sinusoidal_smoothing}
\heaviside= 
\begin{cases}
	0  & \text{ if } \levelset < -\epsilon \\
    1  & \text{ if } \levelset > \epsilon \\  
    \frac{1}{2} \brac{1+ \frac{\levelset}{\epsilon} + \frac{1}{\epsilon}\sin \brac{\frac{\pi \levelset}{\epsilon }} } & \text{ if } - \epsilon \leq \levelset \leq \epsilon
\end{cases}
\end{align}
%-----------
where the interval $\crochet{-\epsilon; \epsilon}$ is an artificial interface thickness around the zero distance.
Defining a diffuse interface rather than a sharp one, is also a common approach in phase field methods \citep{beckermann_modeling_1999,sun_diffuse_2004}.
It is emphasized that the latter methods give physically meaningful analysis of a diffuse interface and the optimal thickness by thoroughly studying the 
intricate phenomena happening at a microscopic scale. However, for level set methods, there has not been a formal 
work leading the same type of analysis. For this reason, many aspects of the level set method lack physical meanings but still computationally useful.
In a recent paper by \citet{gada_derivation_2009}, the authors respond partially to this problem by analysing and deriving conservation equations
using a level set in a more meaningful way, but do not discuss the diffuse interface aspect. Finally, within the diffuse interface, fluids properties 
may vary linearly or not, depending on the mixing law, which is presented in the next section.
%----------------------
\begin{figureth}
% textwidth 
{0.5}
%path 
{Misc/dummy.pdf}
% caption
{(a) Stepwise Heaviside (b) Smoothed Heaviside}
% label
\label{fig:heaviside}
\end{figureth}
%-----------------------------------
%
\subsection{Mixing Laws}
A \emph{monolithic} resolution style, as opposed to a \emph{partitioned} resolution, is based on solving a single set of equations
for both fluids separated by an interface, as if a single fluid were considered. 
Level set is one many methods that use the monolithic style to derive a single set of conservation equations
for both fluids. The switch from one material to the other is implicitly taken care of by using the Heaviside function as well mixing laws. These laws
are crucial to define how properties vary across the diffuse interface in view of a more accurate resolution.
%----------------------
\begin{figureth}
% textwidth 
{0.75}
%path
{Chapter1/Graphics/mixing_laws.pdf}
% caption
{Three mixing laws, arithmetic, logarithmic and harmonic commonly used in monolithic formulations}
% label
\label{fig:mixinglaws}
\end{figureth}
%-----------------------------------
The most frequently used mixing law in the literature is the arithmetic law. Other transitions are less known such as 
the harmonic and logarithmic mixing. The first law is maybe the most intuitive and most used for properties mixture as it emanates
from VOF-based methods. 
If we consider any property $\psi$, for instance the fluid's dynamic viscosity $\mu$, then the arithmetic 
law will give a mixed property $\mix{\psi}$ as follows:
%---
\begin{align}
\label{eq:arithmetic}
\mix{\psi} = \heaviside^{F_1} \psi^{F_1} + \heaviside^{F_2} \psi^{F_2}
\end{align}
%---
Basically, the result is an average property that follows the same trend as the Heaviside function. As for the harmonic law, it writes:
%---
\begin{align}
\label{eq:harmonic}
\mix{\psi} = \brac{\frac{\heaviside^{F_1}}{\psi^{F_1}} + \frac{\heaviside^{F_2}}{\psi^{F_2}} }^{-1}
\end{align}
%---
and last, the logarithmic law writes:
%---
\begin{align}
\label{eq:logarithmic}
\mix{\psi} =  n^{\brac{\heaviside^{F_1} \log_n \psi^{F_1} + \heaviside^{F_2} \log_n \psi^{F_2}}}
\end{align}
%---
where $n$ is any real number serving as a logarithm base, which often is either the exponential $e$ or $10$.
The mixture result with this law is the same, regardless of the value of $n$.
By looking to \cref{fig:mixinglaws}, we clearly see that the difference between all three approaches is the property weight given to each side 
of the level set in the mixture. The arithmetic law, being symmetric, has equal weights, $\psi^{F_1}$ and $\psi^{F_2}$, in the final mixture. 
Nevertheless, the asymmetric harmonic mixing varies inside the diffuse interface with a dominant weight of one property 
over the other. As for the logarithmic mixture, it can be seen as an intermediate transition between the preceding laws.

As long as the interface thickness is small enough, the choice of a mixing law should not drastically change the result, 
inasmuch as it depends on the discretisation resolution of the interface. This fact made the arithmetic mixing
the most applied one, because it is symmetric and easy to implement (no handling of potential division problems like harmonic laws for instance). 
However, \citet{strotos_numerical_2008} claim that the harmonic law proves to conserve better diffusive fluxes at the interface.
Otherwise, little work has been found in the literature on the effect of mixture type on simulation results.
%----------------------------------
%
\subsection{Transport and regularisation} % reinitialisation % regularisation
An interface between two fluids may constanly move, consequently the level set transport allows
following the real interface.

%----------------------
\begin{figureth}
% textwidth 
{0.5}
%path 
{Misc/dummy.pdf}
% caption
{Simulation of moving level set interface (a) without then (b) with regularisation}
% label
\label{fig:tansport_no_reinit}
\end{figureth}
%-----------------------------------
Strong and weak form of transport \\
Numerical stability \\
%
%----------------------------------
\subsubsection{Convective reinitialisation and Hamilton-Jacobi equations}
\subsubsection{Geometric reinitialization}

\subsection{Interface Remeshing}
Importance when using a static level set and more importantly when LS is transported,
influence of mixing area \emph{thickness} and \emph{resolution} (i.e. nb of nodes with the area),
Isotropic or anisotropic ? the first is more important to composition calculation while the second
is more relevant if we mean do thermohydraulics without macrosegregation
