\clearpage
\section*{Résumé chapitre 1}

Dans ce premier chapitre d'introduction, on introduit les notions de base en solidification. 
On s'intéresse particulièrement aux notions de ségrégation se produisant à l'échelle des structures 
de solidification et que l'on appelle \emph{microségrégation}. Ce phénomène est directement lié à la 
différence se solubilité des espèces chimiques aux interfaces séparant les phases présentes dans 
l'alliage subissant la transformation.


Plusieurs facteurs, qu'ils soient relatifs au procédé de solidification ou inhérents aux phénomène de changements de phase,
peuvent les phases en mouvement. Ainsi, toute vitesse relative entre ces phases est à l'origine d'un transport des espèces chimiques
et donc une redistribution à l'échelle des pièces coulées. On parle alors de \emph{macroségrégation}.


Les ségrégations à l'échelle miroscopique peuvent être homogénéisées par le biais 
des traitements thermiques favorisant le transport par diffusion chimique.
Cependant, la macroségrégation est souvent irréversible et donc la cause de rebut de pièces produites. 
Ce défaut, rencontré dans des procédés de coulé continue ou coulée en lingot, est appréhendé par les
sidérurgistes qui investissent dans la recherche afin de mieux contrôler leur production.
La présente thèse s'incscrit dans le cadre de l'étude de la macroségrégation, notamment quand la cause en est
le mouvement de la phase liquide par convection thermosolutale et/ou par retrait à la solidification, tout en supposant
que les différentes phases solides sont fixes et rigides.


Dans le \textbf{chapitre 2}, nous présentons un modèle de solidification
basé sur la résolution des équations de conservation moyennées sur l'ensemble des phases, en utilisant la prise de moyenne
sur des volumes élémentaires représentatifs. Ces équations comportent la conservation de la masse, l'énergie, la masse des espèces
chimiques et la quantité de transport dans la phase liquide. Ce modèle est enrichi dans le \textbf{chapitre 3}, 
en proposant une nouvelle méthode de résolution de l'équation de la conservation d'énergie, avec la température comme variable
principale. Cette méthode utilise des propriétés à l'équilibre thermodynamique tabulés à partir d'une base de données dédiée, donnant accès à
des valeurs qui évoluent selon la composition de l'alliage. Dans le \textbf{chapitre 4}, on emploie cette méthode pour l'énergie,
avec les autres équations de conservation, pour prédire la ségrégation en canaux produite par convection thermosolulate, sans aucun 
changement de volume à la solidification. Comme cela nécessite d'avoir un suivi d'interface métal-air, une méthode implicite de suivi d'interface
est intégrée au modèle de solidification dans le \textbf{chapitre 5}. Le modèle final permet donc de prédire la macroségrgation produite
par retrait à la solidification et par convection thermosolutale avec suivi de front métal.  