\chapter*{Conclusion and Perspectives}
\addstarredchapter{Conclusion and Perspectives}

%\minitoc
%\newpage

% \section*{Conclusions}
The current thesis proposes a numerical model to predict macrosegregation in different contexts: without or
with overall metal volume change. The first case considers no average density change during solidification, 
assuming that the liquid and solid phases in the metal have the same density. 
On the other hand, the second case considers that the difference in metallic phase densities causes the average density to change, causing the metal's volume to
change concurrently. 
% \paragraph{Temperature solver}
In this thesis, we have introduced and validated a finite element method to solve energy conservation with phase change, based 
on thermodynamic data mapping and having the temperature as a main variable (\emph{Tsolver}). The algorithm proved to be faster for several computations
shown in chapters 3 and 4, when compared to the enthalpy-based method (\emph{Hsolver}). The approach is also well suited to predict macrosegregation
of both binary and multicomponent alloys. Some limitations are met nevertheless. It is important to have prior knowledge of composition variations
during solidification in order to adapt to limit the mapping size while also keeping fine composition and temperature steps.
Finer steps ensure more accurate transformation paths. We may also conclude that this thermodynamic mapping approach is still
to equilibrium assumptions (full equilibrium or solid-liquid interface equilibrium). \citet{tourret_multiple_2011} proposed
a similar solution but supports more than just a lever rule for microsegregation, by allowing input of diffusion coefficients.

% \paragraph{Channel segregation}
Using our energy solver, the Navier-Stokes solver and species conservation solver, we attempted modelling an experimental benchmark 
of indium-gallium solidification. Some of the experimental data was used to a get a closer numerical configuration to the experiment.
Two scales of modelling were considered. The macroscopic scale considers only average macroscopic conservation equations on a 
finite-element grid. No information on a lower scale can be determined by the model.
Later on,  


% \section*{Future Work}
% \comment{What did we miss in our models that can be potentially important for the coming years}