\clearpage
\section*{Résumé chapitre 5}

\begin{otherlanguage}{french}
{\small

Ce dernier chapitre est dédié à la prise en compte du retrait à la solidification à l'origine de la 
déformation de la surface libre métal-air, en présence des phénomènes de ségregation.
Pour ce faire, le modèle de solidification utilisé précédemment pour prédire la macroségrégation induite 
par convection thermosolutale en monodomaine, est enrichi par une
une méthode de suivi direct d'interface, la level set. %, déjà introduite dans le chapitre 2. 
Les équations du modèle sont alors reformulées dans un contexte 
eulérien multidomaine-multiphase, i.e. où deux domaines mutliphasés sont séparés
par une interface mobile, l'aspect mutliphase dans chaque domaine étant géré par la méthode de prise de moyenne volumique. 


Un premier cas d'application 1D est ensuite présenté. C'est un cas qui avait fait l'ojet de validation du \emph{Tsolver} dans le chapitre 3,
et qui est refait avec des masse volumiques solide et liquide différentes. Cette application simple permet toutefois de comprendre le phénomène
de ségrégation inverse résultant de l'écoulement du liquide dans la direction du front de solidification pour compenser la différence
des masse volumique des phases, ce qui enrichit en solutés la partie du métal en contact avec le refroidisseur. 
Ce phénomène est souvent observé aux surfaces des lingots en contact avec les moules. Des courbes de refroidissement ainsi qu'un bilan de conservation de masse de métal sont
présentés pour permettre de comprendre l'effet de l'introduction de la méthode level set sur la physique de la solidification.


La seconde application est aussi un cas de validation utilisé dans un chapitre précédent, issu d'une simulation sans retrait présente dans la litérature \citep{carozzani_direct_2013}.
Cependant, le but en est maintenant de tester la robustesse du modèle
en présence de convection naturelle, d'origine thermique dans l'air et thermosolutale dans le métal, avec suivi de déformation de l'interface par retrait à la solidification.
Nous utilisons une méthode de remaillage adaptatif basée sur la projection sur les arêtes \citep{coupez_solution_2013}, 
permettant d'avoir une résolution de maillage fine autour des zones d'intérêtes, notamment l'interface décrite par level set, le vecteur vitesse
et la concentration moyenne. Ce couplage de techniques numériques permet de déterminer à la fois, la retassure en surface du lingot et la macroségregation,
 présente sous forme de canaux.


Dans le dernier cas, il s'agit de la solidification d'une goutte d'acier en microgravité. 
Des essais expérimentaux de solidification déclenchée par contact avec un substrat sont présentés avec la forme finale de la goutte.
Pour pouvoir prédire la déformation de la goutte en présence de mésoségragations, nous considérons trois nuances issues du même alliage:
un binaire (\emph{b1Bin}) Fe-C, un ternaire (\emph{b1Tern}) Fe-C-Si et finalement un quaternaire (\emph{b1Quat}) Fe-C-Mn-Si.
D'abord, une étude paramétrique est faite en se servant de l'alliage binaire, dans le but de déterminer les paramètres optimaux de vitesse
de refroidissement et d'accélération gravitationnelle, pour se rapprocher du profil expérimental de la goutte déformée en fin de solidification.
Ensuite, nous simulons la solidification de chaque alliage en montrant la déformation finale ainsi que la distribution finale
des espèces chimiques.  

}
\end{otherlanguage}
