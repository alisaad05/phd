\clearpage
\section*{Résumé chapitre 3}

\begin{otherlanguage}{french}
{\small

Ce chapitre reprend les détails du solveur pour la conservation d'énergie avec changement de phase utilisé au CEMEF.
Celui-ci est basé sur une méthode enthalpique, dénommée \emph{Hsolver}, dont la variable principale est l'enthalpie
moyenne volumique du système, $\avg{\rho h}$. Ce solveur est aussi compatible avec des données tabulées provenant de bases de données
thermodynamiques, fournissant des valeurs précises pour chaque phase $\phi$ présente au moment de la transformation: 
fraction $\gphi$, composition intrinsèque $\wiphi$, enthalpie massique $\hphi$ et densité $\rphi$. 
Avec ces données, l'équation de conservation de l'énergie est résolue dans son état nonlinéaire 
provenant de la dépendance de $\rho h$ par rapport aux propriétés citées précédemment, sachant que celles-ci varient
aussi en fonction de la composition moyenne du volume élémentaire représentatif.


Cependant, la résolution \emph{Hsolver} nécessite une lourde recherche itérative à chaque pas de temps, consistant à convertir
$\avg{\rho h}$ en température $T$ pour évaluer le résidu du système nonlinéaire. Cette conversion est dénommée \emph{H2T} et elle est
compliquée du fait que les bases thermodynamiques fournissent la température comme donnée d'entrée, ce qui nous oblige de faire 
la recherche inverse itérative.


Dans ce chapitre, on propose de remplacer la conversion \emph{H2T} par une autre, \emph{T2H}. Comme son nom l'indique,
on part de l'idée que la température soit la variable principale du système et on devrait alors trouver l'enthalpie moyenne volumique
à chaque pas de temps. Avec ce changement, on propose donc une nouvelle formulation éléments finis, \emph{Tsolver}, mettant en évidence les principales
différences algorithmiques des deux résolutions.


Nous validons la formulation \emph{Tsolver} dans un cas purement diffusif et comparé à des calculs faits avec la méthode \emph{Hsolver}, 
ainsi qu'une comparaison avec une solution numérique obtenue par une méthode de suivi de front \citep{gandin_constrained_2000}.
Ensuite, nous montrons une application de solidification dirigée d'un système ternaire, \tern{Fe}{0.2}{C}{30}{Cr}, en régime diffusif.
Enfin, les limitations et les voies d'évolutions de la méthode \emph{Tsolver} avec les tabulations sont détaillées.
% avec une analyse de la ségrégation en peau et en coeur de la pièce.

}
\end{otherlanguage}