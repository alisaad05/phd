\chapter{Macrosegregation without solidification shrinkage}
\minitoc
\newpage

this chapter discusses the following points:
\begin{itemize}
\item General introduction about freckles
\item Couling the energy resolution from chapter 2 to fluid mechanics and solute balance
\item Application: Ternary solidification with freckles
\item Application: Macroscopic Freckle predicition: pure FE
\item Application Multiscale Freckle predicition: FE + grain structure (in which lies a part about nucleation-growth and how numerically 
		we reach a smaller scale, the scale of the grain boundaries)
\end{itemize}


\section{Application to multicomponent alloys}

The efficiency of the temperature-based resolution resides in its performance when combined with 
thermodynamic tabulations. A multicomponent alloy consists of at least two solute elements, and 
therefore the tabulation size increases, hence the number of search operations also increases. 
To demonstrate the speed-up ability of the temperature-based approach while predicting all phase 
transformations during macrosegregation, we consider the solidification of a ternary alloy, Fe–2 wt.\%C– 30 wt.\%Cr. 
As illustrated in \textbf{Figure 5a}, the alloy domain has a cylinder shape close to 3-inch height × 1-inch diameter. 
Exact values are reported in \textbf{Table 3} with all material properties, initial and boundary conditions, 
as well as numerical parameters for the simulations. The melt steel is initially at \SI{1395}{\udegC}. The 
temperature of the bottom surface is imposed with a constant decreasing rate of 0.1 K.s-1 starting 
with \SI{1380}{\udegC}, i.e. \SI{40}{\udegC} higher than the nominal liquidus temperature, as shown 
in \textbf{Figure 5b}. The other surfaces are kept adiabatic. The cylinder is held in a vertical position. 
In these conditions, and knowing that the carbon and chromium solutes have lightening effects on the liquid 
at nominal composition, the density inversion resulting from the composition gradient in the interdendritic 
liquid, may cause flow instability (segregation plumes) at the solidification front. While the selected alloy 
is a steel, this application is also representative of directional cooling in a single crystal casting, e.g. 
for nickel-base superalloys \cite{beckermann_development_2000}. \textbf{Figure 5c} also provides the 
transformation path of the alloy at nominal composition, i.e. assuming no macrosegregation and full 
thermodynamic equilibrium as computed with ThermoCalc and the TCFE6 database \cite{thermo-calc_andersson, tcfe6}. 
A total of 5 phases need to be handled, the characteristic temperature for their formation being reported 
in \textbf{Figure 5b}. 
\comment{Figure 5: Configurations for directional casting of (a) a 1 inch diameter × 3 inches height cylindrical
 domain for which (b) temperature-time conditions are imposed at its bottom surface. The alloy is Fe – 2 wt.\% C–
 30 wt.\% Cr, its computed transformation path [20], [21] at nominal composition being displayed in (c)}
 
\subsection{Tabulations}
Full thermodynamic equilibrium is considered in the present case. Due to macrosegregation, 
the average composition is expected to continuously vary in time and space during casting. 
Transformation paths are thus determined a priori for a set of average compositions around 
the nominal value. Hence, carbon content is arbitrarily varied in the interval [1.8 wt.\%, 2.2 wt.\%] 
while chromium content variation is in the interval [27 wt.\%, 33 wt.\%]. The offset of ±10\% with 
respect to the nominal composition value allows tabulating relatively small composition steps to 
ensure a good accuracy when compared to the corresponding ternary phase diagram. The average 
composition step is \bin{}{0.04}{} for carbon and \bin{}{0.6}{} for chromium, thus representing 2\% 
intervals with respect to the nominal composition. The temperature varies in the interval 
[\SI{100}{\udegC},\SI{1600}{\udegC}] by \SI{5}{\udegC} steps. For each triplet (carbon content 
in wt.\% C, \textbf{HERE} , chromium content in wt.\% Cr, \textbf{HERE}, temperature in \si{\udegK}) 
corresponds a phase fraction $g^\phi$ and a pair of intrinsic phase composition (\textbf{HERE}). For the 5 
phases listed in \textbf{Figure 5c} (LIQ$\equiv$liquid, BCC$\equiv$ferrite, FCC$\equiv$austenite, 
$\text{M}_7 \text{C}_3 \equiv$carbide, CEM$\equiv$cementite), the enthalpy $h^\phi$ and density $\rho^\phi$, are tabulated 
as functions of temperature and phase intrinsic composition. If this latter input lies between two tabulated 
values, a linear interpolation is performed to determine the output, i.e. phase enthalpy and density. With 
the advancement of solidification, the liquid is enriched with solute by macrosegregation, which enables new 
solidification paths. It means that the primary solidifying phase is not necessarily the same as when considering 
the nominal composition. For this reason, the tabulation approach is interesting inasmuch as it provides phase 
transformation paths and values of phase properties that are compatible with the system’s actual composition. 
\textbf{Figure 6} summarizes the tabulated thermodynamic data for two sets of average composition for the considered 
ternary system. Note that in the present test case, phase densities are taken constant ($\rhos=\rhol=$ \SI{6725}{\udensity}). 
Therefore they are not tabulated. With this assumption, no shrinkage occurs upon phase change.
\begin{table}[h]
\centering
\begin{tabular}{llll}
\hline  
\textbf{Parameter} & \textbf{Symbol} & \textbf{Value} & \textbf{Unit} \\
\hline 
Nominal composition 			& $\wC_0$ 				& 2 			& \si{\ucomposition} 	\\ 
                    			& $\wCR_0$ 				& 30 			& \si{\ucomposition} 	\\ 
Characteristic temperatures 	& $T_\text{top},T_\text{bottom}$ & \textbf{FIGURE} & \si{\udegC} \\ 
Phase fraction 					& $g^\phi$ 				& Tabulations 	& $-$ 					\\ 
Phase enthalpy 					& $\h^\phi$ 			& Tabulations 	& $-$ 					\\ 
Phase composition 				& $\wC^\phi$ 			& Tabulations 	& \si{\ucomposition}  	\\ 
                   				& $\wCR^\phi$ 			& Tabulations 	& \si{\ucomposition}  	\\ 
Diffusion coefficients 			& $\avg{D_\text{C}}^l$ 	& \num{15e-10} 	& \si{\udiffusivity}  	\\ 
                        		& $\avg{D_\text{Cr}}^l$ & \num{15e-10} 	& \si{\udiffusivity}  	\\ 
Dynamic viscosity  				& $\mul$ 				& \num{2e-3} 	& \si{\uviscosity}  	\\ 
Thermal expansion coefficient 	& \betaT 				& \num{8.96e-5} & \si{\ubetaT}  		\\ 
Solutal expansion coefficient 	& $\betaWlC$ 			& \num{1.54e-3} & \si{\ubetaWl}  		\\  
                              	& $\betaWlCR$ 			& \num{1.72e-2} & \si{\ubetaWl}  		\\ 
Thermal conductivity in the solid & $\ks$ 				& \num{40} 		& \si{\uconductivity}  	\\ 
Thermal conductivity in the liquid & $\kl$ 				& \num{28} 		& \si{\uconductivity}  	\\ 
Dendrite arm spacing 			& $\lambda$ 			& \num{60e-6} 	& \si{\metre}  			\\ 
Density 						& $\rref$ 				& \num{6725} 	& \si{\udensity}  		\\ 
\hline 
Initial temperature & $T_0$ & \num{1395}	& \si{\udegC}  \\ 
Ingot diameter 		&   	& \num{25e-3} 	& \si{\metre}  \\ 
Ingot length 		&   	& \num{75e-3} 	& \si{\metre}  \\ 
\hline 
FE mesh size 		&  		& \num{e-3} 	& \si{\metre}  \\ 
Time step 			& $\dt$ & \num{0.1} 	& \si{\second}  \\ 
Convergence criterion (residual) 	& $\varepsilon_R$ & \num{e-6} & $-$ \\ 
Convergence criterion (temperature) & $\varepsilon_T$ & \num{e-2} & \si{\udegK} \\ 
\hline 
\end{tabular} 
\caption{Parameters for solidification of alloy Fe – 2 wt.\% C – 30 wt.\% Cr }
\label{table:data_case_ternary}
\end{table}
