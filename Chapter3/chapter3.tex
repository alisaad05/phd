\chapter{Macrosegregation with incompressible fluid motion}
\begin{nolinkcolors} 
\minitoc
\end{nolinkcolors}
\newpage

this chapter discusses the following points:
\begin{itemize}
\item Review fluid mechanics briefly (MINI element and VMS and talk about the available solvers in Cimlib)
\item Give the VMS equations referring to Hachem article
\item Coupling the energy resolution from chapter 2 to fluid mechanics and solute balance
\item Make a transition to speak about freckles
\item Application: Ternary solidification with freckles
\item Application: Macroscopic Freckle prediction: pure FE
\item Application Multi-scale Freckle prediction: FE + grain structure (in which lies a part about nucleation-growth and how numerically 
		we reach a smaller scale, the scale of the grain boundaries)
\end{itemize}

\section{Introduction}
The previous chapter covered the energy solver ...

\section{Review fluid mechanics}
Review fluid mechanics briefly (MINI element and VMS and talk about the available solvers in Cimlib)

\section{VMS solver}
Give the VMS equations referring to Hachem article + weak form + stabilization

%taken from chapter 1
As for the last equation in \cref{tab:conservationeqs}, we follow the assumption of an incompressible liquid phase, which gives:
%---
\begin{align}
\label{eq:incompressiblity}
& \frac{\partial \r}{\partial t} = 0 
\end{align}

\section{Computational stability}
\subsection{CFL condition}
\subsection{Integration order}
Using P1 linear elements implies a P2 integration ? what are the advantages (time) and limitations ?




\section{Application to multicomponent alloys}
\comment{Here put the Fe C Cr computation results with  flow and macrosegregation and freckles
}

In the previous chapter, we have considered a static melt upon solidification of multicomponent alloy. However, in the real conditions the melt is in constant motion 
and knowing that the carbon and chromium solutes have lightening effects on the liquid 
at nominal composition, the density inversion resulting from the composition gradient in the interdendritic 
liquid, may cause flow instability (segregation plumes) at the solidification front. While the selected alloy 
is a steel, this application is also representative of directional cooling in a single crystal casting, e.g. 
for nickel-base superalloys \citep{beckermann_development_2000}. Solidification of this class of alloys is carefully
controlled so as to prevent any freckle-type defect to exist in the as-cast state.
In this section, we consider the same simulation parameters defined in \cref{table:data_case_ternary} as well the geometry and thermal boundary conditions
previously defined in \cref{fig:mutlicomponent_geobc}. Moreover, we solve the liquid momentum conservation equation, with non-slip boundary conditions
on all external sides of the cylinder.

\section{Macroscopic freckle prediction}
\comment{I should maybe mention that a constant gradient in the coming simulation is a one big difference compared to the previous FeCrC simulation}
\subsection{Introduction}
\comment{ I have shown the results of multicomponent alloy solidification, where we saw freckles. So let me do an introduction
about freckles, the need to prevent such defect from forming (superalloys, critical use in turbine blades}

\subsection{Experimental work}
\comment{ Then introduce the
experimental benchmark of Natalia and Sven from the article to show that there's an effort to understand, characterize and prevent 
if possible freckle formation. Show figures and some experimental results but quickly, no need to put many things and distract the reader }

\subsection{Macroscopic scale simulations}
\comment{ Introduce the FE model and algorithm then show pure FE RESULTS}
\subsubsection{Discussion}
In the literature, many successful attempts have been made to predict freckles since (for example CITE fellicelli, poireau) ... until coming to kohler thesis results in 2008. These authors tackled the problem from an qualitative perspective. To our knowledge, the only close-to-quantitative work in solidification literature was done by \citet{ramirez_evaluation_2003}, who attempted to draw a correlation (freckling criterion) between the process parameters and the occurence of freckles, (without any size or shape constraints, i.e. any flow instability that may appear and form the smallest freckle is considered). To accomplish this, they took a number of experiments done independently by \citet{pollock_breakdown_1996} and \citet{auburtin_freckle_2000} where the casting parameters vary one at a time: casting speed (R), thermal gradient (G), angle ($\theta$) with respect to vertical orientation and nominal composition ($\avg{w_0}$), giving a database for 6 different superalloys. The experimental results were compared to a modified Rayleigh number that accounts for the various parameters. It allowed them to define a threshold for freckle formation in Nickel-base superalloys, as well as Pb-Sn alloys.
\comment{ They have also investigated Pb-Sn alloys, check }
Other contributions by \citet{yuan_new_2012} and  \citet{karagadde_3-d_2014} used a medium scale model to compare the simulated formation of freckles with the results obtained by \citet{shevchenko_chimney_2013} (explain at bit more)
However, all simulations show common traits in their predictions: (some words about the freckle dimensions, shape, intensity). These properties do not exactly meet with the experimental observations, just like in the In-Ga experiment. We think that the hydrodynamics scale 
at which freckles are born, is much smaller than the FEM scale. Since the relevant physics are not solved, even the finest FE mesh will not
be enough to see the exact grain boundaries. (now it is time to do transition to CAFE)



\section{Meso-Macro freckle prediction}

parachute article :)
















