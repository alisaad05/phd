%%------------------------ FONTS -------------------------------
\usepackage{fourier} % Utopia font-typesetting including mathematical formula compatible with newer TeX-Distributions (>2010)
%\usepackage{utopia} % on older systems -> use this package instead of fourier in combination with mathdesign for better looking results
%\usepackage[adobe-utopia]{mathdesign}

%%------------------------ ACRONYMS -------------------------------
\usepackage[printonlyused,smaller]{acronym}
\renewcommand{\bflabel}[1]{{#1}\hfill} % fix the list of acronyms

%%------------------------ BIBLIOGRAPHY -------------------------------
\usepackage[square,numbers]{natbib} 

%%------------------------ LANGUAGES-FRENCH ACCENTS -------------------------------
\usepackage[utf8]{inputenc}

%%------------------------ LAYOUT -------------------------------
%% --- Margins
\setlength{\textwidth}{146.8mm} % = 210mm - 37mm - 26.2mm
\setlength{\oddsidemargin}{11.6mm} % 37mm - 1in (from hoffset)
\setlength{\evensidemargin}{0.8mm} % = 26.2mm - 1in (from hoffset)
\setlength{\topmargin}{-2.2mm} % = 0mm -1in + 23.2mm 
\setlength{\textheight}{221.9mm} % = 297mm -29.5mm -31.6mm - 14mm (12 to accomodate footline with pagenumber)
\setlength{\headheight}{14pt}
%% --- Paragraph spacing
\setlength{\parindent}{0pt}
%% --- Interline spacing
\usepackage{setspace} % increase interline spacing slightly
\setstretch{1.1}

%%------------------------ Captions  -------------------------------

% Captions: This makes captions of figures use a boldfaced small font. 
%\RequirePackage[small,bf]{caption}
%\RequirePackage[labelsep=space,tableposition=top]{caption} 
%\renewcommand{\figurename}{Fig.} %to support older versions of captions.sty

%%-----------------------  Footnote Header formatting -------------------------------
% From EPFL template
%\usepackage[perpage]{footmisc} %Range of footnote options 

%\usepackage{fancyhdr}
%\renewcommand{\sectionmark}[1]{\markright{\thesection\ #1}}
%\pagestyle{fancy}
%	\fancyhf{}
%	\renewcommand{\headrulewidth}{0.4pt}
%	\renewcommand{\footrulewidth}{0pt}
%	\fancyhead[OR]{\bfseries \nouppercase{\rightmark}}
%	\fancyhead[EL]{\bfseries \nouppercase{\leftmark}}
%	\fancyfoot[EL,OR]{\thepage}
%\fancypagestyle{plain}{
%	\fancyhf{}
%	\renewcommand{\headrulewidth}{0pt}
%	\renewcommand{\footrulewidth}{0pt}
%	\fancyfoot[EL,OR]{\thepage}}
%\fancypagestyle{addpagenumbersforpdfimports}{
%	\fancyhead{}
%	\renewcommand{\headrulewidth}{0pt}
%	\fancyfoot{}
%	\fancyfoot[RO,LE]{\thepage}
%}

%%------------------------ Figures  -------------------------------
%\usepackage{rotating}
%\usepackage{wrapfig}
%\usepackage{float}
%\usepackage{subfig} %note: subfig must be included after the `caption` package. 

% \usepackage{graphicx,xcolor}
\usepackage[dvipsnames]{xcolor}
\definecolor{webgreen}{rgb}{0,.5,0}
\definecolor{webbrown}{rgb}{.6,0,0}
\definecolor{Maroon}{cmyk}{0, 0.87, 0.68, 0.32}
\definecolor{RoyalBlue}{cmyk}{1, 0.50, 0, 0}

%\graphicspath{{images/}}
\makeatletter
\setlength{\@fptop}{0pt}  % for aligning all floating figures/tables etc... to the top margin
\makeatother

%%------------------------ Tables  -------------------------------
%\usepackage{longtable}
%\usepackage{multicol}
%\usepackage{multirow}
%\usepackage{tabularx}

%\usepackage{booktabs}

%%------------------------ Math and SI Units  -------------------------------
\usepackage{amsfonts}
\usepackage{amsmath}
\usepackage{amssymb}
\usepackage{siunitx} % http://ftp.oleane.net/pub/CTAN/macros/latex/contrib/siunitx/siunitx.pdf
% \num{.3e45}   ,  \num{3.45d-4}   , \numlist{10;30;50;70} ,  \numrange{10}{30} , 

%%------------------------ Cross References  -------------------------------
% --- Hyperref and Options
\usepackage{hyperref}
\hypersetup{%
    %draft,	% = no hyperlinking at all (useful in b/w printouts)
    colorlinks=true, linktocpage=true, pdfstartpage=3, pdfstartview=FitV,%
    % uncomment the following line if you want to have black links (e.g., for printing)
    %colorlinks=false, linktocpage=false, pdfborder={0 0 0}, pdfstartpage=3, pdfstartview=FitV,% 
    breaklinks=true, pdfpagemode=UseNone, pageanchor=true, pdfpagemode=UseOutlines,%
    plainpages=false, bookmarksnumbered, bookmarksopen=true, bookmarksopenlevel=1,%
    hypertexnames=true, pdfhighlight=/O,%nesting=true,%frenchlinks,%
    urlcolor=webbrown, %webbrown, 
    linkcolor=RoyalBlue, %RoyalBlue, 
    citecolor=webgreen, %webgreen, 
    %pagecolor=RoyalBlue,%
    %urlcolor=Black, linkcolor=Black, citecolor=Black, %pagecolor=Black,%
    pdftitle={\myTitle},%
    pdfauthor={\myName, \myUni, \myFaculty},%
    pdfsubject={},%
    pdfkeywords={},%
    pdfcreator={pdfLaTeX},%
    pdfproducer={LaTeX with hyperref and classicthesis}%
}

%% --- cleveref and Options
\usepackage{cleveref}
% cleverref should always be invoked AFTER hyperref

%% --- beckref and Options
\newcommand{\backrefnotcitedstring}{\relax}%(Not cited.)
\newcommand{\backrefcitedsinglestring}[1]{(Cited on page~#1.)}
\newcommand{\backrefcitedmultistring}[1]{(Cited on pages~#1.)}
\ifthenelse{\boolean{enable_backrefs}}%
{%
		\usepackage[hyperpageref]{backref}  % to be loaded after hyperref package
		   \renewcommand{\backreftwosep}{ and~} % separate 2 pages
		   \renewcommand{\backreflastsep}{, and~} % separate last of longer list
		   \renewcommand*{\backref}[1]{}  % disable standard
		   \renewcommand*{\backrefalt}[4]{% detailed backref
		      \ifcase #1 %
		         \backrefnotcitedstring%
		      \or%
		         \backrefcitedsinglestring{#2}%
		      \else%
		         \backrefcitedmultistring{#2}%
		      \fi}%
}%{\relax}    

%%--------------------- TOC --------
%\setcounter{secnumdepth}{2}
%\setcounter{tocdepth}{2}

\usepackage{minitoc} % use like this : \minitoc 
\setcounter{minitocdepth}{2}
\setlength{\mtcindent}{24pt} 
\renewcommand{\mtcfont}{\small\rm}
\renewcommand{\mtcSfont}{\small\bf}


%%------------------------ NOMENCLATURE / GLOSSARY -------------------------------
% http://www.latex-community.org/know-how/latex/55-latex-general/263-glossaries-nomenclature-lists-of-symbols-and-acronyms
%\usepackage[toc]{glossaries}
% use it like: \newglossaryentry{label}{definition}
%\usepackage[xindy,toc]{glossaries}  
%\makeglossaries 


%%------------------------ Names of sections for figures, nomenclature ...  -------------------------------
% To change the name of the Nomenclature section, uncomment the following line
%\renewcommand{\contentsname}{My Table of Contents}
%\renewcommand{\listfigurename}{My List of Figures}
%\renewcommand{\listtablename}{My List of Tables}
%\renewcommand{\nomname}{Symbols}

%%------------------------ Appendix -------------------------------
\usepackage[titletoc]{appendix}
% The default value of both \appendixtocname and \appendixpagename is `Appendices'. These names can all be changed via: 
%\renewcommand{\appendixtocname}{List of appendices}
\renewcommand{\appendixname}{Appendicitis}

%%------------------------ MISC PACKAGES -------------------------------
\usepackage{url}
\usepackage{lipsum}
%\usepackage[final]{pdfpages} %\includepdf[options]{filename}
%\usepackage{algpseudocode} 

\usepackage{lineno}
\ifthenelse{\boolean{enable_line_numbers}}%
{%
	%\linenumbers
	\pagewiselinenumbers % requires compiling AT LEAST twice
}{\relax}    

% \usepackage{todonotes}
\usepackage{todonotes}
\newcommand{\comment}[2][]{\todo[backgroundcolor=yellow!50!white, caption={#2}, inline, size=\small, #1]{\renewcommand{\baselinestretch}{0.5}\selectfont#2\par}}

\usepackage{microtype}