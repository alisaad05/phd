\chapter*{General Introduction}
\addstarredchapter{Introduction}

%\minitoc
%\newpage

Metallurgical processes have known a great evolution during the last 60 years. The advancement is attributed to 
research disciplines, like physical metallurgy, which investigated a great deal of solidification-related phenomena.
Presently, research in metallurgy focuses on a deeper understanding of the connection between the different scales involved.

From the nucleation theory to the mechanical behavior of metals, an chain of intricate phenomena occur in a such a way
to create defects in the final product. This has been seen in casting processes like continuous casting and ingot
casting. Suface and volume porosity, hot tearing and composition heterogeneities are known defects to the casting community.
As far as the current project is concerned, the last defect, widely known as macrosegregation, is the subject of our interest.

\section{Casting processes}

\section*{Defects}
\comment{Worth checking notes from the Ecole Thématique CNRS oléron (Check Mail Draft)}
Talk about the cost of recycling defect steel products, giving a good example
\subsection*{Hot tearing}

\subsection*{Porosity}
\url{http://www.afsinc.org/about/content.cfm?ItemNumber=6933} \\
\url{http://en.wikipedia.org/wiki/Casting_defect} \\
Porosity is simply defined as a void formed within the casting or at the outter surface.
In general, it results from the presence of gas phase in the melt. In some cases, fast
solidification conditions may result in a depressurization 
\comment{Here goes the porosity figure}

\subsection*{Freckles and segregated channels} 
The intersection between the effects of microsegregation and gravity field is the origin of freckles. They may appear 
randomly (\blue{giame 1970 ? or earlier} ) in casting processes, but are very specific to directional chill casting. A mass of solute 
segregates on a medium scale between microsegregation and macrosegregation, hence forming a heterogeneous composition, 
on the ingot's surface, as well as in the volume (\blue{thèse kohler exemple}).
\subsection*{Macrosegregation}
In principle macrosegreation stems from the very same causes of freckles and segregated channels. However, this defect is characterized by a composition heterogeneity on much larger scale (\red{give scale}). 
\comment{put the Nancy ingot photo and comment it. Also put a photo either from
Beckermann review 2000 or the big review article by rappaz, rohit, beckermann ...}



\section*{Industrial Worries}
\textbf{Production}
\begin{itemize}
\item Talk about total steel production, variations over the last few decades
\item Quality constraints for many applications that require steel like construction, nuclear engines ? 
\item Difficulties to meet these constraints and what are the present solutions
\end{itemize}
\textbf{Research and Simulation}
\begin{itemize}
\item Need for software handling multicomponent alloys
\item Need for software handling finite diffusion in the solid
\item Need for realistic alloy properties (not only constants)
\item Need for handling moulds along with volume change (creating thermal resistances)
\end{itemize}
\comment{Worth discussing Isabelle Poitraut and  David Cardinaux -  and Claudine Allentin (respo comm Arcelor Dunkerque, search for mail)}


\section*{CCEMLCC contribution}
\begin{itemize}
\item some words about this ESA project
\item in what ways does this project tries to alleviate the aforementionned problems ?
\item academic and industrial partners and how does each of them contribute actually
\item mention \emph{Thercast} as the final developped code destination ?
\end{itemize}

%\section*{Trying \emph{SIUNITX}}
%Here i wana test the SI units package via the commands \num{.3e45} and the unit \si{\kilo\metre}
%then i wanted to see if we combine both via \SI{.3e45}{\kilo\metre} then finally my personal command
%\SI{231e-4}{\uacceleration} \\
%\SI{231e-4}{\ucomposition} C \\
%\SI{231e-4}{\uvelocity} \\
%\SI{231e-4}{\uconductivity} \\
%\SI{231e-4}{\umasscapacity} \\
%\SI{231e-4}{\uvolumecapacity} \\
%Will it work \num{.3e45}   ,  \num{3.45d-4}   , \numlist{10;30;50;70} ,  \numrange{10}{30}
