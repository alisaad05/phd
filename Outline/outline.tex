\documentclass[10pt,a4paper]{article}
\usepackage{fourier}
\usepackage[utf8]{inputenc}
%%------------------------ BEGIN  -------------------------------
\begin{document}

\section*{Thesis outline}

\begin{enumerate}
\item Biblio
\begin{itemize}
\item may be changed to partial biblio as each chapter intro, depending on the total number of chapters
\end{itemize}

\item Energy resolution temperature with tabulations
\begin{itemize}
\item basically similar to the HvsT article’s structure
\end{itemize}

\item Couple with fluid mechanics and solute balance: no shrinkage
\begin{itemize}
\item Talk about freckles in general (maybe talk about Fe-C-Cr freckles from HvsT article)
\item	Freckles’ Experiment 
\begin{itemize}
\item FE simulation without shrinkage
\item CAFE simulation without shrinkage
\end{itemize}

\item discussions (+ref to article freckles ?)
\end{itemize}

\item Couple with fluid mechanics and solute balance: with shrinkage
\begin{itemize}
\item density tabulation
\item SMACS FE simulation with shrinkage
\item with CAFE ?? (if we want to keep a consistent layout of ideas)
\end{itemize}

\item Shrinkage with a deformable solid ?
\item Application to TEXUS \\
\end{enumerate}

PS: if I do not have enough time to go Chapter 5 (solid velocity >0), I will keep chapters 3 and 4 as they are (no merge), therefore I will speak about Thercast in chapter 4. Application to Texus will be chapter 5 in this case \\
In contrast, if we can do the deformable solid case, then I will merge chapters 3 and 4 in one chapter (called chapter 3), and therefore chapter 4 will contain the coupling with Thercast and deformable solid with shrinkage , and chapter 5 will talk about the comparison with TEXUS \\
To sum up, we should have maximum 5 chapters without the conclusion chapter.

\end{document}